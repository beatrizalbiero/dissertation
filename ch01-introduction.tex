\chapter{Aprendendo Verbos Irregulares}
\label{ch:01}

Este capítulo será dedicado primeiramente a uma apresentação a respeito da origem da discussão sobre o processo de aprendizado de flexão de verbos irregulares.  Assim, construiremos uma visão geral do problema e apresentaremos as principais hipóteses e pesquisas envolvidas. Em seguida, discutiremos as motivações que levaram à produção desta pesquisa. Nesse sentido, veremos que, por um lado, há um interesse de caráter linguístico, motivado por uma maior complexidade da língua portuguesa se comparada ao inglês. Por outro, veremos também que há um interesse de cunho computacional, motivado pelo surgimento de novos recursos de modelagem. Por último, veremos uma seção destinada à delimitação do escopo da pesquisa.

\section{Contextualização}
A questão do aprendizado infantil, referente ao processo de flexão de verbos irregulares na língua inglesa, está certamente entre um dos temas de debate mais controversos  
do campo da linguística (\cite{chomsky:1968}, \cite{Pinker:1999}, \cite{Pinker:1988}, \cite{Albright2003RulesVA}, \cite{kirov:2018}). O cerne do debate está na exata caracterização dos mecanismos que possibilitam que um falante seja capaz de relacionar um verbo na forma não flexionada (\textit{walk}, por exemplo) à sua forma flexionada no \textit{Simple Past} (\textit{walked}).

Os verbos no tempo passado do inglês podem ser subdivididos em uma variedade de famílias. Um primeiro grupo é a forma \textit{regular}, cuja forma corresponde à aplicação da regra \textit{\text{radical} + ed} (como no exemplo visto do verbo \textit{to walk}). 
Dentre os verbos irregulares, estes podem ser considerados supletivos, ou seja, possuem um processo de flexão único e sem regra aparente, como por exemplo \textit{go} $\rightarrow$ \textit{went}, ou podem se conglomerar seguindo padrões fonéticos de flexão similares (\cite{Nelson:2010}):

\begin{enumerate}
    \item blow – blew, grow – grew, know – knew, throw – threw
    \item bear – bore, swear – swore, tear – tore, wear – wore
    \item drink – drank, shrink – shrank, sink – sank, stink – stank 
\end{enumerate}

É possível pensar que o aprendizado de tais padrões dependeria de uma memorização caso a caso. No entanto, a pesquisa de \cite{Bybee:1983} mostra um estudo psicolinguístico em que indivíduos são apresentados a diversos verbos inventados (hipoteticamente em uma forma não flexionada). A pesquisa revelou que, ao invés de aplicarem sistematicamente a regra regular (verbo + \textit{ed}), os indivíduos apresentaram tendências à alocação de alguns verbos em alguns subgrupos irregulares. Por exemplo, para o verbo inventado “\textit{spling}”, a maioria dos indivíduos optou pela forma “\textit{splang}”  ou “\textit{splung}”. Este exemplo contradiz a ideia de que os falantes poderiam estar apenas reproduzindo formas memorizadas e sugere que eles estejam ativamente identificando padrões. Além disso, possuem uma intuição natural sobre a adequabilidade da alocação de um verbo a um grupo de verbos ou a outro. 

A partir do exemplo dado, é razoável deduzir que a motivação por de trás de tais tendências ocorra a partir das similaridades entre as unidades fonéticas dos verbos inventados e os verbos reais que já apresentam uma flexão de caráter irregular. Entretanto, as circunstâncias que levam à aquisição dessa \textit{intuição} linguística são indeterminadas. Por um lado, faz sentido dizer que para que um ser humano seja capaz de introduzir-se ao mundo dos falantes, é necessário que ele seja dotado de algumas pré-disposições para tal, caso contrário seria possível observar ou ensinar essa forma de comunicação para outras espécies. Em contrapartida, estudos mostram que crianças privadas do contato com uma sociedade falante se tornam permanentemente incapazes de dominar integralmente a gramática de uma língua (\cite{Pinker:languageinstinct}), o que nos leva a concluir que a experiência das crianças com a sociedade, assim como as suas próprias pré-disposições genéticas são ambas parcialmente responsáveis pelo processo de desenvolvimento da linguagem. A dificuldade, está portanto, na tentativa de se quantificar, delimitar e apontar os conhecimentos adquiridos a partir do contato cultural, bem como os conhecimentos linguísticos ditos \textit{inatos}. É, portanto, em torno desta questão que tem início o debate a respeito do aprendizado dos verbos irregulares da língua inglesa.

De um lado do debate, encontra-se a teoria da Fonologia Gerativa de Chomsky e Halle (\citeyear{chomsky:1968}). Nesta teoria, os indivíduos seriam portadores de um dispositivo de aquisição de linguagem (\textit{LAD} - Language Acquisition Device) responsável pela \textit{formulação} e \textit{manipulação} de estruturas fonológicas abstratas em um sistema de regras. De modo simplificado, a teoria propõe que o falante seja capaz de identificar e formular regras intuitivamente para dar conta do aprendizado das formas irregulares da língua. Um exemplo disso é a família dos verbos terminados em “-ind”.

\begin{center}
bind – bound, find – found, grind – ground, wind – wound
\end{center}

Vemos que, de modo simplificado, pode-se propor uma regra como:

\begin{center}
a\textsci $\rightarrow$ a\textupsilon / \textbf{X}  \_\_nd]+past
\end{center}

A regra proposta sugere que o segmento [a\textsci] se transforma em [a\textupsilon] quando terminado em [nd] e flexionado para o passado. O símbolo \_\_ representa o local onde ocorre tal transformação e \textbf{X} representa uma unidade fonológica arbitrária. 

Em outras palavras, pode-se dizer que o conhecimento dito \textit{inato} defendido por Chomsky e Halle refere-se a uma certa capacidade cognitiva de formulação de regras a partir da identificação de alguns elementos fundamentais (como por exemplo os elementos apontados na regra proposta). Uma estrutura como essa permitiria ao falante construir generalizações e, eventualmente, abstrair as regras fonológicas de sua língua. \\

Do outro lado do debate, os pesquisadores \cite{rumelhart:1986} confrontam a teoria anterior ao argumentar que comportamentos de caráter regrado podem ser reproduzidos por mecanismos que não dependam de nenhuma manipulação simbólica. Ao invés disso, os pesquisadores sugerem que os mecanismos envolvidos no processo de flexão verbal possam ser construídos de tal forma que a sua performance possa ser descrita através de regras, mas que as regras em si não estejam representadas explicitamente em nenhuma parte do processo. Para sustentar essa ideia, \cite{rumelhart:1986} apresentam um modelo computacional baseado em padrões associativos que não fazem uso de construções com regras desse tipo. Posteriormente, o modelo construído foi fundamental para o surgimento de uma nova escola dentro das ciências cognitivas: o conexionismo.\\

\definecolor{blue}{RGB}{159, 192, 176}
\definecolor{green}{RGB}{160, 227, 127}
\definecolor{orange}{RGB}{243, 188, 125}
\definecolor{red}{RGB}{253, 123, 84}
\definecolor{nephritis}{RGB}{39, 174, 96}
\definecolor{emerald}{RGB}{46, 204, 113}
\definecolor{turquoise}{RGB}{39, 174, 96}
\definecolor{green-sea}{RGB}{22, 160, 133}
\definecolor{purple}{RGB}{181, 124, 215}
% Tikzstyles for Computation Graphs

% nodes
\tikzstyle{noop} = [circle, draw=none, fill=red, minimum size = 10pt]
\tikzstyle{op} = [circle, draw=red, line width=1.5pt, fill=red!70, text=black, text centered, font=\bf \normalsize, minimum size = 25pt]

\tikzstyle{opintense} = [circle, draw=red, line width=1.5pt, fill=red!150, text=black, text centered, font=\bf \normalsize, minimum size = 25pt]


%new style
\tikzstyle{gp} = [circle, draw=red, line width=4pt, text=black, text centered, font=\bf \normalsize, minimum size = 4.cm]

\tikzstyle{box} = [rectangle, draw=red, line width=1.5pt, fill=red!70, text=black, align=center, font=\bf \normalsize, minimum size = 45pt]

\tikzstyle{box2} = [rectangle, draw=black, line width=0.9pt, text=black, align=center, font=\bf \normalsize, minimum size = 20pt]

\tikzstyle{box3} = [rectangle, draw=black, line width=0.9pt, fill=black, text=black, align=center, font=\bf \normalsize, minimum size = 20pt]

\tikzstyle{state} = [circle, draw=blue, line width=1.5pt, fill=blue!70, text=black, text centered, font=\bf \normalsize, minimum size = 25pt]

\tikzstyle{output} = [circle, draw=purple, line width=1.5pt, fill=purple!70, text=black, text centered, font=\bf \normalsize, minimum size = 25pt]


\tikzstyle{gradient} = [circle, draw=nephritis, line width=1.5pt, fill=nephritis!60, text=black, text centered, font=\bf \normalsize, minimum size = 25pt]
\tikzstyle{textonly} = [draw=none, fill=none, text centered, font=\bf \normalsize]
\tikzstyle{boxtextonly} = [draw=none, fill=none, align=center, font=\bf \normalsize]

\tikzstyle{normal} = [circle, draw=black, line width=1.0pt, fill=none, text=black, text centered, font=\bf \normalsize, minimum size = 20pt]


% edges
\tikzstyle{tedge}  = [draw, thick, >=latex, ->]
\tikzstyle{tedge_dashed}  = [draw, thick, >=latex, ->, dashed]
\tikzstyle{nedge}  = [draw, thick, >=latex]
\tikzstyle{nedge_dashed}  = [draw, thick, >=latex, dashed]


% namedscope
\tikzstyle{namedscope} = [circle, draw=orange, line width=1.5pt, fill=orange!60, align=center, inner sep=0pt]
\begin{figure}[ht!]
\centering

\scalebox{1.0}{
\begin{tikzpicture}[auto]

% operations =========
% phon features 1
\node[textonly] (1pho1) {int-vogal-int};

% Legenda
\node[textonly, above=10pt of 1pho1] (leg1) {Unidades de Input};


% FNN input
\node[normal, right=5pt of 1pho1] (x1) {};
\node[normal, below=25pt of x1] (x2) {};
\node[normal, below=25pt of x2] (x3) {};
\node[normal, below=25pt of x3] (x4) {};
\node[normal, below=25pt of x4] (x5) {};
\node[normal, below=25pt of x5] (x6) {};

% FNN output
\node[normal, right=45pt of x1] (y1) {};
\node[normal, right=45pt of x2] (y2) {};
\node[normal, right=45pt of x3] (y3) {};
\node[normal, right=45pt of x4] (y4) {};
\node[normal, right=45pt of x5] (y5) {};
\node[normal, right=45pt of x6] (y6) {};

% phon features 2
\node[textonly, right=5pt of y1] (2pho1) {int-vogal-int};
\node[textonly, above=10pt of 2pho1] (leg2) {Unidades de Output};
\node[textonly, left=25pt of x2] (1pho2) {anterior-nasal-posterior};
\node[textonly, right=25pt of y2] (2pho2) {anterior-nasal-posterior};
\node[textonly, left=25pt of x3] (3pho1) {...};
\node[textonly, right=25pt of y3] (1pho3) {...};
\node[textonly, left=25pt of x4] (4pho1) {nasal-cont-ocl};
\node[textonly, right=25pt of y4] (1pho4) {nasal-cont-ocl};
\node[textonly, left=25pt of x5] (5pho1) {médio-cont-baixa};
\node[textonly, right=25pt of y5] (1pho5) {médio-cont-baixa};
\node[textonly, left=25pt of x6] (6pho1) {vogal-fric-\#};
\node[textonly, right=25pt of y6] (1pho6) {vogal-fric-\#};
% edges FNN
\path[nedge] (x1) -- (y1);
\path[nedge] (x1) -- (y2);
\path[nedge] (x1) -- (y3);
\path[nedge] (x1) -- (y4);
\path[nedge] (x1) -- (y5);
\path[nedge] (x1) -- (y6);
\path[nedge] (x2) -- (y1);
\path[nedge] (x2) -- (y2);
\path[nedge] (x2) -- (y3);
\path[nedge] (x2) -- (y4);
\path[nedge] (x2) -- (y5);
\path[nedge] (x2) -- (y6);
\path[nedge] (x3) -- (y1);
\path[nedge] (x3) -- (y2);
\path[nedge] (x3) -- (y3);
\path[nedge] (x3) -- (y4);
\path[nedge] (x3) -- (y5);
\path[nedge] (x3) -- (y6);
\path[nedge] (x4) -- (y1);
\path[nedge] (x4) -- (y2);
\path[nedge] (x4) -- (y3);
\path[nedge] (x4) -- (y4);
\path[nedge] (x4) -- (y5);
\path[nedge] (x4) -- (y6);
\path[nedge] (x5) -- (y1);
\path[nedge] (x5) -- (y2);
\path[nedge] (x5) -- (y3);
\path[nedge] (x5) -- (y4);
\path[nedge] (x5) -- (y5);
\path[nedge] (x5) -- (y6);
\path[nedge] (x6) -- (y1);
\path[nedge] (x6) -- (y2);
\path[nedge] (x6) -- (y3);
\path[nedge] (x6) -- (y4);
\path[nedge] (x6) -- (y5);
\path[nedge] (x6) -- (y6);


\end{tikzpicture}
}\caption{Esquema da rede neural utilizada pelos pesquisadores Rumelhart e McClelland} 
\label{fig:esquemafdd}
\end{figure}


O modelo desenvolvido foi criado por analogia à estrutura em que se relacionam os neurônios no cérebro. Ele é composto basicamente por uma rede artificial de nódulos interconectados paralelamente (Fig. \ref{fig:esquemafdd}).

A primeira camada de nódulos do modelo é responsável por receber os dados de entrada (os \textit{inputs}), que são os dados referentes aos traços fonéticos distintivos que caracterizam os sons de um verbo não flexionado. Traços fonéticos podem ser caracterizados como propriedades distintivas das unidades fônicas (\cite{paraconhecer:2015}). Tais propriedades podem ser baseadas em critérios acústicos, articulatórios ou perceptuais. Na figura, cada nódulo é apresentado ao lado de uma sequência de três traços. O primeiro nódulo, por exemplo, refere-se à sequência \textbf{ocl-vogal-ocl}. Nesse caso, \textbf{ocl} (uma abreviação para \textit{oclusiva}) indica uma propriedade comum entre algumas consoantes, referente à interrupção do fluxo de ar (como no fone [k], por exemplo). Na figura temos ainda \textbf{fric} para fricativas, \textbf{vogal} para vogais, \textbf{nasal} para nasalidade, locais da execução (\textbf{anterior} e \textbf{posterior}), traços de corpo da língua (\textbf{média}, \textbf{baixa}), entre outros. Ainda sobre a camada de \textit{input}, é possível observar que cada nódulo é representado por uma tríade de traços fonéticos. Esta foi uma solução encontrada pelos autores para realizar o mapeamento entre os traços dos verbos da forma não flexionada para o \textit{Past Simple}. Os \textit{inputs} são estruturados dessa forma para contornar a dificuldade de inserção de dados de natureza sequencial e de tamanho variável (como é o caso de um verbo - composto por uma sequência de sons). Cada tríade é uma associação de três traços, cada um referente a um fone. Por exemplo, para o verbo \textit{came} (transcrito em forma fonética pelos autores como \textit{/kAm/}), temos que cada um dos fones possui múltiplos traços. O fone [\textit{k}], por exemplo, é uma consoante oclusiva, surda e anterior. Os fones subsequentes também são constituídos a partir de seus respectivos traços fonéticos, de modo que podemos representar o verbo como uma sequência de listas de traços fonéticos (Tabela \ref{tab:trigrams}). O tema dos traços fonéticos utilizados, bem como todo o esquema de pré-processamento utilizado pelos autores \cite{rumelhart:1986} será abordado em maior profundidade no Cap. \ref{ch:02}.

\begin{table}[H]
\begin{center}
\begin{tabular}{ccc}
k                    & A                    & m                    \\ \hline
surda                & longa                & nasal                \\
interrompida         & vogal                & interrompida         \\
anterior             & baixa                & posterior            \\
consoante            & posterior            & consoante            \\
\multicolumn{1}{l}{} & \multicolumn{1}{l}{} & \multicolumn{1}{l}{}
\end{tabular}
\caption{Tríades de Traços Fonéticos Utilizados nos \textit{Inputs} do modelo de Rumelhart e McClelland}
\label{tab:trigrams}
\end{center}
\end{table}

Voltando à Fig. \ref{fig:esquemafdd}, vemos após a camada de \textit{input} uma rede de conexões. Cada conexão, por sua vez, possui um \textit{peso}. Esses pesos irão funcionar como uma espécie de filtro dos \textit{inputs}, fazendo com que pesos com valores maiores passem as informações adiante com mais efeito (ou força), e os pesos menores, com menos. A segunda camada de nódulos da Fig.\ref{fig:esquemafdd} é uma camada de resposta (conhecida como camada de \textit{output}) que tem como objetivo retornar dados referentes aos traços que caracterizam os sons do mesmo verbo fornecido no \textit{input}, porém no tempo passado. 

Durante o processo de aprendizado do modelo, os dados de saída da camada de \textit{output} deverão ser então comparados à forma correta do verbo no tempo passado, através de uma espécie de gabarito, usualmente conhecido como \textit{alvo} ou também \textit{target} (Fig.\ref{fig:gabarito}). Feita essa comparação, é possível alterar a rede de conexões entre as camadas de \textit{input} e \textit{output} de modo a reforçar (ou enfraquecer) os pesos das mesmas para atingir o aprendizado proposto. Antes da primeira comparação, a rede é inicializada com pesos aleatórios. Conforme o número de comparações aumenta, a tendência é que os pesos sejam pouco a pouco calibrados para o modelo atingir o seu objetivo, que nesse caso, é aprender os padrões de flexão dos verbos.

\definecolor{blue}{RGB}{159, 192, 176}
\definecolor{green}{RGB}{160, 227, 127}
\definecolor{orange}{RGB}{243, 188, 125}
\definecolor{red}{RGB}{253, 123, 84}
\definecolor{nephritis}{RGB}{39, 174, 96}
\definecolor{emerald}{RGB}{46, 204, 113}
\definecolor{turquoise}{RGB}{39, 174, 96}
\definecolor{green-sea}{RGB}{22, 160, 133}
\definecolor{purple}{RGB}{181, 124, 215}
% Tikzstyles for Computation Graphs

% nodes
\tikzstyle{noop} = [circle, draw=none, fill=red, minimum size = 10pt]
\tikzstyle{op} = [circle, draw=red, line width=1.5pt, fill=red!70, text=black, text centered, font=\bf \normalsize, minimum size = 25pt]

\tikzstyle{opintense} = [circle, draw=red, line width=1.5pt, fill=red!150, text=black, text centered, font=\bf \normalsize, minimum size = 25pt]


%new style
\tikzstyle{gp} = [circle, draw=red, line width=4pt, text=black, text centered, font=\bf \normalsize, minimum size = 4.cm]

\tikzstyle{box} = [rectangle, draw=red, line width=1.5pt, fill=red!70, text=black, align=center, font=\bf \normalsize, minimum size = 45pt]

\tikzstyle{box2} = [rectangle, draw=black, line width=0.9pt, text=black, align=center, font=\bf \normalsize, minimum size = 20pt]

\tikzstyle{box3} = [rectangle, draw=black, line width=0.9pt, fill=black, text=black, align=center, font=\bf \normalsize, minimum size = 20pt]

\tikzstyle{state} = [circle, draw=blue, line width=1.5pt, fill=blue!70, text=black, text centered, font=\bf \normalsize, minimum size = 25pt]

\tikzstyle{output} = [circle, draw=purple, line width=1.5pt, fill=purple!70, text=black, text centered, font=\bf \normalsize, minimum size = 25pt]


\tikzstyle{gradient} = [circle, draw=nephritis, line width=1.5pt, fill=nephritis!60, text=black, text centered, font=\bf \normalsize, minimum size = 25pt]
\tikzstyle{textonly} = [draw=none, fill=none, text centered, font=\bf \normalsize]
\tikzstyle{boxtextonly} = [draw=none, fill=none, align=center, font=\bf \normalsize]

\tikzstyle{normal} = [circle, draw=black, line width=1.0pt, fill=none, text=black, text centered, font=\bf \normalsize, minimum size = 20pt]


% edges
\tikzstyle{tedge}  = [draw, thick, >=latex, ->]
\tikzstyle{tedge_dashed}  = [draw, thick, >=latex, ->, dashed]
\tikzstyle{nedge}  = [draw, thick, >=latex]
\tikzstyle{nedge_dashed}  = [draw, thick, >=latex, dashed]


% namedscope
\tikzstyle{namedscope} = [circle, draw=orange, line width=1.5pt, fill=orange!60, align=center, inner sep=0pt]
\begin{figure}[h]
\centering

\scalebox{1.0}{
\begin{tikzpicture}[auto]

% operations =========
% phon features 1
\node[textonly] (out1) {Output};
\node[textonly, right=25pt of out1] (gab) {Alvo};


% FNN output
\node[normal, below=40pt of out1] (x1) {$y_{1}$};
\node[normal, below=35pt of x1] (x2) {$y_{2}$};
\node[normal, below=35pt of x2] (x3) {$y_{3}$};

% from input
\node[text, left=45pt of x1] (in1) {};
\node[text, left=45pt of x2] (in2) {};
\node[text, left=45pt of x3] (in3) {};


% comparison
\node[text, right=31pt of x1] (nada1) {};
\node[text, below=5pt of nada1] (nada2) {\small{Comparação}};
\node[text, below=10pt of out1] (update) {\small{Atualização}};
\node[text, right=31pt of x2] (nada6) {};
\node[text, right=31pt of x3] (nada7) {};

\node[text, left=31pt of x1] (nada3) {};
\node[text, left=31pt of x2] (nada4) {};
\node[text, left=31pt of x3] (nada5) {};


% FNN target
\node[normal, right=65pt of x1] (y1) {$\hat{y_{1}}$};
\node[normal, right=65pt of x2] (y2) {$\hat{y_{2}}$};
\node[normal, right=65pt of x3] (y3) {$\hat{y_{2}}$};
\node[text, below=15pt of x3] (nada) {};



% edges FNN
\path[arrows_dashed] (x1) -- (y1);
\path[arrows_dashed] (x2) -- (y2);
\path[arrows_dashed] (x3) -- (y3);

\draw[arrows_dashed, ->] (nada1) to [out=135,in=115] (nada3);
\draw[arrows_dashed, ->] (nada6) to [out=135,in=115] (nada4);
\draw[arrows_dashed, ->] (nada7) to [out=135,in=115] (nada5);

\path[tedge] (in1) -- (x1);
\path[tedge] (in2) -- (x1);
\path[tedge] (in3) -- (x1);

\path[tedge] (in1) -- (x2);
\path[tedge] (in2) -- (x2);
\path[tedge] (in3) -- (x2);

\path[tedge] (in1) -- (x3);
\path[tedge] (in2) -- (x3);
\path[tedge] (in3) -- (x3);



\end{tikzpicture}
}\caption{Comparações entre o \textit{Output} e o \textit{Target}} 
\label{fig:gabarito}
\end{figure}

Para realizar o treinamento, \cite{rumelhart:1986} introduzem 420 verbos no modelo repetidamente (200 vezes cada um, 84.000 inserções no total).  Após o treinamento, o modelo foi capaz de prever corretamente todos os 420 verbos que já haviam sido inseridos. Além disso, em um novo grupo com 86 verbos desconhecidos, acertou cerca de 2/3 do conjunto. Dentre os novos verbos irregulares apresentados, cometeu alguns erros interessantes de \textit{super-regularização} (como \textit{catched} (ao invés de caught) e \textit{digged} (ao invés de dug)). Esses erros foram observados em 11 dos 14 verbos irregulares testados (\cite{pinker:1993}).  

Além desses resultados, \cite{rumelhart:1986} relatam que o processo de aprendizado do modelo apresentou um fenômeno interessante, reproduzindo um desempenho similar a comportamentos observáveis em crianças durante a fase de aquisição: a Curva de Desenvolvimento em U (U-shaped Development, \cite{marcus:1992}). A Curva de Desenvolvimento em U basicamente se refere a um processo de aprendizado que ocorre em três estágios: \\

(i) inicialmente, crianças acertam a flexão de verbos irregulares (\textit{come}$\rightarrow$\textit{came});

(ii) em seguida passam por um processo de \textit{super-regularização} (em que produzem formas como \textit{comed}), e passam a errar flexões que antes acertavam;

(iii) por fim, elas passam a reproduzir corretamente tanto os verbos regulares quanto irregulares. \\


\cite{rumelhart:1986} descrevem como foi possível observar tal comportamento também no modelo computacional desenvolvido.
Na fase inicial do processo de treinamento, o modelo foi alimentado com uma quantidade pequena de verbos, como: \textit{come}, \textit{get}, \textit{give}, \textit{look}, \textit{take}, \textit{go}, \textit{have}, \textit{live} e \textit{feel}. A performance do modelo foi compatível com o primeiro estágio da curva, ou seja, para esses verbos foi capaz de identificar corretamente a forma correspondente no \textit{Simple Past}. Em um segundo momento, o modelo foi alimentado com uma quantidade muito maior de verbos. Nesse estágio é possível verificar que o modelo está passando por um processo de regularização sistemática dos verbos. Ele produziu resultados como: \textit{breaked}, \textit{comed}, \textit{gived}; e também combinações entre padrões regulares e irregulares (ex. \textit{gaved}). 
Após uma série de muitas inserções repetidas, o modelo finalmente foi capaz de responder corretamente a uma quantidade maior de verbos, assim como no último estágio do processo de aprendizado natural. 

Os resultados de \cite{rumelhart:1986} causaram um considerável impacto na comunidade científica da época. Muitos pesquisadores viam o novo modelo como uma completa mudança de paradigma, não apenas na linguística, mas também como uma nova forma de se estudar aprendizado em geral (\cite{Schneider1987}). 

Apesar disso, \cite{Pinker:1988} dão continuidade ao debate ao apontar uma série de questões pertinentes que o modelo falhou em explicar. Primeiramente, como o modelo recebe apenas uma representação fonética do verbo como \textit{input}, ele é incapaz de gerar duas respostas diferentes para verbos com sonoridade idêntica (por exemplo \textit{break}$\rightarrow$\textit{broke} e \textit{brake}$\rightarrow$\textit{braked}). Para realizar essas predições corretamente, o modelo precisaria de um módulo adicional para distinguir entre as duas palavras, o que o descaracterizaria como modelo puramente associativo. Em segundo lugar, o modelo é extremamente dependente dos padrões observados durante o treinamento, tendo uma capacidade baixa para generalizações. \cite{Pinker:1999} comenta que o modelo ficou mudo quando alimentado com os verbos \textit{jump}, \textit{pump}, \textit{warm, trail} e \textit{glare} (que dispõem de uma sonoridade razoavelmente incomum). Além disso, o modelo apresentou alguns resultados completamente distorcidos, como: \textit{squat – squakt, tour – toureder} e \textit{mail – membled}; associações inaceitáveis para qualquer falante nativo. 

Com relação ao padrão de aprendizado observado (a Curva de Desenvolvimento em U), \cite{Pinker:1999} explica que esse comportamento foi causado segundo a forma em que os verbos foram inseridos no modelo durante o treinamento: Rumelhart \& McClelland realizaram o treinamento em partes e controlando a quantidade de repetições de cada lote de verbos. Na primeira parte do treinamento, selecionaram alguns verbos de alta frequência na língua inglesa (muitos deles irregulares), reproduzindo o estágio (i) da curva. Em seguida, treinaram o modelo com esses verbos, reintroduzindo-os múltiplas vezes até que o modelo conseguisse atingir um desempenho razoável nesses verbos. Na sequência introduziram uma quantidade maior de verbos, sendo estes menos frequentes que os anteriores, mas em sua maioria regulares. Dessa forma, o modelo começou a se ajustar para aplicar a regra regular e assim foi possível observar os estágios (ii) e (iii) da curva. Ainda, segundo \cite{pluket:1991}, os estágios de desenvolvimento (i), (ii) e (iii) podem ser considerados parte de um comportamento \textit{macro U-shape}, mas ainda é possível observar a ocorrência de um comportamento \textit{micro U-shape}. \cite{pluket:1991} observam que a reprodução dos verbos irregulares na fala espontânea de crianças aprendizes oscila bastante entre flexões corretas e \textit{super-regularizadas}. Eles também notam que estas oscilações ocorrem em proporções diferentes para cada verbo e que crianças raramente “\textit{irregularizam}” verbos regulares (como \textit{ping}$\rightarrow$\textit{pang}), e raramente misturam uma forma irregular com uma regular (fato que ocorreu durante o aprendizado do modelo com \textit{gaved}).

Para concluir, \cite{Pinker:1988} apresentam a formulação de uma nova hipótese para tal questão: uma teoria híbrida na qual a fonologia gerativa se aplicaria ao processo de flexão regular e um mecanismo associativo se aplicaria ao processo de flexão irregular. Os pesquisadores propõem que as formas regulares sejam computadas a partir de um mecanismo que deve abstrair o radical do verbo e combiná-lo ao sufixo –ed.  Tal mecanismo pode ser aplicado a qualquer palavra, em um processo independente da memória. As formas irregulares, por sua vez, passam por um processo diferente: verbos irregulares precisam passar por um processo de memorização, havendo não somente a associação entre uma forma verbal e outra mas também entre as propriedades (traços fonéticos, rima, radical, núcleo, etc.) das mesmas, parecido com o que foi proposto por Rumelhart e McClelland.

\section{Motivação}
\label{sec:motivation}

Uma vez que uma contextualização acerca do problema já foi apresentada, focaremos agora nas questões que motivaram o desenvolvimento desta pesquisa. Assim, esta seção será dividida em duas partes: (i) motivação no campo da linguística, e (ii) motivação no campo da ciência da computação.

\subsection{Motivação no campo da linguística}
\label{sec:aprendizado_port}
 
A morfologia verbal da língua inglesa é bastante simples, se comparada à portuguesa. Em primeiro lugar, os verbos do português se distribuem em três classes módicas (\textit{conjugações}), sendo cada uma destas definida a partir de uma \textit{vogal temática} (\textit{/a/}, \textit{/e/} e \textit{/i/}). Dado um verbo em sua forma infinitiva, por exemplo \textit{“amar”}, a vogal temática é a aquela que se encontra entre o morfema lexical do verbo (o radical) e a desinência de infinitivo \textit{r}.

\begin{align*}
    \text{Am + a + r}\\
    \text{Radical + VT + r} 
\end{align*}

Com isto, os três possíveis tipos de conjugação são: 1\aup{a} - ar (amar, brigar), 2\aup{a} - er (beber, comer) e 3\aup{a} (rir, descobrir). Na língua inglesa, essa distinção não existe. 

Uma criança em processo de aquisição de linguagem no sistema do português brasileiro passa por muitos desafios. Uma parte do processo é justamente perceber a relação entre a vogal temática e as possíveis conjugações verbais regulares. Nesse processo, não é incomum observarmos o surgimento de trocas de conjugação. \cite{wuerges:2014} apresenta dados linguísticos produzidos por crianças com diversas destas trocas. Alguns exemplos de seu trabalho podem ser observados na Tab. \ref{tab:aquisicao}. 

\begin{table}[H]
\begin{center}
\begin{tabular}{cccc}
Verbo & Observado & Correto & Troca  \\ \hline
botar & “eu boti” & botei & 1\aup{a} com 2\aup{a} ou 3\aup{a} \\
comer & “eu comei” & comi & 2\aup{a} com 1\aup{a} \\
jantar & “eu janti” & jantei & 1\aup{a} com 2\aup{a} ou 3\aup{a} \\ \hline

\end{tabular}
\caption{Exemplos de Trocas de Conjugação Durante o Processo de Aquisição Verbal}
\label{tab:aquisicao}
\end{center}
\end{table}

As formas verbais irregulares apresentam-se como uma dificuldade adicional nesse processo para as crianças falantes da língua portuguesa. \cite{wuerges:2014} também aponta exemplos observados de \textit{regularização} de verbos irregulares: “eu \textit{consego}” (regularização do verbo conseguir) e “eu \textit{podo}” (regularização do verbo poder).

Um verbo é dito irregular se apresentar alterações no radical (em relação ao radical da forma infinitiva) e/ou no sufixo flexional (em relação ao padrão regular imposto por cada conjugação) (\cite{wuerges:2014}). Os sufixos flexionais (SF) são os segmentos acrescentados após o radical do verbo. Eles podem ser divididos em dois tipos: (i) sufixo modo-temporal (SMT) e (ii) sufixo número-pessoal (SNP). Para o verbo “gostávamos”, por exemplo, considera-se o segmento \textit{/gost/} como o radical do verbo, \textit{/av/} como o sufixo modo-temporal, que neste caso marca simultaneamente o modo indicativo e tempo pretérito imperfeito; e \textit{amos} como o sufixo número-pessoal indicando primeira pessoa do plural (nós).

Seguindo a definição proposta, é necessário reforçar que o interesse deste estudo está em capturar irregularidades no nível fonético, portanto verbos como: “gosto”, “boto” e “coloco”, cuja ortografia apresenta o padrão regular, serão classificados como irregulares. Isso acontecerá em razão das transformações fonéticas que ocorrem na fala e que não são capturadas na escrita.  No caso de “gosto”, por exemplo, vemos que a primeira vogal “/o/”, na verdade tem som de /\textopeno/ (ó). Assim, a Tabela \ref{tab:irreg} exibe alguns exemplos das classificações realizadas.

\begin{center}
\begin{table}[H]
\centering
\begin{tabular}{ccc}
\multicolumn{1}{l}{\textbf{Verbo Infinitivo}} & \multicolumn{1}{l}{\textbf{Verbo Flexionado}} & \multicolumn{1}{l}{\textbf{Classificação}} \\ \hline
Falar & Falo & Regular \\
Gostar & Gosto & Irregular \\
Testar & Testo & Irregular \\
Ansiar & Anseio & Irregular \\
Pedir & Peço & Irregular \\
Mentir & Minto & Irregular \\
Por & Ponho & Irregular
\end{tabular}
\caption{Exemplos de Classificações de Verbos Quanto a Presença de Irregularidades}
\label{tab:irreg}
\end{table}
\end{center}

Uma observação sobre a disposição das irregularidades presentes no português brasileiro (levando em consideração apenas a 1\aup{a} Pessoa do Singular (tempo Presente e modo Indicativo) nos permite observar algumas regularidades (padrões) dentre os verbos irregulares:\\
\\
Bobear – Bobeio, Bloquear – Bloqueio, Chatear – Chateio, Clarear – Clareio, Golpear – Golpeio\\
\\
Agredir – Agrido, Conseguir – Consigo, Inserir – Insiro, Perseguir – Persigo, Preferir – Prefiro, Proferir – Profiro, Repetir – Repito, Servir –  Sirvo, Vestir – Visto\\
\\
Cobrir – Cubro, Dormir – Durmo, Engolir – Engulo\\
\\
 Al[e]gar – Al[ε]go, C[e]gar – C[ε]go, Compl[e]tar – Compl[ε]to,  Col[e]tar – Col[ε]to, Entr[e]gar – Entr[ε]go, Pr[e]gar – Pr[ε]g,\\
\\
Ad[o]rar – Ad[\textopeno]ro, Ad[o]tar – Ad[\textopeno]to, B[o]tar – B[\textopeno]to, C[o]lar – C[\textopeno]lo, F[o]car – F[\textopeno]co, M[o]rar – M[\textopeno]ro, S[o]ltar – S[\textopeno]lto, S[o]lar – S[\textopeno]lo, T[o]car – T[\textopeno]co, M[o]strar – M[\textopeno]stro\\
\\
Mentir - Minto, Sentir - Sinto\\

Os padrões observados a partir da exposição de algumas classes irregulares, permitem, assim como no inglês, a proposição de fórmulas, ou regras fonéticas, que explicam as flexões realizadas em cada classe. É possível notar, por exemplo, que um verbo da mesma família de \textit{conseguir} segue a regra:

\begin{center}
e $\rightarrow$ i/\_C]ir 
\end{center}

A regra proposta indica que /e/ se transforma em /i/ quando em um contexto de terceira conjugação (ir). No caso, C indica uma consoante qualquer. 

Os padrões encontrados sugerem não somente a possibilidade de elaboração de regras, como também a possibilidade do desenvolvimento de redes capazes de capturar tais dependências. 

\subsection{Motivação no campo da ciência da computação}
\label{sec:compmot}

Desde a apresentação da pesquisa de \cite{rumelhart:1986}, o modelo associativo utilizado pelos autores já passou por diversos avanços. Na realidade, esse tipo de modelagem hoje é chamado de Rede Neural Artificial e passou a ser utilizado em uma variedade de tarefas computacionais, como classificação de imagens, classificação de texto, tradução automática, agentes conversacionais, entre outros. 

Nos últimos anos, o poder computacional aumentou muito. O desenvolvimento dos \textit{hardwares} possibilita que hoje sejam realizadas muito mais computações e com muito mais velocidade do que na década de 80. Com isso, arquiteturas mais complexas puderam ser exploradas e desenvolvidas. A título de exemplo, é possível acrescentar camadas intermediárias entre as camadas de \textit{input} e \textit{output}. Redes construídas dessa forma possibilitam que as informações de entrada sejam distribuídas ao longo de mais conexões e com isso, potencializem o aprendizado. Isso acontece pois o tipo de arquitetura sem camadas intermediárias consegue aproximar apenas um tipo de função, as funções lineares. Ao aumentar o número de camadas intermediárias, é possível ampliar o universo de soluções para a resolução de problemas mais complexos (ver \cite{Goodfellow-et-al-2016} para explicações mais detalhadas sobre as camadas intermediárias). Esse tipo de modelagem que segue um fluxo com um sentido único (do \textit{input} ao \textit{output}) é chamado de \textit{Feedforward} (FFD). Entretanto, existem muitos tipos de arquiteturas cujos fluxos não seguem essa configuração. Nesse âmbito, um tipo de arquitetura que ficou muito conhecido é o \textit{Convolucional} (\textit{Redes Neurais Convolucionais (RNC's)}) (bastante utilizado na área de visão computacional (\cite{Krizhevsky:2012}, por exemplo). No campo de tarefas linguísticas, as redes do tipo \textit{Recorrente} (\textit{Redes Neurais Recorrentes (RNR's}) são bastante utilizadas, uma vez que a arquitetura viabiliza a inserção de dados sequenciais  (\cite{pengfei:2016}, por exemplo). O tema dos modelos de Redes Neurais, em especial a arquitetura de RNR's, será abordado com mais detalhes no Capítulo \ref{ch:03}.

No que diz respeito à questão do aprendizado dos verbos irregulares do inglês, uma série de novos experimentos se sucederam após as críticas de (\cite{Pinker:1988}). (\cite{pluket:1991}, (\citeyear{PLUNKETT:1993})) simplificam a questão ao considerar apenas verbos de tamanho fixo (3 sílabas) e abordam o problema fazendo uso de uma arquitetura com adição de camadas intermediárias (\textit{Multi-Layered Perceptron - MLP}). Outros trabalhos transformaram a questão em um problema de \textit{classificação}, desse modo o modelo não teria mais como objetivo encontrar uma forma flexionada. Ao invés disso, teriam um conjunto finito e pré determinado de formas possíveis. \cite{Nakisa1996WhereDD}, por exemplo, classificam os plurais dos substantivos da língua alemã. \cite{plunkett:1997} atacam o mesmo problema, porém para a língua Árabe.

\cite{wetermann:1997} apresentam um modelo construído para mapear verbos não flexionados da língua alemã para a forma no particípio. O modelo apresentado é capaz de lidar com dados com sequências de tamanhos variáveis e utiliza uma arquitetura baseada em RNR's. Entretanto, o modelo foi construído a partir de um mecanismo de rota dupla, de modo que verbos irregulares passavam por uma rota específica de memorização e verbos regulares por outra rota, com a aplicação da regra. Apesar disso, o desempenho do modelo deixou a desejar. Alguns experimentos diferentes foram realizados, sendo que um deles consistia no treinamento de grupos de verbos isoladamente. Para tal, os verbos regulares ficaram em um grupo e os irregulares foram divididos em outros dois. Nesse caso, a porcentagem de acerto do modelo chegou a quase 100\% para os grupos treinados isoladamente. Entretanto, ao misturar os verbos em um único treinamento, a porcentagem de acerto se estabilizou em torno de 60 a 70\%.  

No âmbito da construção de modelos não associativos, ou seja, baseados em regras, \cite{Albright2003RulesVA} apresentam o modelo \textit{Minimal Generalization Learner (MGL)}, cuja implementação se aproxima muito da teoria proposta por \cite{Pinker:1988}. O modelo MGL se baseia na descoberta e atribuição de pesos a regras pré-estabelecidas para as transformações irregulares. Falaremos mais sobre os resultados desse modelo na seção de discussão dos resultados (Seç. \ref{sec:discuss}).

\subsection{Arquitetura Estado-da-Arte}

\cite{enc-dec:2014} e \cite{seq2seq:2014} apresentam um novo tipo de arquitetura de Redes Neurais construído para mapear duas sequências de tamanhos variáveis. A nova arquitetura, conhecida como \textit{Encoder-Decoder}, ou também \textit{Seq2Seq}, é reconhecida especialmente pelo seu bom desempenho em tarefas linguísticas, em especial no ramo da tradução automática (\cite{Wu:2016}). Essa arquitetura consiste na concatenação de duas RNR's. A primeira rede, \textit{Encoder}, lê cada símbolo de uma sequência de entrada (por exemplo, uma palavra em inglês) e gera como resposta uma representação abstrata da palavra lida. A segunda rede, \textit{Decoder}, recebe como entrada a representação devolvida pelo \textit{Encoder} e tem como objetivo produzir uma outra sequência alvo correspondente (sendo nesse caso de uma tarefa tradução, a palavra traduzida para outra língua). A arquitetura do modelo \textit{Encoder-Decoder} será abordada em maiores detalhes no Capítulo \ref{ch:05}.

Na tarefa do aprendizado dos verbos irregulares, \cite{faruqui:2015} elabora a questão a nível de \textit{caracteres} (letras), ou seja, não utiliza traços fonéticos como \textit{inputs} no modelo \textit{Encoder-Decoder}. \cite{kann-schutze-2016-med} utilizam rótulos morfológicos (marcações de particípio, gerúndio, etc) como \textit{inputs} no mesmo modelo. \cite{cotterell-sigmorphon2016} atingem performance \textit{estado-da-arte} em um problema de flexão morfológica, postulado em uma tarefa compartilhada (\textit{SIGMORPHON Shared Task} (\url{http://www.sigmorphon.org/})). Neste problema foram introduzidos conjuntos de dados morfológicos para 10 idiomas (Espanhol, Alemão, Finlandês, Russo, Turco, Georgiano, Navajo, Árabe e Húngaro) com diversas características tipológicas. Para a tarefa de geração de flexões de lemas, o sistema de \cite{cotterell-sigmorphon2016} obteve uma média de acurácia de 95,56\% em todos os idiomas, variando de
Maltês (88,99\%) para húngaro (99,30\%).

Assim, temos como motivação para esta pesquisa o uso do modelo \textit{Encoder-Decoder} para o aprendizado de flexão verbal. O modelo \textit{Encoder-Decoder}, apesar de ter apresentado resultados promissores, ainda não foi aplicado a nível exclusivamente fonético (como em (\cite{rumelhart:1986})), uma vez que as pesquisas anteriores exibem resultados a nível de caracteres e com rótulos morfológicos. Além disso, vemos também que o modelo nunca foi testado na língua portuguesa.


\subsection{Delimitação de escopo}
\label{sec:escopo}

Em comparação ao inglês, é possível argumentar que a língua portuguesa é mais complexa. No inglês, considerando o tempo presente, vemos a distinção de apenas duas formas: (i) a forma para \textit{I, We, They} (como, por exemplo, o verbo \textit{walk}) e (ii) para \textit{he/she/it }(\textit{walks}). Uma exceção a essa regra é o verbo \textit{to be}, que apresenta três formas possíveis: (i) I \textit{am}, (ii) he/she/it \textit{is} e (iii) they \textit{are}. No \textit{Simple Past}, apresenta maior número de formas também apenas para o grupo \textit{to be}: (i) I/he/she/it \textit{was}, e (ii) They/We \textit{were}; os demais não apresentam marcação de pessoa (\cite{Nelson:2010}). No português, a norma tradicional distingue seis pessoas: 1\aup{a}: Eu, 2\aup{a}: Tu, 3\aup{a}: Ele/Ela, 4\aup{a}: Nós, 5\aup{a}: Vós, 6\aup{a}: Eles. Entretanto, atualmente vemos que as formas associadas às 2\aup{a} e 5\aup{a} pessoas estão caindo em desuso. A 2\aup{a} pessoa, o “Tu”, é utilizada ainda em algumas regiões do Brasil, mas normalmente tem sido reproduzida como a forma da 3\aup{a} pessoa (ou seja, Você/Ele \textit{gosta}). Do mesmo modo, a forma de 5\aup{a} tem sido substituída pela forma de 6\aup{a} pessoa (Vocês/Eles \textit{gostam}) (\cite{1999:camara}).\footnote{Também é interessante notar que ultimamente a 4\aup{a} (Nós) tem sido alternada com o uso de “A gente”, cuja forma verbal corresponde à 3\aup{a} pessoa também (A gente \textit{gosta}).}

Com relação às irregularidades, no inglês os verbos irregulares encontram-se apenas no \textit{Simple Past} e \textit{Past Participle}, enquanto que o sistema verbal do português é repleto de irregularidades em todos os tempos, modos e pessoas. Vejamos o verbo “\textit{dizer}”, por exemplo. No tempo presente e modo indicativo, notamos a presença de irregularidade na primeira e terceira pessoa do singular (“\textit{digo}” e \textit{“diga”}). No pretérito perfeito, o paradigma é totalmente irregular, com: “\textit{disse}”, “\textit{disseste}”, “\textit{disse}”, “\textit{dissemos}”, “\textit{dissestes}” e “\textit{disseram}”. No futuro do presente, temos: “\textit{direi}”, “\textit{dirás}”, “\textit{dirá}”, “\textit{diremos}”, “\textit{direis}” e “\textit{dirão}”.\footnote{Também é possível observar irregularidades nos modos subjuntivo e imperativo em todos os paradigmas.} 

Apesar da complexidade do português mostrar-se relevante para o desafio do aprendizado de flexão de verbos irregulares, optamos pela realização de um estudo mais restrito. A razão para tal limitação decorre principalmente da ausência de um corpus preparado para a realização da tarefa. Desse modo, ao limitarmos as irregularidades a um único paradigma, colocamos o nível de dificuldade em um patamar similar ao da pesquisa de \cite{rumelhart:1986}, tendo como principal diferença (e possivelmente dificuldade extra) as três conjugações possíveis. 

O escopo foi, portanto, restrito à 1\aup{a} Pessoa do Singular no tempo Presente e modo Indicativo (com exemplos já explorados na Seção \ref{sec:aprendizado_port}). Desse modo, verbos que apresentarem irregularidade em outro tempo, modo ou pessoa que não na 1\aup{a} pessoa do singular no tempo presente e modo indicativo, serão tratados como pertencentes à classe dos regulares. Como exemplo, considere o verbo \textit{correr}. Esse verbo apresenta flexão regular para a 1\aup{a} Pessoa (\textit{corro}), mas é irregular para a 3\aup{a} Pessoa (\textit{corre}). Assim, apesar de \textit{correr} ser um verbo irregular, como ele apresenta flexão regular no paradigma escolhido, será tratado como regular para os fins dessa pesquisa. 
