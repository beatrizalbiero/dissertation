\chapter{Introdução}
\label{ch:01}

A questão do aprendizado infantil, referente ao processo de flexão de verbos irregulares na língua inglesa, está certamente entre um dos temas de debate mais controversos entre as principais correntes teóricas no estudo da linguística (Pinker, \citeyear{Pinker:1999}). O cerne do debate está na exata caracterização dos mecanismos que possibilitam que um falante seja capaz de relacionar um verbo no infinitivo à sua forma no tempo passado.

O tempo passado do inglês é composto por uma variedade de famílias, ocorrendo não somente a formação de um padrão regular (verbo + \textit{ed}), mas também a formação de subgrupos de verbos irregulares, como por exemplo:

\begin{center}
like - liked, ask - asked, move - moved\\
blow – blew, grow – grew, know – knew, throw – threw\\
bear – bore, swear – swore, tear – tore, wear – wore\\
drink – drank, shrink – shrank, sink – sank, stink – stank \\
\end{center}

É possível pensar que o aprendizado de tais padrões decorreria de uma memorização caso a caso. No entanto, experimentos realizados mostraram que, quando apresentados a verbos inventados, os indivíduos testados apresentaram tendências com relação à alocação dos verbos em classes. Por exemplo, para o verbo artificial spling, a maioria dos indivíduos optou pela forma splang  ou splung (Bybee e Molder \citeyear{Bybee:1983}). Esse exemplo contradiz a ideia de que os falantes poderiam estar apenas reproduzindo formas memorizadas e sugere que eles estejam ativamente identificando padrões, e mais: possuem uma intuição natural sobre a adequabilidade da alocação de um verbo a uma classe ou a outra.

Uma alternativa é pensar que as próprias unidades fonéticas das palavras possam fornecer pistas aos falantes que os permitam relacionar verbos com traços fonéticos similares a uma mesma família. É a caracterização de tal processo de categorização que estabelece o debate entre as correntes teóricas racionalistas e empiristas.  

Para a teoria racionalista da Fonologia Gerativa de Chomsky e Halle (\citeyear{chomsky:1968}), os indivíduos seriam portadores de um dispositivo de aquisição de linguagem (Language Acquisition Device) responsável pela formulação e manipulação de estruturas fonológicas abstratas em um sistema intrincado de regras. De modo simplificado, a teoria propõe que o falante seja capaz de identificar e formular regras intuitivamente para dar conta do aprendizado das formas irregulares da língua. Um exemplo disso é a família dos verbos terminados em “-ind”.

\begin{center}
bind – bound, find – found, grind – ground, wind – wound
\end{center}

Vemos que, de modo simplificado, a regra em uso aqui é algo como:
%verificar essa formula fonetica
\begin{center}
ai $\rightarrow$ au / \_nd]+past
\end{center}

Uma estrutura como essa permitiria ao falante construir generalizações e, consequentemente, aprender com eficiência e rapidez. 

Apesar dos argumentos apresentados, a teoria racionalista foi confrontada com um forte questionamento. Tal questionamento apresenta-se com relação a este sistema de manipulação simbólica sugerido pela teoria racionalista, o qual os pesquisadores Rumelhart e McClelland intitularam de Regras Explícitas Inacessíveis (\textit{Explicit Inaccessible Rules}). Rumelhart e McClelland argumentam que comportamentos de caráter regrado podem ser produzidos por mecanismos em que não existam representações explícitas das regras em uso. Ao invés disso, os pesquisadores sugerem que os mecanismos envolvidos no processo de flexão verbal possam ser construídos de tal forma que a sua performance possa ser caracterizada por regras, mas que as regras em si não estejam representadas explicitamente em nenhuma parte do processo. Para sustentar essa ideia, Rumelhart e McClelland apresentaram um modelo computacional  de caráter empirista que foi fundamental para o surgimento de uma nova escola dentro das ciências cognitivas: o conexionismo.


\definecolor{blue}{RGB}{159, 192, 176}
\definecolor{green}{RGB}{160, 227, 127}
\definecolor{orange}{RGB}{243, 188, 125}
\definecolor{red}{RGB}{253, 123, 84}
\definecolor{nephritis}{RGB}{39, 174, 96}
\definecolor{emerald}{RGB}{46, 204, 113}
\definecolor{turquoise}{RGB}{39, 174, 96}
\definecolor{green-sea}{RGB}{22, 160, 133}
\definecolor{purple}{RGB}{181, 124, 215}
% Tikzstyles for Computation Graphs

% nodes
\tikzstyle{noop} = [circle, draw=none, fill=red, minimum size = 10pt]
\tikzstyle{op} = [circle, draw=red, line width=1.5pt, fill=red!70, text=black, text centered, font=\bf \normalsize, minimum size = 25pt]

\tikzstyle{opintense} = [circle, draw=red, line width=1.5pt, fill=red!150, text=black, text centered, font=\bf \normalsize, minimum size = 25pt]


%new style
\tikzstyle{gp} = [circle, draw=red, line width=4pt, text=black, text centered, font=\bf \normalsize, minimum size = 4.cm]

\tikzstyle{box} = [rectangle, draw=red, line width=1.5pt, fill=red!70, text=black, align=center, font=\bf \normalsize, minimum size = 45pt]

\tikzstyle{box2} = [rectangle, draw=black, line width=0.9pt, text=black, align=center, font=\bf \normalsize, minimum size = 20pt]

\tikzstyle{box3} = [rectangle, draw=black, line width=0.9pt, fill=black, text=black, align=center, font=\bf \normalsize, minimum size = 20pt]

\tikzstyle{state} = [circle, draw=blue, line width=1.5pt, fill=blue!70, text=black, text centered, font=\bf \normalsize, minimum size = 25pt]

\tikzstyle{output} = [circle, draw=purple, line width=1.5pt, fill=purple!70, text=black, text centered, font=\bf \normalsize, minimum size = 25pt]


\tikzstyle{gradient} = [circle, draw=nephritis, line width=1.5pt, fill=nephritis!60, text=black, text centered, font=\bf \normalsize, minimum size = 25pt]
\tikzstyle{textonly} = [draw=none, fill=none, text centered, font=\bf \normalsize]
\tikzstyle{boxtextonly} = [draw=none, fill=none, align=center, font=\bf \normalsize]

\tikzstyle{normal} = [circle, draw=black, line width=1.0pt, fill=none, text=black, text centered, font=\bf \normalsize, minimum size = 20pt]


% edges
\tikzstyle{tedge}  = [draw, thick, >=latex, ->]
\tikzstyle{tedge_dashed}  = [draw, thick, >=latex, ->, dashed]
\tikzstyle{nedge}  = [draw, thick, >=latex]
\tikzstyle{nedge_dashed}  = [draw, thick, >=latex, dashed]


% namedscope
\tikzstyle{namedscope} = [circle, draw=orange, line width=1.5pt, fill=orange!60, align=center, inner sep=0pt]
\begin{figure}[ht!]
\centering

\scalebox{1.0}{
\begin{tikzpicture}[auto]

% operations =========
% phon features 1
\node[textonly] (1pho1) {int-vogal-int};

% Legenda
\node[textonly, above=10pt of 1pho1] (leg1) {Unidades de Input};


% FNN input
\node[normal, right=5pt of 1pho1] (x1) {};
\node[normal, below=25pt of x1] (x2) {};
\node[normal, below=25pt of x2] (x3) {};
\node[normal, below=25pt of x3] (x4) {};
\node[normal, below=25pt of x4] (x5) {};
\node[normal, below=25pt of x5] (x6) {};

% FNN output
\node[normal, right=45pt of x1] (y1) {};
\node[normal, right=45pt of x2] (y2) {};
\node[normal, right=45pt of x3] (y3) {};
\node[normal, right=45pt of x4] (y4) {};
\node[normal, right=45pt of x5] (y5) {};
\node[normal, right=45pt of x6] (y6) {};

% phon features 2
\node[textonly, right=5pt of y1] (2pho1) {int-vogal-int};
\node[textonly, above=10pt of 2pho1] (leg2) {Unidades de Output};
\node[textonly, left=25pt of x2] (1pho2) {anterior-nasal-posterior};
\node[textonly, right=25pt of y2] (2pho2) {anterior-nasal-posterior};
\node[textonly, left=25pt of x3] (3pho1) {...};
\node[textonly, right=25pt of y3] (1pho3) {...};
\node[textonly, left=25pt of x4] (4pho1) {nasal-cont-ocl};
\node[textonly, right=25pt of y4] (1pho4) {nasal-cont-ocl};
\node[textonly, left=25pt of x5] (5pho1) {médio-cont-baixa};
\node[textonly, right=25pt of y5] (1pho5) {médio-cont-baixa};
\node[textonly, left=25pt of x6] (6pho1) {vogal-fric-\#};
\node[textonly, right=25pt of y6] (1pho6) {vogal-fric-\#};
% edges FNN
\path[nedge] (x1) -- (y1);
\path[nedge] (x1) -- (y2);
\path[nedge] (x1) -- (y3);
\path[nedge] (x1) -- (y4);
\path[nedge] (x1) -- (y5);
\path[nedge] (x1) -- (y6);
\path[nedge] (x2) -- (y1);
\path[nedge] (x2) -- (y2);
\path[nedge] (x2) -- (y3);
\path[nedge] (x2) -- (y4);
\path[nedge] (x2) -- (y5);
\path[nedge] (x2) -- (y6);
\path[nedge] (x3) -- (y1);
\path[nedge] (x3) -- (y2);
\path[nedge] (x3) -- (y3);
\path[nedge] (x3) -- (y4);
\path[nedge] (x3) -- (y5);
\path[nedge] (x3) -- (y6);
\path[nedge] (x4) -- (y1);
\path[nedge] (x4) -- (y2);
\path[nedge] (x4) -- (y3);
\path[nedge] (x4) -- (y4);
\path[nedge] (x4) -- (y5);
\path[nedge] (x4) -- (y6);
\path[nedge] (x5) -- (y1);
\path[nedge] (x5) -- (y2);
\path[nedge] (x5) -- (y3);
\path[nedge] (x5) -- (y4);
\path[nedge] (x5) -- (y5);
\path[nedge] (x5) -- (y6);
\path[nedge] (x6) -- (y1);
\path[nedge] (x6) -- (y2);
\path[nedge] (x6) -- (y3);
\path[nedge] (x6) -- (y4);
\path[nedge] (x6) -- (y5);
\path[nedge] (x6) -- (y6);


\end{tikzpicture}
}\caption{Esquema da rede neural utilizada pelos pesquisadores Rumelhart e McClelland} 
\label{fig:esquemafdd}
\end{figure}


O modelo desenvolvido foi criado por analogia à estrutura em que se relacionam os neurônios no cérebro, por isso, recebeu o nome de rede neural artificial (artificial neural network). Ele é composto basicamente por uma rede artificial de nódulos interconectados paralelamente (Fig. \ref{fig:esquemafdd}).

A primeira camada de nódulos é responsável por receber os dados de entrada (input), que são os dados referentes aos traços fonéticos que caracterizam os sons de um verbo no infinitivo. A segunda camada é uma camada de resposta (output) que deve tentar retornar dados referentes aos traços que caracterizam os sons do mesmo verbo fornecido no input, porém no tempo passado. Concluída esta etapa, os dados de saída obtidos deverão ser então comparados à forma correta do verbo no tempo passado, através de uma espécie de gabarito. A função das conexões entre as camadas é fortalecer (ou enfraquecer) as relações entre as camadas de input e output de acordo com as comparações realizadas entre a camada de output e o gabarito. É importante notar que, a priori, a rede não possui qualquer tipo de informação para seu funcionamento, essa aprendizagem irá decorrer ao longo de múltiplas iterações.

O modelo de Rumelhart e McClelland apresentou ótimos resultados na tarefa de prever as formas verbais esperadas para o simple past, conseguindo identificar associações corretamente entre todos os 420 verbos em que foi treinado. Além disso, teve um desempenho satisfatório ao ser apresentado a 86 novos verbos que não fizeram parte do treinamento, obtendo uma taxa de acerto de 92\% para verbos regulares e 84\% para verbos irregulares (91\% de acerto para todos os verbos no total). O modelo serviu, portanto, para corroborar o argumento de que é possível realizar essa tarefa eficientemente dispensando o uso de regras explícitas. 

Além desses resultados, o processo de aprendizagem do modelo computacional exibiu um fenômeno interessante, reproduzindo resultados similares a comportamentos observáveis em crianças durante a fase de aquisição: a Curva de Desenvolvimento em U (U-shaped Development, \cite{marcus:1992}). Na fase inicial do processo, o modelo foi exposto a uma quantidade pequena de verbos de alta frequência na língua inglesa, como: come, get, give, look, take, go, have, live e feel. A performance do modelo foi compatível com o primeiro estágio da curva, ou seja, para uma pequena quantidade de verbos, foi capaz de identificar corretamente a forma correspondente no passado simples. Em um segundo momento, o modelo foi exposto a uma quantidade muito maior de verbos. Nesse estágio fica evidente que o modelo está passando por um processo de regularização sistemática dos verbos. O modelo produziu resultados como: breaked, comed, gived; e também combinações entre padrões regulares e irregulares (ex. gaved),  compatível com o estágio intermediário do processo de aprendizagem. Após uma série de ensaios, o modelo foi finalmente capaz de responder corretamente a uma quantidade maior de verbos, assim como no último estágio do processo da aprendizado natural. 

Apesar do desempenho do modelo conexionista, os pesquisadores Pinker e Prince \citeyear{Pinker:1988} dão continuidade ao debate ao apontar uma série de questões pertinentes que a proposta conexionista falhou em explicar.  Pelo fato de ser meramente um mecanismo associativo entre traços fonéticos, o modelo acaba dependendo profundamente dos padrões encontrados entre os traços fonológicos das palavras fornecidas no treinamento. Isso significa que o modelo é incapaz de responder de maneira eficiente a verbos com traços fonéticos que não passaram pelo treinamento. \cite{Pinker:1999} destaca a vulnerabilidade do modelo diante da tarefa de prever a forma no passado para alguns verbos costumeiros na língua inglesa, mas que dispõem de uma sonoridade razoavelmente incomum. Quando exposto aos verbos jump, pump, warm,  trail e glare, o modelo não gerou nenhuma resposta. Além disso, apresentou alguns resultados completamente distorcidos, como: squat – squakt, tour – toureder e mail – membled; associações inaceitáveis para qualquer falante nativo. 

Pinker e Prince \citeyear{Pinker:1988} apresentam a formulação de uma nova teoria linguística para tal questão: uma teoria híbrida em que a fonologia gerativa se aplica ao processo de flexão regular e um modelo conexionista se aplica ao processo de flexão irregular. Os pesquisadores propõem que as formas regulares sejam computadas a partir de um mecanismo que deve abstrair o radical do verbo e combiná-lo ao sufixo –ed.  Tal mecanismo pode ser aplicado a qualquer palavra, em um processo independente da memória. As formas irregulares, por sua vez, passam por um processo diferente: verbos irregulares precisam passar por um processo de memorização, uma memorização que ocorre de maneira associativa, havendo não somente a associação entre um verbo e outro mas também entre as propriedades (traços fonéticos, rima, stem, núcleo, etc.) de um verbo e de outro, parecido com o que foi proposto por Rumelhart e McClelland.

\section{Motivação}
\label{sec:motivation}

\subsection{Motivação no campo da Linguística}

Em comparação com o inglês, o português brasileiro apresenta um sistema verbal flexional mais complexo. Primeiramente, é importante notar que o paradigma conjugacional do português brasileiro apresenta uma maior distinção flexional quanto a pessoas e número, enquanto que o inglês é mais limitado nesse aspecto (no presente simples distingue apenas as pessoas \textit{he/she/it} das demais). Quanto às formas regulares, o português dispõe de três conjugações diferentes determinadas pela vogal temática, enquanto que o inglês possui apenas a conjugação regular de passado (o sufixo –ed). Por último, é importante destacar que o sistema verbal do português é repleto de irregularidades em todos os tempos verbais, enquanto que o inglês apresenta irregularidades apenas no \textit{simple past} e \textit{past participle} (\cite{wuerges:2014}).

Um aprendiz da linguagem no sistema do português brasileiro é desafiado a superar uma série de obstáculos.  Uma parte do processo é justamente perceber a relação entre a vogal temática e as possíveis conjugações verbais regulares. Nesse processo não é incomum observarmos o surgimento de trocas de conjugação. \cite{wuerges:2014} apresenta dados linguísticos produzidos por crianças com diversas destas trocas: “eu \textit{boti}“ (como tentativa de produzir \textit{botei}; troca de primeira conjugação para terceira), “eu \textit{comei}“ (como tentativa de produzir \textit{comi}; troca de segunda conjugação para primeira), “eu \textit{janti}“ (como tentativa de produzir \textit{jantei}; troca de primeira conjugação para terceira). Além destas ainda ocorre uma grande confusão com relação aos verbos da família do verbo \textit{pôr}, \cite{wuerges:2014} comenta as produções de “eu \textit{puso}“ e “eu \textit{ponhei}“ como tentativa de produzir a forma correspondente no tempo passado.  

As formas verbais irregulares apresentam-se como uma dificuldade adicional nesse processo para as crianças falantes da língua portuguesa. \cite{wuerges:2014} também aponta exemplos observados de tentativas de regularização de verbos irregulares no tempo presente: “eu \textit{consego}“ (regularização do verbo conseguir) e “eu \textit{podo}“ (regularização do verbo poder). 

%Outra dificuldade é ter de lidar com o fato de que os verbos irregulares no português apresentam, em pelo menos uma forma verbal de seu paradigma, alterações no radical e/ou na sua desinência. Isto fica evidente quando observamos a enunciação de formas como: “eu consego*” ou “eu podo*” (poder). É interessante também notar enunciações criativas para verbos de natureza um pouco mais complicada, como o verbo por:  puso* (eu), ponhei* (eu) (\cite{wuerges:2014}).

Uma análise sobre a disposição das irregularidades presentes no português brasileiro (levando em consideração apenas a 1\aup{a} pessoa do singular (tempo presente - modo indicativo) nos permite observar algumas regularidades (padrões) dentre os verbos irregulares:\\

\begin{center}

Bobear – Bobeio, Bloquear – Bloqueio, Chatear – Chateio, Clarear – Clareio, Golpear – Golpeio;\\

Agredir – Agrido, Conseguir – Consigo, Inserir – Insiro, Perseguir – Persigo, Preferir – Prefiro, Proferir – Profiro, Repetir – Repito, Servir –  Sirvo, Vestir – Visto;\\

Cobrir – Cubro, Dormir – Durmo, Engolir – Engulo;\\

 Al[e]gar – Al[ε]go, C[e]gar – C[ε]go, Compl[e]tar – Compl[ε]to,  Col[e]tar – Col[ε]to, Entr[e]gar – Entr[ε]go, Pr[e]gar – Pr[ε]go;\\

Ad[o]rar – Ad[\textopeno]ro, Ad[o]tar – Ad[\textopeno]to, B[o]tar – B[\textopeno]to, C[o]lar – C[\textopeno]lo, F[o]car – F[\textopeno]co, M[o]rar – M[\textopeno]ro, S[o]ltar – S[\textopeno]lto, S[o]lar – S[\textopeno]lo, T[o]car – T[\textopeno]co, M[o]strar – M[\textopeno]stro;\\

Mentir - Minto, Sentir - Sintu;

\end{center}

Os padrões observados a partir da exposição de algumas classes irregulares, permitem, assim como no inglês, a proposição de fórmulas, ou regras fonéticas, que explicam as flexões realizadas em cada classe. É possível notar, por exemplo, que um verbo da mesma família de \textit{conseguir} segue a regra:

% Inserir regrinha formal
\begin{center}
e $\rightarrow$ i/\_C*]ir \footnote{C* indica uma sequência de uma ou mais consoantes}
\end{center}


As previsibilidades encontradas sugerem não somente a possibilidade de elaboração das regras, como também a possibilidade do desenvolvimento de redes capazes de capturar tais dependências. Apesar disso, deve-se destacar que a língua portuguesa apresenta um comportamento diferente da língua inglesa. Uma análise foi realizada a partir de uma base de verbos irregulares do inglês em notação fonética (retirados de \url{https://www.apronus.com/learn_english/irregularverbs.htm}) e concluiu-se que a média de número de fonemas por verbo é de 4.16 fonemas. Em contrapartida, construiu-se uma base de verbos da línguas portuguesa e concluiu-se que a média de número de fonemas por verbo é de 6.39 fonemas.

\subsection{Motivação no campo da Computação}

Desde a apresentação do modelo conexionista em 1986 (ref), os modelos de redes neurais já passaram por diversos avanços. As melhorias nos poderes computacionais em conjunto com o acesso a grandes quantidades de dados impulsionaram as pesquisas na área de forma a facilitar o desenvolvimento de novas arquiteturas. Entre elas, destaca-se o Encoder-Decoder (ref), uma arquitetura introduzida em 2014. O Encoder-Decoder é uma arquitetura especialmente conhecida pelo seu bom desempenho em tarefas linguísticas (ref). Sendo assim, parte da motivação desta pesquisa consiste na avaliação desse modelo na tarefa de aprendizado de flexão de verbos irregulares. 


% ------------------------------------------------------------------------
\section{Objetivo}
\label{sec:objectives}

Esse trabalho tem como principal objetivo revisitar a questão de aprendizado de verbos irregulares do Português Brasileiro através de um novo tipo de modelagem de rede neural, o Encoder-Decoder. 

Por se tratar de um trabalho de caráter experimental no campo da Linguística Computacional, em nenhum momento se prestará à defesa de uma corrente teórica em aquisição, porém apresenta novos resultados que poderão fomentar futuras discussões sobre o assunto. 

O desenvolvimento dessa pesquisa também resultou na construção de um Corpus que compõe 423 verbos flexionados na primeira pessoa do singular do modo indicativo, suas respectivas formas no infinitivo e suas respectivas transcrições fonéticas\footnote{Realizadas com base na chave de transcrição a ser apresentada no Capítulo \ref{ch:02}}.


% ------------------------------------------------------------------------
\section{Organização}
\label{sec:organization}

O Capítulo \ref{ch:02} expõe o embasamento teórico linguístico necessário para a construção dos dados de treinamento do modelo. 

O Capítulo \ref{ch:03} apresenta uma introdução aos modelos de redes neurais e também introduz os conceitos de Modelo de Linguagem e Redes Neurais Recorrentes - conceitos imprescindíveis para o entendimento do modelo final desenvolvido.

O Capítulo \ref{ch:05} exibe a teoria do modelo Encoder-Decoder e em seguida a aplicação do mesmo para a questão do aprendizado de flexão dos verbos. Também apresenta uma seção dedicada à formação do Corpus utilizado.

O Capítulo \ref{ch:07} apresenta os resultados obtidos pelo modelo.

O Capítulo \ref{ch:08} expõe uma discussão sobre os resultados obtidos e as dificuldades encontradas. Também destaca sugestões para pesquisas futuras sobre o assunto.

No Apêndice encontram-se disponíveis todas as predições do modelo e o Corpus completo utilizado.

