\chapter{Introdução}
\label{ch:01}

A questão do aprendizado infantil, referente ao processo de flexão de verbos irregulares na língua inglesa, está certamente entre um dos temas de debate mais controversos %entre as principais correntes teóricas no estudo da 
do campo da Linguística (Pinker, \citeyear{Pinker:1999}). O cerne do debate está na exata caracterização dos mecanismos que possibilitam que um falante seja capaz de relacionar um verbo na forma não flexionada (\textit{walk}, por exemplo) à sua forma flexionada no \textit{Simple Past} (\textit{walked}).

Os verbos no tempo passado do inglês podem ser subdivididos em uma variedade de famílias. Um primeiro grupo é a forma aceita como a \textit{regular}, cuja forma ortográfica corresponde à aplicação da regra \textit{\text{stem} + ed}, como em \textit{walk}. %Entretanto, sob uma perspectiva fonológica, ainda é possível dividir esse grupo em três menores seguindo as variações possíveis do segmento \textit{ed}: (i) [-\textsci d], (ii) [-d] e (iii) [-t]. Em (i), observa-se que o segmento [-\textsci d] é utilizado sempre nos casos em que o último fonema do \textit{stem} for um [t] ou [d], como por exemplo o verbo \textit{pad} $\rightarrow$ \textit{padded} ([p\ae d] $\rightarrow$ [p\ae d\textsci d]). A situação (ii) é utilizada sempre que último fonema do \textit{stem} for uma vogal ou uma consoante sonora, como por exemplo em \textit{drag}$\rightarrow$ dragged ([dr\ae g]) $\rightarrow$ [dr\ae gd] ou show$\rightarrow$ showed ([\textesh o\textupsilon]$\rightarrow$[\textesh o\textupsilon d]). Por sua vez, o caso (iii) é aplicado sempre após uma consoante surda (\textit{sack}$\rightarrow$ sacked ([s\ae k] $\rightarrow$ [s\ae kt]). 
% ocorrendo não somente a formação de um padrão regular (que ortograficamente corresponde à composição verbo + \textit{ed}), mas também à formação de subgrupos de verbos irregulares, como por exemplo:
Dentre os verbos irregulares, estes podem ser considerados ou supletivos, como por exemplo \textit{go} $\rightarrow$ \textit{went}, ou podem se conglomerar seguindo padrões fonéticos de flexão similares (\cite{Nelson:2010}):

\begin{enumerate}
    \item blow – blew, grow – grew, know – knew, throw – threw
    \item bear – bore, swear – swore, tear – tore, wear – wore
    \item drink – drank, shrink – shrank, sink – sank, stink – stank 
\end{enumerate}

É possível pensar que o aprendizado de tais padrões dependeria de uma memorização caso a caso. No entanto, a pesquisa de \cite{Bybee:1983} mostra um estudo psicolinguístico em que indivíduos são apresentados a diversos verbos inventados (hipoteticamente em uma forma não flexionada). A pesquisa revelou que, ao invés de aplicarem sistematicamente a regra regular (verbo + \textit{ed}), os indivíduos apresentaram tendências à alocação de alguns verbos em alguns subgrupos irregulares. % um estudo psicolinguístico conduzido a partir da  verbos inventados revelou que os indivíduos apresentam tendências com relação à alocação dos verbos também em grupos de verbos irregulares, ao invés de sistematicamente aplicar a regra regular ( + \textit{ed}). 
Por exemplo, para o verbo inventado “\textit{spling}”, a maioria dos indivíduos optou pela forma “\textit{splang}”  ou “\textit{splung}”. Este exemplo contradiz a ideia de que os falantes poderiam estar apenas reproduzindo formas memorizadas e sugere que eles estejam ativamente identificando padrões, e mais: possuem uma intuição natural sobre a adequabilidade da alocação de um verbo a um grupo de verbos ou a outro. 

A partir do exemplo dado, é razoável deduzir que a motivação por de trás de tais tendências ocorra a partir das similaridades entre as unidades fonéticas dos verbos inventados e os verbos reais que já apresentam uma flexão de caráter irregular. Entretanto, as circunstâncias que levam à aquisição dessa \textit{intuição} linguística são indeterminadas. Por um lado, faz sentido dizer que para que um ser humano seja capaz de introduzir-se ao mundo dos falantes, é necessário que ele seja dotado de algumas pré-disposições para tal, caso contrário seria possível ensinar essa forma de comunicação para outras espécies. Em contrapartida, estudos mostram que crianças privadas do contato com uma sociedade falante se tornam permanentemente incapazes de dominar integralmente a gramática de uma língua (\cite{Pinker:languageinstinct}), o que nos leva a concluir que a experiência das crianças com a sociedade, assim como as suas próprias pré-disposições genéticas são parcialmente responsáveis pelo processo de desenvolvimento da linguagem. A dificuldade, está portanto, na tentativa de se quantificar, delimitar e apontar os conhecimentos adquiridos a partir do contato cultural, bem como os conhecimentos linguísticos ditos \textit{inatos}. É, portanto, em torno desta questão que tem início o debate a respeito do aprendizado dos verbos irregulares da língua inglesa.

%https://plato.stanford.edu/entries/rationalism-empiricism/


De um lado do debate, encontra-se a teoria da Fonologia Gerativa de Chomsky e Halle (\citeyear{chomsky:1968}). Nesta teoria, os indivíduos seriam portadores de um dispositivo de aquisição de linguagem (\textit{LAD} - Language Acquisition Device) responsável pela \textit{formulação} e \textit{manipulação} de estruturas fonológicas abstratas em um sistema intrincado de regras. De modo simplificado, a teoria propõe que o falante seja capaz de identificar e formular regras intuitivamente para dar conta do aprendizado das formas irregulares da língua. Um exemplo disso é a família dos verbos terminados em “-ind”.

\begin{center}
bind – bound, find – found, grind – ground, wind – wound
\end{center}

Vemos que, de modo simplificado, pode-se propor uma regra baseada em uma gramática sensível a contextos (Context-Sensitive Grammar (CSG) (ref)) 
%verificar essa formula fonetica

\begin{center}
a\textsci $\rightarrow$ a\textupsilon / \textbf{X}  \_\_nd]+past
\end{center}

A regra proposta sugere que o segmento [a\textsci] se transforma em [a\textupsilon] quando terminado em [nd] e flexionado para o passado. O símbolo \_\_ representa o local aonde ocorre tal transformação e \textbf{X} representa uma unidade fonológica arbitrária. 

Em outras palavras, pode-se dizer que o conhecimento dito \textit{inato} defendido por Chomsky e Halle refere-se a uma certa capacidade cognitiva de formulação de regras a partir da identificação de alguns elementos fundamentais (como por exemplo os elementos apontados na regra proposta). Uma estrutura como essa permitiria ao falante construir generalizações e, eventualmente, abstrair as regras fonológicas de sua língua. \\

Do outro lado do debate, os pesquisadores \cite{rumelhart:1986} confrontam a teoria anterior ao argumentar que comportamentos de caráter regrado podem ser reproduzidos por mecanismos que não dependam de nenhuma manipulação simbólica. Ao invés disso, os pesquisadores sugerem que os mecanismos envolvidos no processo de flexão verbal possam ser construídos de tal forma que a sua performance possa ser caracterizada por regras, mas que as regras em si não estejam representadas explicitamente em nenhuma parte do processo. Para sustentar essa ideia, \cite{rumelhart:1986} apresentam um modelo computacional baseado em padrões associativos que não fazem uso de construções com regras desse tipo. Posteriormente, o modelo construído foi fundamental para o surgimento de uma nova escola dentro das ciências cognitivas: o conexionismo.\\

\input{definitions/colors}
\input{definitions/styles}
\begin{figure}[ht!]
\centering

\scalebox{1.0}{
\begin{tikzpicture}[auto]

% operations =========
% phon features 1
\node[textonly] (1pho1) {int-vogal-int};

% Legenda
\node[textonly, above=10pt of 1pho1] (leg1) {Unidades de Input};


% FNN input
\node[normal, right=5pt of 1pho1] (x1) {};
\node[normal, below=25pt of x1] (x2) {};
\node[normal, below=25pt of x2] (x3) {};
\node[normal, below=25pt of x3] (x4) {};
\node[normal, below=25pt of x4] (x5) {};
\node[normal, below=25pt of x5] (x6) {};
\node[text, below=10pt of x6] (nada) {};

% FNN output
\node[normal, right=45pt of x1] (y1) {};
\node[normal, right=45pt of x2] (y2) {};
\node[normal, right=45pt of x3] (y3) {};
\node[normal, right=45pt of x4] (y4) {};
\node[normal, right=45pt of x5] (y5) {};
\node[normal, right=45pt of x6] (y6) {};

% phon features 2
\node[textonly, right=5pt of y1] (2pho1) {int-vogal-int};
\node[textonly, above=10pt of 2pho1] (leg2) {Unidades de Output};
\node[textonly, left=25pt of x2] (1pho2) {anterior-nasal-posterior};
\node[textonly, right=25pt of y2] (2pho2) {anterior-nasal-posterior};
\node[textonly, left=25pt of x3] (3pho1) {...};
\node[textonly, right=25pt of y3] (1pho3) {...};
\node[textonly, left=25pt of x4] (4pho1) {nasal-cont-ocl};
\node[textonly, right=25pt of y4] (1pho4) {nasal-cont-ocl};
\node[textonly, left=25pt of x5] (5pho1) {médio-cont-baixa};
\node[textonly, right=25pt of y5] (1pho5) {médio-cont-baixa};
\node[textonly, left=25pt of x6] (6pho1) {vogal-fric-\#};
\node[textonly, right=25pt of y6] (1pho6) {vogal-fric-\#};
% edges FNN
\path[nedge] (x1) -- (y1);
\path[nedge] (x1) -- (y2);
\path[nedge] (x1) -- (y3);
\path[nedge] (x1) -- (y4);
\path[nedge] (x1) -- (y5);
\path[nedge] (x1) -- (y6);
\path[nedge] (x2) -- (y1);
\path[nedge] (x2) -- (y2);
\path[nedge] (x2) -- (y3);
\path[nedge] (x2) -- (y4);
\path[nedge] (x2) -- (y5);
\path[nedge] (x2) -- (y6);
\path[nedge] (x3) -- (y1);
\path[nedge] (x3) -- (y2);
\path[nedge] (x3) -- (y3);
\path[nedge] (x3) -- (y4);
\path[nedge] (x3) -- (y5);
\path[nedge] (x3) -- (y6);
\path[nedge] (x4) -- (y1);
\path[nedge] (x4) -- (y2);
\path[nedge] (x4) -- (y3);
\path[nedge] (x4) -- (y4);
\path[nedge] (x4) -- (y5);
\path[nedge] (x4) -- (y6);
\path[nedge] (x5) -- (y1);
\path[nedge] (x5) -- (y2);
\path[nedge] (x5) -- (y3);
\path[nedge] (x5) -- (y4);
\path[nedge] (x5) -- (y5);
\path[nedge] (x5) -- (y6);
\path[nedge] (x6) -- (y1);
\path[nedge] (x6) -- (y2);
\path[nedge] (x6) -- (y3);
\path[nedge] (x6) -- (y4);
\path[nedge] (x6) -- (y5);
\path[nedge] (x6) -- (y6);


\end{tikzpicture}
}\caption{Esquema do Modelo Apresentado pelos Pesquisadores Rumelhart e McClelland} 
\label{fig:esquemafdd}
\end{figure}


O modelo desenvolvido foi criado por analogia à estrutura em que se relacionam os neurônios no cérebro. Ele é composto basicamente por uma rede artificial de nódulos interconectados paralelamente (Fig. \ref{fig:esquemafdd}).

A primeira camada de nódulos do modelo é responsável por receber os dados de entrada (os \textit{inputs}), que são os dados referentes aos traços fonéticos distintivos que caracterizam os sons de um verbo não flexionado. Traços fonéticos podem ser caracterizados como propriedades distintivas das unidades fônicas (\cite{paraconhecer:2015}). Tais propriedades podem ser baseadas em critérios acústicos, articulatórios ou perceptuais. Na figura, cada nódulo é apresentado ao lado de uma sequência de três traços. O primeiro nódulo, por exemplo, refere-se à sequência \textbf{int-vogal-int}. Nesse caso, \textbf{int} (uma abreviação para \textit{interrompida}) indica uma propriedade comum entre algumas consoantes, referente à interrupção do fluxo de ar (como no fone [k], por exemplo). Na figura temos ainda \textbf{fric} para fricativas; \textbf{ocl} para oclusivas; vogais; \textbf{nasal} para nasalidade; locais da execução (anterior e posterior); traços de corpo da língua (média, baixa); entre outros. 

Ainda sobre a camada de \textit{input}, é possível observar que cada nódulo é representado por uma tríade de traços fonéticos. Esta foi uma solução encontrada pelos autores para realizar o mapeamento entre os traços dos verbos (da forma não flexionada para \textit{Past Simple}). A ideia era contornar a questão da inserção de dados de natureza sequencial variável (como é o caso de um verbo - composto por uma sequência de sons) de uma só vez no modelo. Nesse caso, cada tríade representa três traços fonéticos, cada um referente a um fone. Por exemplo, para o verbo \textit{came}, temos que cada um dos fones possui múltiplos traços. O fone /\textit{k}/, por exemplo, é uma consoante oclusiva, surda, anterior\footnote{O tema dos traços fonéticos utilizados será abordado em maior profundidade no Cap. \ref{ch:02}.}. Os fones subsequentes também são constituídos a partir de seus respectivos traços fonéticos. Desse modo, cada traço está associado a cada fone de uma sequência de três. Em seguida, o \textit{input} recebido é passado adiante para a próxima camada através de uma rede de conexões. 

A segunda camada é uma camada de resposta (\textit{output}) que tem como objetivo retornar dados referentes aos traços que caracterizam os sons do mesmo verbo fornecido no \textit{input}, porém no tempo passado. Concluída esta etapa, os dados de saída obtidos deverão ser então comparados à forma correta do verbo no tempo passado, através de uma espécie de gabarito, usualmente conhecido como \textit{alvo} ou também \textit{target} (Fig.\ref{fig:gabarito}). Feita essa comparação, é possível alterar a rede de conexões entre as camadas de \textit{input} e \textit{output} de modo a fortalecer (ou enfraquecer) as mesmas para atingir o objetivo proposto. Antes da primeira comparação, a rede é inicializada com conexões aleatórias. Conforme o número de comparações aumenta, a tendência é que as atualizações realizadas tenham servido como um aprendizado (uma espécie de \textit{treinamento}) e permitam que o modelo seja utilizado para flexionar os verbos.

\input{definitions/colors}
\input{definitions/styles}
\begin{figure}[h]
\centering

\scalebox{1.0}{
\begin{tikzpicture}[auto]

% operations =========
% phon features 1
\node[textonly] (out1) {Output};
\node[textonly, right=25pt of out1] (gab) {Target};


% FNN output
\node[normal, below=40pt of out1] (x1) {$y_{1}$};
\node[normal, below=35pt of x1] (x2) {$y_{2}$};
\node[normal, below=35pt of x2] (x3) {$y_{3}$};

% from input
\node[text, left=45pt of x1] (in1) {};
\node[text, left=45pt of x2] (in2) {};
\node[text, left=45pt of x3] (in3) {};


% comparison
\node[text, right=31pt of x1] (nada1) {};
\node[text, below=5pt of nada1] (nada2) {\small{Comparação}};
\node[text, right=31pt of x2] (nada6) {};
\node[text, right=31pt of x3] (nada7) {};

\node[text, left=31pt of x1] (nada3) {};
\node[text, left=31pt of x2] (nada4) {};
\node[text, left=31pt of x3] (nada5) {};


% FNN target
\node[normal, right=65pt of x1] (y1) {$\hat{y_{1}}$};
\node[normal, right=65pt of x2] (y2) {$\hat{y_{2}}$};
\node[normal, right=65pt of x3] (y3) {$\hat{y_{2}}$};
\node[text, below=15pt of x3] (nada) {};



% edges FNN
\path[arrows_dashed] (x1) -- (y1);
\path[arrows_dashed] (x2) -- (y2);
\path[arrows_dashed] (x3) -- (y3);

\draw[arrows_dashed, ->] (nada1) to [out=135,in=115] (nada3);
\draw[arrows_dashed, ->] (nada6) to [out=135,in=115] (nada4);
\draw[arrows_dashed, ->] (nada7) to [out=135,in=115] (nada5);

\path[tedge] (in1) -- (x1);
\path[tedge] (in2) -- (x1);
\path[tedge] (in3) -- (x1);

\path[tedge] (in1) -- (x2);
\path[tedge] (in2) -- (x2);
\path[tedge] (in3) -- (x2);

\path[tedge] (in1) -- (x3);
\path[tedge] (in2) -- (x3);
\path[tedge] (in3) -- (x3);



\end{tikzpicture}
}\caption{Comparações entre o \textit{Output} e o \textit{Target}} 
\label{fig:gabarito}
\end{figure}

Para realizar o treinamento, \cite{rumelhart:1986} introduzem 420 verbos no modelo repetidamente (200 vezes cada um, 84.000 inserções no total).  Após o treinamento, o modelo foi capaz de prever corretamente todos os 420 verbos inseridos. Além disso, em um novo conjunto com 86 verbos desconhecidos, acertou cerca de 3/4 dos verbos regulares presentes. Dentre os novos verbos irregulares, cometeu erros interessantes de \textit{regularização} (como \textit{catched} (ao invés de caught) e \textit{digged} (ao invés de dug). %(\cite{Pinker:1999}).  

Além desses resultados, \cite{rumelhart:1986} relatam que o processo de aprendizado do modelo apresentou um fenômeno interessante, reproduzindo um desempenho similar a comportamentos observáveis em crianças durante a fase de aquisição: a Curva de Desenvolvimento em U (U-shaped Development, \cite{marcus:1992}). A Curva de Desenvolvimento em U basicamente se refere a um processo de aprendizado que ocorre em três estágios: \\

(i) inicialmente, crianças aprendem a flexionar verbos corretamente (\textit{come}$\rightarrow$\textit{came});

(ii) em seguida passam por um processo de \textit{super-regularização} (em que produzem formas como \textit{comed}), conforme são capazes de assimilar uma quantidade maior de verbos e compreender que existe uma fórmula genérica que funciona quase sempre;

(iii) por fim, elas entendem (intuitivamente) que a fórmula regular não dá conta de todos os casos e passam a reproduzir corretamente tanto os verbos regulares quanto irregulares. \\
% micro e macro u shaped development


\cite{rumelhart:1986} descrevem como foi possível observar tal comportamento também no modelo computacional desenvolvido.
Na fase inicial do processo de treinamento, o modelo foi alimentado com uma quantidade pequena de verbos, como: \textit{come}, \textit{get}, \textit{give}, \textit{look}, \textit{take}, \textit{go}, \textit{have}, \textit{live} e \textit{feel}. A performance do modelo foi compatível com o primeiro estágio da curva, ou seja, para esses verbos foi capaz de identificar corretamente a forma correspondente no \textit{Simple Past}. Em um segundo momento, o modelo foi alimentado com uma quantidade muito maior de verbos. Nesse estágio é possível verificar que o modelo está passando por um processo de regularização sistemática dos verbos. Ele produziu resultados como: \textit{breaked}, \textit{comed}, \textit{gived}; e também combinações entre padrões regulares e irregulares (ex. \textit{gaved}). 
Após uma série de muitas inserções repetidas, o modelo finalmente foi capaz de responder corretamente a uma quantidade maior de verbos, assim como no último estágio do processo de aprendizado natural. 

Os resultados de \cite{rumelhart:1986} foram capazes de causar bastante alvoroço na comunidade científica da época. Muitos pesquisadores viam o novo modelo como uma completa mudança de paradigma, não apenas na Linguística, mas também como uma nova forma de se estudar aprendizado em geral (\cite{Schneider1987}). 

Apesar disso, \cite{Pinker:1988} dão continuidade ao debate ao apontar uma série de questões pertinentes em que o modelo falhou em explicar. Primeiramente, como o modelo recebe apenas uma representação fonética do verbo como \textit{input}, ele é incapaz de gerar duas respostas diferentes para verbos com sonoridade idêntica (por exemplo \textit{break}$\rightarrow$\textit{broke} e \textit{brake}$\rightarrow$\textit{braked}). Para realizar essas predições corretamente, o modelo precisaria de um módulo adicional para distinguir entre as duas palavras, o que o descaracterizaria como modelo puramente associativo. Em segundo lugar, o modelo é extremamente dependente dos padrões observados durante o treinamento, tendo uma capacidade baixa para generalizações. \cite{Pinker:1999} comenta que o modelo ficou mudo quando alimentado com os verbos \textit{jump}, \textit{pump}, \textit{warm, trail} e \textit{glare} (que dispõem de uma sonoridade razoavelmente incomum). Além disso, o modelo apresentou alguns resultados completamente distorcidos, como: \textit{squat – squakt, tour – toureder} e \textit{mail – membled}; associações inaceitáveis para qualquer falante nativo. 

Com relação ao padrão de aprendizado observado (a Curva de Desenvolvimento em U), \cite{Pinker:1999} explica que esse comportamento foi provocado segundo a forma em que os verbos foram inseridos no modelo durante o treinamento: Rumelhart & McClelland realizaram o treinamento \textit{em partes}. Na primeira parte do treinamento, alimentaram o modelo com alguns verbos de alta frequência na língua inglesa (muitos deles irregulares). Em seguida, treinaram o modelo com esses verbos, reintroduzindo-os múltiplas vezes até que o modelo conseguisse atingir um alto desempenho nesses verbos. Depois introduziram uma quantidade maior de verbos, sendo estes menos frequentes que os anteriores mas em sua maioria regulares. Dessa forma, o modelo começou a se ajustar para aplicar a regra regular e assim foi possível observar o comportamento pretendido. Ainda, segundo \cite{pluket:1991}, os estágios de desenvolvimento (i), (ii) e (iii) podem ser considerados parte de um comportamento \textit{macro U-shape}, mas ainda é possível observar a ocorrência de um comportamento \textit{micro U-shape}. \cite{pluket:1991} observam que a reprodução dos verbos irregulares oscila bastante entre flexões corretas e \textit{super-regularizadas}. Eles também notam que estas oscilações ocorrem em proporções diferentes para cada verbo e que crianças raramente “\textit{irregularizam}“ verbos regulares (como \textit{ping}$\rightarrow$\textit{pang}), fato que foi observado no modelo (\textit{gaved}).\\



\\
Pinker e Prince \citeyear{Pinker:1988} apresentam a formulação de uma nova teoria linguística para tal questão: uma teoria híbrida em que a fonologia gerativa se aplica ao processo de flexão regular e um modelo conexionista se aplica ao processo de flexão irregular. Os pesquisadores propõem que as formas regulares sejam computadas a partir de um mecanismo que deve abstrair o radical do verbo e combiná-lo ao sufixo –ed.  Tal mecanismo pode ser aplicado a qualquer palavra, em um processo independente da memória. As formas irregulares, por sua vez, passam por um processo diferente: verbos irregulares precisam passar por um processo de memorização, uma memorização que ocorre de maneira associativa, havendo não somente a associação entre um verbo e outro mas também entre as propriedades (traços fonéticos, rima, stem, núcleo, etc.) de um verbo e de outro, parecido com o que foi proposto por Rumelhart e McClelland.

\section{Motivação}
\label{sec:motivation}

\subsection{Motivação no campo da Linguística}

\subsubsection{Complexidade da Língua Portuguesa}

A morfologia verbal da língua Inglesa é bastante simples, se comparada à Portuguesa. Em primeiro lugar, os verbos do Português se distribuem em três classes módicas (\textit{conjugações}), sendo cada uma destas definida a partir de uma \textit{vogal temática} (\textit{/a/}, \textit{/e/} e \textit{/i/}). Dado um verbo em sua forma infinitiva, por exemplo \textit{Amar}, a vogal temática é a aquela que se encontra entre o morfema lexical do verbo (o radical) e o fone \textit{r}.

\begin{align*}
    \text{Am + a + r}\\
    \text{Radical + VT + r} 
\end{align*}

Com isto, os três possíveis tipos de conjugação são: 1\aup{a} - ar (amar, brigar), 2\aup{a} - er (beber, comer) e 3\aup{a} (rir, descobrir). Na língua Inglesa, essa distinção não existe. 

Outra diferença entre as línguas pode ser observada com relação às pessoais gramaticais das línguas. No Inglês, com exceção do grupo \textit{to be} que apresenta três formas possíveis (considerando o tempo presente): (i) I \textit{am}, (ii) he/she/it \textit{is} e (iii) they \textit{are}; o restante dos verbos apresenta apenas duas: a forma base para \textit{I, We} e \textit{They} (por exemplo, \textit{walk}) e base + s para \textit{he/she/it }(\textit{walks}). No \textit{Past Simple} apresenta maior número de formas também apenas para o grupo \textit{to be}: (i) I/he/she/it \textit{was} e (ii) They/We \textit{were}, os demais não apresentam marcação \cite{Nelson:2010}. 

No Português, a norma tradicional distingue seis pessoas: 1\aup{a}: Eu, 2\aup{a}: Tu, 3\aup{a}: Ele/Ela, 4\aup{a}: Nós, 5\aup{a}: Vós, 6\aup{a}: Eles. Mesmo com a exclusão da 2\aup{a} e da 5\aup{a} pessoa (cujo uso está em decadência (\cite{1999:camara}), o número de formas possíveis para cada verbo é o dobro do número de opções do Inglês.

Com relação às irregularidades, no Inglês os verbos irregulares encontram-se apenas no \textit{Simple Past} e \textit{Past Participle}, enquanto que o sistema verbal do Português é repleto de irregularidades em todos os tempos verbais (\cite{wuerges:2014}).

\subsubsection{Aprendizado de Verbos na Língua Portuguesa}
\label{sec:aprendizado_port}

Uma criança em processo de aquisição de linguagem no sistema do português brasileiro é posta a superar todas as complexidades mencionadas. Uma parte do processo é justamente perceber a relação entre a vogal temática e as possíveis conjugações verbais regulares. Nesse processo não é incomum observarmos o surgimento de trocas de conjugação. \cite{wuerges:2014} apresenta dados linguísticos produzidos por crianças com diversas destas trocas: 


\begin{table}[H]
\begin{center}
\begin{tabular}{cccc}
Verbo & Execução & Objetivo & Troca  \\ \hline
botar & “eu boti“ & botei & 1\aup{a} com 2\aup{a} ou 3\aup{a} \\
comer & “eu comei“ & comi & 2\aup{a} com 1\aup{a} \\
jantar & “eu janti“ & jantei & 1\aup{a} com 2\aup{a} ou 3\aup{a} \\ \hline
& & & 
\end{tabular}
\caption{Exemplos de Trocas de Conjugação Durante o Processo de Aquisição Verbal}
\label{tab:aquisicao}
\end{center}
\end{table}

As formas verbais irregulares apresentam-se como uma dificuldade adicional nesse processo para as crianças falantes da língua portuguesa. \cite{wuerges:2014} também aponta exemplos observados de \textit{regularização} de verbos irregulares: “eu \textit{consego}” (regularização do verbo conseguir) e “eu \textit{podo}” (regularização do verbo poder).

Um verbo é dito irregular se apresentar alterações no radical (em relação ao radical da forma infinitiva) e/ou no sufixo flexional (em relação ao padrão regular imposto por cada conjugação) (\cite{wuerges:2014}). Os sufixos flexionais (SF) são basicamente os segmentos acrescentados após o radical do verbo. Eles podem ser divididos em dois tipos: (i) sufixo modo-temporal (SMT) e (ii) sufixo número-pessoal (SNP). Para o verbo “gostaremos”, por exemplo, considera-se o segmento \textit{/gost/} como o radical do verbo, \textit{/o/} como o sufixo flexional, que neste caso marca simultaneamente modo indicativo, 1\aup{a} pessoa do singular e tempo Presente.


% \cite{1972:camara} define a estrutura dos verbos da seguinte forma:

% \begin{equation}
%     T (R + VT) + SF (SMT + SNP)
% \end{equation}

% Em que T representa o tema do verbo (composto pelo radical (R) e seguido por uma vogal temática (VT)). SF representa o sufixo flexional do verbo, composto pelos sufixos modo-temporal e número-pessoal. Nessa fórmula, leva-se em conta a alomorfia de cada um dos sufixos flexionais e a possibilidade de zero (ø) para um deles ou ambos. Para o verbo “gostaremos”, por exemplo, considera-se o segmento “\textit{gost}” como o radical do verbo, “\textit{a}” como vogal temática  e “\textit{emos}” como o sufixo flexional, que neste caso marca simultaneamente modo indicativo, 1\aup{a} pessoa do plural e tempo Futuro do Presente.

Seguindo a definição proposta, é necessário reforçar que o interesse deste estudo está em capturar irregularidades no nível fonético, portanto verbos como: “gosto”, “boto” e “coloco”, cuja ortografia apresenta o padrão regular; serão classificados como irregulares. A Tabela \ref{tab:irreg} exibe alguns exemplos das classificações realizadas.

\begin{center}
\begin{table}[H]
\centering
\begin{tabular}{ccc}
\multicolumn{1}{l}{\textbf{Verbo Infinitivo}} & \multicolumn{1}{l}{\textbf{Verbo Flexionado}} & \multicolumn{1}{l}{\textbf{Classificação}} \\ \hline
Falar & Falo & Regular \\
Gostar & Gosto & Irregular \\
Testar & Testo & Irregular \\
Ansiar & Anseio & Irregular \\
Pedir & Peço & Irregular \\
Mentir & Minto & Irregular \\
Por & Ponho & Irregular
\end{tabular}
\caption{Exemplos de Classificações de Verbos Quanto a Presença de Irregularidades}
\label{tab:irreg}
\end{table}
\end{center}

Uma análise sobre a disposição das irregularidades presentes no português brasileiro (levando em consideração apenas a 1\aup{a} Pessoa do Singular (tempo Presente e modo Indicativo) nos permite observar algumas regularidades (padrões) dentre os verbos irregulares:\\

\begin{center}

Bobear – Bobeio, Bloquear – Bloqueio, Chatear – Chateio, Clarear – Clareio, Golpear – Golpeio;\\
\\
Agredir – Agrido, Conseguir – Consigo, Inserir – Insiro, Perseguir – Persigo, Preferir – Prefiro, Proferir – Profiro, Repetir – Repito, Servir –  Sirvo, Vestir – Visto;\\
\\
Cobrir – Cubro, Dormir – Durmo, Engolir – Engulo;\\
\\
 Al[e]gar – Al[ε]go, C[e]gar – C[ε]go, Compl[e]tar – Compl[ε]to,  Col[e]tar – Col[ε]to, Entr[e]gar – Entr[ε]go, Pr[e]gar – Pr[ε]go;\\
\\
Ad[o]rar – Ad[\textopeno]ro, Ad[o]tar – Ad[\textopeno]to, B[o]tar – B[\textopeno]to, C[o]lar – C[\textopeno]lo, F[o]car – F[\textopeno]co, M[o]rar – M[\textopeno]ro, S[o]ltar – S[\textopeno]lto, S[o]lar – S[\textopeno]lo, T[o]car – T[\textopeno]co, M[o]strar – M[\textopeno]stro;\\
\\
Mentir - Minto, Sentir - Sintu;

\end{center}



Os padrões observados a partir da exposição de algumas classes irregulares, permitem, assim como no inglês, a proposição de fórmulas, ou regras fonéticas, que explicam as flexões realizadas em cada classe. É possível notar, por exemplo, que um verbo da mesma família de \textit{conseguir} segue a regra:

% Inserir regrinha formal
\begin{center}
e $\rightarrow$ i/\_C*]ir 
\end{center}

A regra proposta indica que /e/ se transforma em /i/ quando em um contexto de terceira conjugação (ir). No caso, C* indica uma sequência de consoantes. 

As previsibilidades encontradas sugerem não somente a possibilidade de elaboração de regras, como também a possibilidade do desenvolvimento de redes capazes de capturar tais dependências. 

\subsubsection{Delimitação de Escopo}

Dada a maior complexidade do Português e a existência de irregularidades verbais em múltiplos paradigmas de tempos, pessoas e modos, o presente estudo se limitará a estudar apenas os padrões irregulares encontrados no paradigma da 1\aup{a} Pessoa do Singular no tempo Presente e modo Indicativo (com exemplos já explorados na Seção \ref{sec:aprendizado_port}). Desse modo, verbos que apresentem irregularidade em outro tempo, modo ou pessoa que não 1\aup{a} pessoa do singular no tempo presente e modo indicativo, serão tratados como pertencentes à classe dos regulares. Como exemplo, considere o verbo \textit{correr}. Esse verbo apresenta flexão regular para a 1\aup{a} Pessoa (\textit{corro}), mas é irregular para a 3\aup{a} Pessoa (\textit{corre}). Desse modo, apesar de \textit{correr} ser um verbo irregular, por apresentar flexão regular no paradigma escolhido, será tratado como regular para os fins dessa pesquisa.

Mesmo com a delimitação escolhida, podemos dizer que a complexidade da tarefa ainda é maior do que o exercício de aprendizado dos verbos irregulares do passado do inglês. 

 


% \subsubsection{Verbos Irregulares}

% %Em princípio, a «irregularidade» pode-se referir ao sufixo flexional, como vimos em nota ao capítulo anterior para SNP = -des, em credes, ledes, etc. Muito mais relevante há a mudança no radical, que passa a contribuir para as noções gramaticais de modo-tempo e número-pessoa. A mudança no radical é que é verdadeiramente importante e cria uma série de padrões morfológicos verbais, que vamos apreciar no presente capítulo. pg 111 camara jr
% Como foi introduzido no Capítulo \ref{ch:01}, o cerne desta pesquisa está na construção de um modelo de redes neurais que consiga capturar os processos flexionais dos verbos irregulares do português brasileiro. Para tanto, fazem-se necessárias definições e delimitações a respeito do objeto de estudo para que se possa construir uma representação vetorial que permita ao modelo capturar os padrões esperados.

% A primeira definição que deve ser feita diz respeito à escolha do paradigma conjugacional de estudo. Diferentemente do inglês que apresenta irregularidades apenas no \textit{Simple Past}, os verbos irregulares do português brasileiro se distribuem livremente entre diferentes tempos, modos e pessoas. Como observado no capítulo introdutório, o paradigma conjugacional de 1\aup{a} pessoa do singular no tempo presente e modo indicativo apresenta uma diversidade de verbos irregulares que podem ser agrupados em classes de acordo com os mesmos processos flexionais e foi portanto escolhido como o paradigma conjugacional desta pesquisa.
% Feita essa escolha, resta definir o que será considerado como verbo irregular dentro desse escopo. 


%Outra dificuldade é ter de lidar com o fato de que os verbos irregulares no português apresentam, em pelo menos uma forma verbal de seu paradigma, alterações no radical e/ou na sua desinência. Isto fica evidente quando observamos a enunciação de formas como: “eu consego*” ou “eu podo*” (poder). É interessante também notar enunciações criativas para verbos de natureza um pouco mais complicada, como o verbo por:  puso* (eu), ponhei* (eu) (\cite{wuerges:2014}).

% Uma análise sobre a disposição das irregularidades presentes no português brasileiro (levando em consideração apenas a 1\aup{a} pessoa do singular (tempo presente - modo indicativo) nos permite observar algumas regularidades (padrões) dentre os verbos irregulares:\\

% \begin{center}

% Bobear – Bobeio, Bloquear – Bloqueio, Chatear – Chateio, Clarear – Clareio, Golpear – Golpeio;\\

% Agredir – Agrido, Conseguir – Consigo, Inserir – Insiro, Perseguir – Persigo, Preferir – Prefiro, Proferir – Profiro, Repetir – Repito, Servir –  Sirvo, Vestir – Visto;\\

% Cobrir – Cubro, Dormir – Durmo, Engolir – Engulo;\\

%  Al[e]gar – Al[ε]go, C[e]gar – C[ε]go, Compl[e]tar – Compl[ε]to,  Col[e]tar – Col[ε]to, Entr[e]gar – Entr[ε]go, Pr[e]gar – Pr[ε]go;\\

% Ad[o]rar – Ad[\textopeno]ro, Ad[o]tar – Ad[\textopeno]to, B[o]tar – B[\textopeno]to, C[o]lar – C[\textopeno]lo, F[o]car – F[\textopeno]co, M[o]rar – M[\textopeno]ro, S[o]ltar – S[\textopeno]lto, S[o]lar – S[\textopeno]lo, T[o]car – T[\textopeno]co, M[o]strar – M[\textopeno]stro;\\

% Mentir - Minto, Sentir - Sintu;

% \end{center}

% Os padrões observados a partir da exposição de algumas classes irregulares, permitem, assim como no inglês, a proposição de fórmulas, ou regras fonéticas, que explicam as flexões realizadas em cada classe. É possível notar, por exemplo, que um verbo da mesma família de \textit{conseguir} segue a regra:

% % Inserir regrinha formal
% \begin{center}
% e $\rightarrow$ i/\_C*]ir \footnote{C* indica uma sequência de uma ou mais consoantes}
% \end{center}


% As previsibilidades encontradas sugerem não somente a possibilidade de elaboração das regras, como também a possibilidade do desenvolvimento de redes capazes de capturar tais dependências. Apesar disso, deve-se destacar que a língua portuguesa apresenta um comportamento diferente da língua inglesa. %Uma análise foi realizada a partir de uma base de verbos irregulares do inglês em notação fonética (retirados de \url{https://www.apronus.com/learn_english/irregularverbs.htm}) e concluiu-se que a média de número de fonemas por verbo é de 4.16 fonemas. Em contrapartida, construiu-se uma base de verbos da línguas portuguesa e concluiu-se que a média de número de fonemas por verbo é de 6.39 fonemas.

\subsection{Motivação no campo da Computação}

Desde a apresentação da pesquisa de \cite{rumelhart:1986}, o modelo associativo utilizado pelos autores já passou por diversos avanços. Na realidade, essa modelagem hoje é chamada de Rede Neural Artificial (ou também \textit{Deep Learning}) e passou a ser utilizada em uma variedade de tarefas computacionais, como classificação de imagens, classificação de texto, tradução automática, agentes conversacionais, entre outros. 

Ao longo dos últimos anos, o poder computacional foi aumentando, e com isso arquiteturas mais robustas foram desenvolvidas. Inclusive, o nome \textit{Deep Learning} surgiu em decorrência da adição de camadas intermediárias de nódulos entre as camadas de \textit{input} e \textit{output}. Tal incremento possibilita que as informações de entrada sejam distribuídas ao longo de mais conexões e com isso, o aprendizado se torna mais \textit{profundo}. Isso acontece pois o tipo de arquitetura sem camadas intermediárias consegue encontrar apenas funções com fronteiras de decisão lineares. Ao aumentar o número de camadas intermediárias, é possível ampliar o universo de soluções para a resolução de problemas mais complexos. Esse tipo de modelagem que segue um fluxo com um sentido único (do \textit{input} ao \textit{output}) é chamado de \textit{Feedforward}. Entretanto, existem muitos tipos de arquiteturas cujos fluxos não seguem essa configuração. Nesse âmbito, um tipo de arquitetura que ficou muito famoso é o \textit{Convolucional} (\textit{Convolutional Neural Networks} - CNN's) (bastante utilizado na área de visão computacional (\cite{Krizhevsky:2012}, por exemplo). No campo de tarefas linguísticas, as redes do tipo \textit{Recorrente} (\textit{Recurrent Neural Networks} - RNN's) são bastante utilizadas, uma vez que a arquitetura viabiliza a ingestão de dados sequenciais  (\cite{pengfei:2016}, por exemplo). O tema dos modelos de Redes Neurais, em especial a arquitetura de RNN's, será abordado com mais detalhes no Capítulo \ref{ch:03}.

No que diz respeito à questão do aprendizado dos verbos do \textit{Past Simple} do Inglês, uma série de novos experimentos deram sequência após as críticas de (\cite{Pinker:1988}). (\cite{pluket:1991}, \citeyear{PLUNKETT:1993}) simplificam a questão ao considerar apenas verbos de tamanho fixo (3 sílabas) e abordam o problema fazendo uso de uma arquitetura com adição de camadas intermediárias (\textit{Multi-Layered Perceptron - MLP}). Outros trabalhos transformaram a questão em um problema de \textit{classificação}, desse modo o modelo não teria mais como objetivo encontrar uma forma flexionada. Ao invés disso, teriam um conjunto finito e pré determinado de formas possíveis. \cite{Nakisa1996WhereDD}, por exemplo, classificam os plurais dos substantivos da língua Alemã. \cite{plunkett:1997} atacam o mesmo problema, porém para a língua Árabe. \cite{wetermann:1997} apresentam um modelo construído para mapear verbos não flexionados da língua Alemã para a forma no particípio. O modelo apresentado é capaz de lidar com dados de sequência variável e utiliza uma arquitetura baseada em Redes Neurais Recorrentes. Entretanto, o modelo foi construído a partir de um mecanismo de rota dupla, de modo que verbos irregulares passavam por uma rota específica de memorização. Apesar disso, o modelo falhou em capturar completamente os padrões irregulares. Segundo os autores, a explicação pode estar no fato de terem utilizado uma arquitetura de Rede Neural Recorrente muito simples, cujo treinamento apresentou problemas durante a atualização das conexões. Uma arquitetura RNN mais robusta poderia ter apresentad  


. Hoje, com o avanço nas arquiteturas e acréscimo de camadas intermediárias (entre as camadas de \textit{input} e \textit{output}, também são conhecidos como modelos de \textit{Deep Learning} (\cite{Goodfellow-et-al-2016}). Os modelos de redes neurais 
As melhorias nos poderes computacionais em conjunto com o acesso a grandes quantidades de dados impulsionaram as pesquisas na área e com isso o surgimento de novas arquiteturas. Entre elas, destaca-se o modelo Encoder-Decoder (\cite{enc-dec:2014}). O Encoder-Decoder é uma arquitetura especialmente conhecida pelo seu bom desempenho em tarefas, em especial no ramo da tradução automática. Sendo assim, parte da motivação desta pesquisa consiste na avaliação desse modelo na tarefa de aprendizado de flexão de verbos irregulares. 


% ------------------------------------------------------------------------
\section{Objetivo}
\label{sec:objectives}

Este trabalho tem como principal objetivo estudar a questão de aprendizado de verbos irregulares do Português Brasileiro através de um novo tipo de modelagem de rede neural, o Encoder-Decoder. 

Por se tratar de um trabalho de caráter experimental no campo da Linguística Computacional, em nenhum momento se prestará à defesa de uma corrente teórica em aquisição, porém apresenta novos resultados que poderão fomentar futuras discussões sobre o assunto. 

O desenvolvimento dessa pesquisa também resultou na construção de um Corpus que compõe 423 verbos flexionados na primeira pessoa do singular do modo indicativo, suas respectivas formas no infinitivo e suas respectivas transcrições fonéticas\footnote{Realizadas com base na chave de transcrição a ser apresentada no Capítulo \ref{ch:02}}.


% ------------------------------------------------------------------------
\section{Organização}
\label{sec:organization}

O Capítulo \ref{ch:02} expõe o embasamento teórico linguístico necessário para a construção dos dados de treinamento do modelo. 

O Capítulo \ref{ch:03} apresenta uma introdução aos modelos de redes neurais e também introduz os conceitos de Modelo de Linguagem e Redes Neurais Recorrentes - conceitos imprescindíveis para o entendimento do modelo final desenvolvido.

O Capítulo \ref{ch:05} exibe a arquitetura conhecida como \textit{Encoder-Decoder} e em seguida a aplicação do mesmo para a questão do aprendizado de flexão dos verbos. Também apresenta uma seção dedicada à formação do Corpus utilizado.

O Capítulo \ref{ch:07} apresenta os resultados obtidos pelo modelo.

O Capítulo \ref{ch:08} expõe uma discussão sobre os resultados obtidos e as dificuldades encontradas. Também destaca sugestões para pesquisas futuras sobre o assunto.

No Apêndice encontram-se disponíveis todas as predições do modelo e o Corpus completo utilizado.

