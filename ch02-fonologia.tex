\chapter{Representações Fonéticas}
\label{ch:02}

introdução: falar sobre o primeiro passo da construção do modelo: para entender os inputs e outputs precisamos de representaçoes foneticas que nos permitam encontrar essas relaçoes.

falar sobre a representacao letra-som e falar que tem multiplas possibilidades dessas representaçõs, comentar sobre os alfabetos AFI e SAMPA, mas explicar porque os mesmos não foram utilizados (mas poderiam). %nao foram utilizados porque eu podia ter representacoes mais simples

\section{Alfabeto fonético internacional}

\begin{center}
\scalebox{0.9}{
    \begin{tabular}{|l|cc|cc|cc|cc|cc|cc|cc|cc|cc|cc|cc|}
%\begin{tabular}{|l|cc|}
        \hline & 
            \multicolumn{2}{|c|}{\footnotesize{Bilabial}} &					% Bilabial
            \multicolumn{2}{|c|}{\footnotesize{Lab. dent.}} & 			% Labiodental
            \multicolumn{2}{|c|}{\footnotesize{Dental}} & 					% Dental
            \multicolumn{2}{|c|}{\footnotesize{Alveolar}} & 				% Alveolar
            \multicolumn{2}{|c|}{\footnotesize{P-alveo.}} & 		% Post-alveolar
            \multicolumn{2}{|c|}{\footnotesize{Retroflex}} & 				% Retroflex
            \multicolumn{2}{|c|}{\footnotesize{Palatal}} & 					% Palatal
            \multicolumn{2}{|c|}{\footnotesize{Velar}} & 					% Velar
            \multicolumn{2}{|c|}{\footnotesize{Uvular}} & 					% Uvular
            \multicolumn{2}{|c|}{\footnotesize{Pharyng.}} & 			% Pharyngeal
            \multicolumn{2}{|c|}{\footnotesize{Glottal}}  \\					% Glottal

        \hline Plosive &  							% Plosive
            p & b &													% Bilabial
            &&														% Labiodental
            \multicolumn{3}{|r}{t}&							% Dental
            \multicolumn{3}{l|}{d}&							% Alveolar
                                                                        % Post-alveolar
            \ipa{\:t} & \ipa{\:d}&									% Retroflex
            c & \textbardotlessj &														% Palatal
            k & g &													% Velar
            q & \ipa{\;G} &										% Uvular
            & \BlankCell        &								% Pharyngeal
            \ipa{P}& \BlankCell         \\								% Glottal

        \hline Nasal & 							% Nasal
            & m &													% Bilabial
            & \ipa{M} &											% Labiodental
            \multicolumn{3}{|r}{}&								% Dental
            \multicolumn{3}{l|}{n}&							% Alveolar
                                                                        % Post-alveolar
            & \ipa{\:n} &														% Retroflex
            & \textltailn &														% Palatal
            & \ipa{N} &														% Velar
            & \ipa{\;N} &														% Uvular
            \BlankCell        & \BlankCell        &		% Pharyngeal
            \BlankCell        & \BlankCell         \\		% Glottal

        \hline Trill &  								% Trill
            & \ipa{\;B}&											% Bilabial
            & &														% Labiodental
            \multicolumn{3}{|r}{}&								% Dental
            \multicolumn{3}{l|}{r}&								% Alveolar
                                                                        % Post-alveolar
            & &														% Retroflex
            & &														% Palatal
            \BlankCell        & \BlankCell        &		% Velar
            & \ipa{\;R}&											% Uvular
            & &														% Pharyngeal
            \BlankCell        & \BlankCell         \\		% Glottal

        \hline Tap/Flap &  						% Tap /Flap
            & &													% Bilabial
            & &														% Labiodental
            \multicolumn{3}{|r}{} &					% Dental
            \multicolumn{3}{l|}{\ipa{R}} &					% Alveolar
                                                                        % Post-alveolar
            & \ipa{\:r} &														% Retroflex
            & &														% Palatal
            \BlankCell        & \BlankCell        &		% Velar
            & &														% Uvular
            & &														% Pharyngeal
            \BlankCell        & \BlankCell         \\		% Glottal

        \hline Fricative & 						% Fricative
            \ipa{F} & \ipa{B} &									% Bilabial
            f & v &													% Labiodental
            \ipa{T} & \ipa{D} &									% Dental
            s & z &													% Alveolar
            \ipa{S} & \ipa{Z} &									% Post-alveolar
            \ipa{\:s} & \ipa{\:z} &								% Retroflex
            \ipa{\c{c}} & \ipa{J} &								% Palatal
            x & \ipa{G} &											% Velar
            \ipa{X} & \ipa{K} &									% Uvular
            \textcrh & \ipa{Q} &								% Pharyngeal
            h & \texthth \\										% Glottal

        \hline Lat. Fric. & 					% Lat. Fricative
            \BlankCell        & \BlankCell        &		% Bilabial
            \BlankCell        & \BlankCell        &		% Labiodental
            \multicolumn{3}{|r}{\textbeltl} &				% Dental
            \multicolumn{3}{l|}{\textlyoghlig} &			% Alveolar
                                                                        % Post-alveolar
            & &														% Retroflex
            & &														% Palatal
            & &														% Velar
            & &														% Uvular
            \BlankCell        & \BlankCell        			% Pharyngeal
            & \BlankCell        & \BlankCell         \\   % Glottal

        \hline Approx & 							% Approx.
            & &														% Bilabial
            & \ipa{V} &											% Labiodental
            \multicolumn{3}{|r}{}&								% Dental
            \multicolumn{3}{l|}{\ipa{\*r}} &					% Alveolar
                                                                        % Post-alveolar
            & \ipa{\:R} &											% Retroflex
            & j &														% Palatal
            & \textturnmrleg &									% Velar
            & &														% Uvular
            & &														% Pharyngeal
            \BlankCell        & \BlankCell         \\		% Glottal

        \hline Lat. appr. & 					% Lat. Approx
            \BlankCell        & \BlankCell        &		% Bilabial
            \BlankCell        & \BlankCell        &		% Labiodental
            \multicolumn{3}{|r}{}&								% Dental
            \multicolumn{3}{l|}{l}&								% Alveolar
                                                                        % Post-alveolar
            & \ipa{\:l} &											% Retroflex
            & \ipa{L} &												% Palatal
            & \ipa{\;L} &											% Velar
            & &														% Uvular
            \BlankCell        & \BlankCell        &		% Pharyngeal
            \BlankCell        & \BlankCell         \\		% Glottal
        \hline
    \end{tabular}
}%scalebox
\captionof{table}{Consoantes IPA}\label{tab:ipa1}
\end{center}

\begin{center}
    \begin{vowel}
        %    \putcvowel[l]{i}{1}
        \putvowel[l]{i}{0pt}{0pt}
        \putcvowel[r]{y}{1}
        \putcvowel[l]{e}{2}
        \putcvowel[r]{\o}{2}
        \putcvowel[l]{\textepsilon}{3}
        \putcvowel[r]{\oe}{3}
        \putcvowel[l]{a}{4}
        \putcvowel[r]{\textscoelig}{4}
        \putcvowel[l]{\textscripta}{5}
        \putcvowel[r]{\textturnscripta}{5}
        \putcvowel[l]{\textturnv}{6}
        \putcvowel[r]{\textopeno}{6}
        \putcvowel[l]{\textramshorns}{7}
        \putcvowel[r]{o}{7}
        \putcvowel[l]{\textturnm}{8}
        \putcvowel[r]{u}{8}
        \putcvowel[l]{\textbari}{9}
        \putcvowel[r]{\textbaru}{9}
        \putcvowel[l]{\textreve}{10}
        \putcvowel[r]{\textbaro}{10}
        \putcvowel{\textschwa}{11}
        \putcvowel[l]{\textrevepsilon}{12}
        \putcvowel[r]{\textcloserevepsilon}{12}
        \putcvowel{\textsci\ \textscy}{13}
        \putcvowel{\textupsilon}{14}
        \putcvowel{\textturna}{15}
        \putcvowel{\ae}{16}
    \end{vowel}
\captionof{table}{Vogais IPA}\label{tab:ipa2}   
\end{center} 

\section{Traços Distintivos}
uma breve introdução e comentários sobre traços fonológicos

\subsection{Representação de Rumelhart e McClelland}
uma explicação sobre a representação escolhida pelos pesquisadores
dar um exemplo de input

\subsection{Uma nova representação}

\begin{center}
\scalebox{0.9}{
    \begin{tabular}{|l|cc|cc|cc|cc|cc|cc|cc|}
        \hline & 
            \multicolumn{2}{|c|}{\footnotesize{Bilabial}} &					% Bilabial
            \multicolumn{2}{|c|}{\footnotesize{Lab. dent.}} & 			% Labiodental
            \multicolumn{2}{|c|}{\footnotesize{Alveolar}} & 				% Alveolar
            \multicolumn{2}{|c|}{\footnotesize{P-alveo.}} & 		% Post-alveolar

            \multicolumn{2}{|c|}{\footnotesize{Palatal}} & 					% Palatal
            \multicolumn{2}{|c|}{\footnotesize{Velar}} & 					% Velar
            \multicolumn{2}{|c|}{\footnotesize{Glottal}}  \\					% Glottal

        \hline Plosive &  							% Plosive
            p & b &	% Bilabial
            &&	% Labiodental
            t & d	% Alveolar
            & &% Post-alveolar
             &  &% Palatal
            & k & g 											& 		% Velar

            & \BlankCell         \\								% Glottal

        \hline Nasal & 							% Nasal
            & m 	% Bilabial
            &  &  & % Labiodental
            & n 	% Alveolar
            & & % Post-alveolar
            &  &	% Palatal
            & & N 	% Velar
            &  &          \\		% Glottal

        \hline Tap/Flap &  						% Tap /Flap
            & &													% Bilabial
            & &														% Labiodental
           & r &					% Alveolar
                                                      &&&                 % Post-alveolar
            & 													% Palatal
            \BlankCell        & \BlankCell        &		% Velar
            \BlankCell        & \BlankCell         \\		% Glottal

        \hline Fricative & 						% Fricative
            &  &									% Bilabial
            f & v &													% Labiodental
            s & z &													% Alveolar
            x & j &									% Post-alveolar
             &  &								% Palatal
            &  &											% Velar
            h &  \\										% Glottal

        \hline Lat. appr. & 					% Lat. Approx
            \BlankCell        & \BlankCell        &		% Bilabial
            \BlankCell        & \BlankCell        &		% Labiodental

           & l &							% Alveolar
                                                      &&                  % Post-alveolar
  
            & & & L & 												% Palatal
            											% Velar
            \BlankCell        & \BlankCell         \\		% Glottal
        \hline
    \end{tabular}
}%scalebox
\captionof{table}{Consoantes na nova representação}\label{tab:new_rep}
\end{center}

\begin{center}
\begin{table}[H]
\begin{center}
    \begin{tabular}{lll}
        \hline
         & Anterior & Posterior \\
         \hline
        Fechada & i & u \\
        \hline
        Meia-fechada & e & o \\
        \hline
        Meia-aberta & E & O \\
        \hline
        Aberta & a &  \\
        \hline
        Nasais & A (ã) &\\ & 3 ($\tilde{e}$) &\\ 
        \hline
    \end{tabular}
\end{center}
\caption{Vogais na nova representação}
\label{tab:new_vocals}
\end{table}
\end{center}

apresentar a minha tabela com a nova representaçao fonética

lembrar que nao ha consenso sobre o numero de vogais

dar exemplos de transcrição
% 

\section{Fonologia e Morfologia}

Classes irregulares
O portugues tem 3 conjugacoes
Paradigmas morfologicos
tradicionalmente raiz+terminacoes

definicao do que é irregular pra mim

paradigmas que espero que sejam capturados 
dar exemplos


