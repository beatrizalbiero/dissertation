\chapter{Pré-Processamento dos Verbos para Redes Neurais}
\label{ch:02}

\section{Verbos Irregulares}

Como foi introduzido no Capítulo \ref{ch:01}, o cerne desta pesquisa está na construção de um modelo de redes neurais que consiga capturar os processos flexionais dos verbos irregulares do português brasileiro. Para tanto, fazem-se necessárias definições e delimitações a respeito do objeto de estudo para que se possa construir uma representação vetorial que permita ao modelo capturar os padrões esperados.

A primeira definição que deve ser feita diz respeito à escolha do paradigma conjugacional de estudo. Diferentemente do inglês que apresenta irregularidades apenas no \textit{Simple Past}, os verbos irregulares do português brasileiro se distribuem livremente entre diferentes tempos, modos e pessoas. Como observado no capítulo introdutório, o paradigma conjugacional de 1\aup{a} pessoa do singular no tempo presente e modo indicativo apresenta uma diversidade de verbos irregulares que podem ser agrupados em classes de acordo com os mesmos processos flexionais e foi portanto escolhido como o paradigma conjugacional desta pesquisa.
Feita essa escolha, resta definir o que será considerado como verbo irregular dentro desse escopo. 

\begin{center}
\framebox{

\parbox[t][2.0cm]{10.50cm}{

\addvspace{0.2cm} \centering 

Serão considerados verbos irregulares aqueles cujas formas flexionadas apresentarem alterações no radical ou na desinência no paradigma conjugacional escolhido.

} 

}\\
\end{center}

Por essa definição, entende-se que verbos que apresentem irregularidade em outro tempo, modo ou pessoa que não 1\aup{a} pessoa do singular no tempo presente e modo indicativo, serão tratados como pertencentes à classe dos regulares. Também é necessário reforçar que o interesse deste estudo está em capturar irregularidades no nível fonético, portanto verbos como "gosto", "boto", "coloco"; serão classificados como irregulares. A Tabela \ref{tab:irreg} exibe alguns exemplos das classificações realizadas.

\begin{center}
\begin{table}[H]
\centering
\begin{tabular}{ccc}
\multicolumn{1}{l}{\textbf{Verbo Infinitivo}} & \multicolumn{1}{l}{\textbf{Verbo Flexionado}} & \multicolumn{1}{l}{\textbf{Classificação}} \\ \hline
Falar & Falo & Regular \\
Gostar & Gosto & Irregular \\
Testar & Testo & Irregular \\
Ansiar & Anseio & Irregular \\
Pedir & Peço & Irregular \\
Mentir & Minto & Irregular \\
Por & Ponho & Irregular
\end{tabular}
\caption{Exemplos de Classificações de Verbos Quanto a Presença de Irregularidades}
\label{tab:irreg}
\end{table}
\end{center}


\section{Representações Fonéticas}
\label{sec:transcr}
Na década de 60, quando foi realizado o experimento dos pesquisadores Rumelhart e McClelland, o poder computacional para a construção de modelos de redes neurais era muito baixo e forçou os pesquisadores a desenvolverem um sistema de codificação composto de várias etapas. De maneira simplificada, o sistema apresentado consistia, primeiramente, na transformação dos verbos (sequências de fones) em vetores de trigramas de fones somados a uma demarcação de início e final de palavra, representada pelo símbolo \#. Em seguida, cada um dos fones era substituído por uma lista de traços distintivos %\ref{} colocar aqui a referencia do chomsky
. Ao final, todas as possíveis combinações de três traços distintivos subsequentes foram mapeados como chaves em um dicionário de tamanho fixo em que os valores poderiam assumir somente 0 ou 1; 1 caso aquela sequência de traços estivesse presente no verbo e 0 caso contrário. A Figura \ref{fig:wick} ilustra o esquema de codificação para o verbo "came" ([kAm] segundo a chave de transcrição apresentada pelos pesquisadores). 

\definecolor{blue}{RGB}{159, 192, 176}
\definecolor{green}{RGB}{160, 227, 127}
\definecolor{orange}{RGB}{243, 188, 125}
\definecolor{red}{RGB}{253, 123, 84}
\definecolor{nephritis}{RGB}{39, 174, 96}
\definecolor{emerald}{RGB}{46, 204, 113}
\definecolor{turquoise}{RGB}{39, 174, 96}
\definecolor{green-sea}{RGB}{22, 160, 133}
\definecolor{purple}{RGB}{181, 124, 215}
% Tikzstyles for Computation Graphs

% nodes
\tikzstyle{noop} = [circle, draw=none, fill=red, minimum size = 10pt]
\tikzstyle{op} = [circle, draw=red, line width=1.5pt, fill=red!70, text=black, text centered, font=\bf \normalsize, minimum size = 25pt]

\tikzstyle{opintense} = [circle, draw=red, line width=1.5pt, fill=red!150, text=black, text centered, font=\bf \normalsize, minimum size = 25pt]


%new style
\tikzstyle{gp} = [circle, draw=red, line width=4pt, text=black, text centered, font=\bf \normalsize, minimum size = 4.cm]

\tikzstyle{box} = [rectangle, draw=red, line width=1.5pt, fill=red!70, text=black, align=center, font=\bf \normalsize, minimum size = 45pt]

\tikzstyle{box2} = [rectangle, draw=black, line width=0.9pt, text=black, align=center, font=\bf \normalsize, minimum size = 20pt]

\tikzstyle{box3} = [rectangle, draw=black, line width=0.9pt, fill=black, text=black, align=center, font=\bf \normalsize, minimum size = 20pt]

\tikzstyle{state} = [circle, draw=blue, line width=1.5pt, fill=blue!70, text=black, text centered, font=\bf \normalsize, minimum size = 25pt]

\tikzstyle{output} = [circle, draw=purple, line width=1.5pt, fill=purple!70, text=black, text centered, font=\bf \normalsize, minimum size = 25pt]


\tikzstyle{gradient} = [circle, draw=nephritis, line width=1.5pt, fill=nephritis!60, text=black, text centered, font=\bf \normalsize, minimum size = 25pt]
\tikzstyle{textonly} = [draw=none, fill=none, text centered, font=\bf \normalsize]
\tikzstyle{boxtextonly} = [draw=none, fill=none, align=center, font=\bf \normalsize]

\tikzstyle{normal} = [circle, draw=black, line width=1.0pt, fill=none, text=black, text centered, font=\bf \normalsize, minimum size = 20pt]


% edges
\tikzstyle{tedge}  = [draw, thick, >=latex, ->]
\tikzstyle{tedge_dashed}  = [draw, thick, >=latex, ->, dashed]
\tikzstyle{nedge}  = [draw, thick, >=latex]
\tikzstyle{nedge_dashed}  = [draw, thick, >=latex, dashed]


% namedscope
\tikzstyle{namedscope} = [circle, draw=orange, line width=1.5pt, fill=orange!60, align=center, inner sep=0pt]
\begin{figure}[ht!]
\centering

\scalebox{1.0}{
\begin{tikzpicture}[H]

%vetor
\node[box2] (box1) {};
\node[box2, below=0pt of box1] (box2) {};
\node[box3, below=0pt of box2] (box3) {};
\node[box3, below=0pt of box3] (box4) {};
\node[textonly, below=0pt of box4] (box5) {\reflectbox{$\vdots$}};
\node[box3, below=0pt of box5] (box6) {};
\node[box3, below=0pt of box6] (box7) {};
\node[box2, below=0pt of box7] (box8) {};
\node[box3, below=0pt of box8] (box9) {};
\node[box2, below=0pt of box9] (box10) {};
\node[textonly, below=0pt of box10] (dim) {460x1};
\node[textonly, below=0pt of dim] (space) {};
%wickelfeatures
\node[textonly, right=20pt of box1] (wi1) {\#, oclusiva, média};
\node[textonly, right=20pt of box2] (wi2) {\#, anterior, fricativa};
\node[textonly, right=20pt of box3] (wi3) {\#, fricativa, vogal};
\node[textonly, right=20pt of box4] (wi4) {\#, contínua, anterior};
\node[textonly, right=20pt of box5] (wi5) {\reflectbox{$\vdots$}};
\node[textonly, right=20pt of box6] (wi6) {\#, não vozeada, aberta};
\node[textonly, right=20pt of box7] (wi7) {\#, contínua, posterior};
\node[textonly, right=20pt of box8] (wi8) {contínua, vogal, vogal };
\node[textonly, right=20pt of box9] (wi9) {contínua, vogal, líquida};
\node[textonly, right=20pt of box10] (wi10) {fricativa, posterior, \#};

%pré processamento
\node[textonly, left=230pt of box4] (verb1) {falar};
\node[textonly, below=10pt of verb1] (verb2) {fala};
\node[textonly, below=10pt of verb2] (verb3) {\#fala\#};

%trigramas
\node[textonly, right=20pt of verb1] (tri1) {\#,f,a};
\node[textonly, below=10pt of tri1] (tri2) {f,a,l};
\node[textonly, below=10pt of tri2] (tri3) {a,l,a};
\node[textonly, below=10pt of tri3] (tri4) {l,a,\#};

%features
\node[textonly, left=50pt of box2] (f1) {\#,cont,vogal};
\node[textonly, below=0pt of f1] (f2) {\#,cont,aberta};
\node[textonly, below=0pt of f2] (f3) {\#,cont,anterior};
\node[textonly, below=0pt of f3] (f4) {\#,cont,baixa};
\node[textonly, below=0pt of f4] (f5) {\reflectbox{$\vdots$}};
\end{tikzpicture}
}\caption{Esquema de Codificação de Wickelfeatures} 
\label{fig:wick}
\end{figure}

Na época, a dificuldade de se representar uma sequência de itens em um único vetor numérico era muito grande. O modelo apresentou problemas especialmente em casos em que uma mesma sequência de traços distintivos aparecia mais de uma vez no mesmo verbo, visto que o modelo marca apenas presença ou ausência da sequência. Hoje, novas arquiteturas de redes neurais (a ver no Cap \ref{ch:03} estão disponíveis para lidar com dados sequenciais com a utilização de represetações vetoriais menos complicadas. 

Uma possível escolha para tal representação é o tradicional Alfabeto Fonético Internacional (AFI), apresentado nas Tabelas \ref{tab:ipa1} e \ref{tab:ipa2}. O AFI é um sistema de notação fonética, criado pela Associação Fonética Internacoional para promover uma padronização na transcrição de dados de diferentes idiomas. Ele organiza símbolos que representam unidades sonoras presentes nas línguas humanas a partir de características de execução dessas unidades. A Tabela \ref{tab:ipa1} reune o conjunto do sons consoantes e exibe na dimensão das colunas o ponto de articulação dos sons. As linhas compõe os diferentes modos de articulação possíveis. Por fim, dentro de uma mesma célula pode ocorrer um som com ou sem a vibração das cordas vocais, é o caso do par \textbf{[p b]}. O símbolo da esquerda representa o som surdo e o símbolo da direita, sonoro. As vogais estão organizadas na Tabela \ref{tab:ipa2}. As colunas dessa tabela se referem ao local de reprodução dos sons e as linhas à abertura da boca durante a execução. Quando os símbolos aparecem em pares, aquele da direita representa uma vogal arredondada. %\cite{para conhecer} 
\begin{center}
\scalebox{0.9}{
    \begin{tabular}{|l|cc|cc|cc|cc|cc|cc|cc|cc|cc|cc|cc|}
%\begin{tabular}{|l|cc|}
        \hline & 
            \multicolumn{2}{|c|}{\footnotesize{Bilabial}} &					% Bilabial
            \multicolumn{2}{|c|}{\footnotesize{Lab. dent.}} & 			% Labiodental
            \multicolumn{2}{|c|}{\footnotesize{Dental}} & 					% Dental
            \multicolumn{2}{|c|}{\footnotesize{Alveolar}} & 				% Alveolar
            \multicolumn{2}{|c|}{\footnotesize{P-alveo.}} & 		% Post-alveolar
            \multicolumn{2}{|c|}{\footnotesize{Retroflex}} & 				% Retroflex
            \multicolumn{2}{|c|}{\footnotesize{Palatal}} & 					% Palatal
            \multicolumn{2}{|c|}{\footnotesize{Velar}} & 					% Velar
            \multicolumn{2}{|c|}{\footnotesize{Uvular}} & 					% Uvular
            \multicolumn{2}{|c|}{\footnotesize{Pharyng.}} & 			% Pharyngeal
            \multicolumn{2}{|c|}{\footnotesize{Glottal}}  \\% Glottal

        \hline 
        Plosive &	% Plosive
            p & b &													% Bilabial
            &&														% Labiodental
            \multicolumn{3}{|r}{t}&							% Dental
            \multicolumn{3}{l|}{d}&							% Alveolar
                                                                        % Post-alveolar
            \ipa{\:t} & \ipa{\:d}&									% Retroflex
            c & \textbardotlessj &														% Palatal
            k & g &													% Velar
            q & \ipa{\;G} &										% Uvular
            & \BlankCell        &								% Pharyngeal
            \ipa{P}& \BlankCell         \\								% Glottal

        \hline Nasal & 							% Nasal
            & m &													% Bilabial
            & \ipa{M} &											% Labiodental
            \multicolumn{3}{|r}{}&								% Dental
            \multicolumn{3}{l|}{n}&							% Alveolar
                                                                        % Post-alveolar
            & \ipa{\:n} &														% Retroflex
            & \textltailn &														% Palatal
            & \ipa{N} &														% Velar
            & N &														% Uvular
            \BlankCell        & \BlankCell        &		% Pharyngeal
            \BlankCell        & \BlankCell         \\		% Glottal

        \hline Trill &  								% Trill
            & \ipa{\;B}&											% Bilabial
            & &														% Labiodental
            \multicolumn{3}{|r}{}&								% Dental
            \multicolumn{3}{l|}{r}&								% Alveolar
                                                                        % Post-alveolar
            & &														% Retroflex
            & &														% Palatal
            \BlankCell        & \BlankCell        &		% Velar
            & \ipa{\;R}&											% Uvular
            & &														% Pharyngeal
            \BlankCell        & \BlankCell         \\		% Glottal

        \hline Tap/Flap &  						% Tap /Flap
            & &													% Bilabial
            & &														% Labiodental
            \multicolumn{3}{|r}{} &					% Dental
            \multicolumn{3}{l|}{\ipa{R}} &					% Alveolar
                                                                        % Post-alveolar
            & \ipa{\:r} &														% Retroflex
            & &														% Palatal
            \BlankCell        & \BlankCell        &		% Velar
            & &														% Uvular
            & &														% Pharyngeal
            \BlankCell        & \BlankCell         \\		% Glottal

        \hline Fricative & 						% Fricative
            \ipa{F} & \ipa{B} &									% Bilabial
            f & v &													% Labiodental
            \ipa{T} & \ipa{D} &									% Dental
            s & z &													% Alveolar
            \ipa{S} & \ipa{Z} &									% Post-alveolar
            \ipa{\:s} & \ipa{\:z} &								% Retroflex
            \ipa{\c{c}} & \ipa{J} &								% Palatal
            x & \ipa{G} &											% Velar
            \ipa{X} & \ipa{K} &									% Uvular
            \textcrh & \ipa{Q} &								% Pharyngeal
            h & \texthth \\										% Glottal

        \hline Lat. Fric. & 					% Lat. Fricative
            \BlankCell        & \BlankCell        &		% Bilabial
            \BlankCell        & \BlankCell        &		% Labiodental
            \multicolumn{3}{|r}{\textbeltl} &				% Dental
            \multicolumn{3}{l|}{\textlyoghlig} &			% Alveolar
                                                                        % Post-alveolar
            & &														% Retroflex
            & &														% Palatal
            & &														% Velar
            & &														% Uvular
            \BlankCell        & \BlankCell        			% Pharyngeal
            & \BlankCell        & \BlankCell         \\   % Glottal

        \hline Approx & 							% Approx.
            & &														% Bilabial
            & \ipa{V} &											% Labiodental
            \multicolumn{3}{|r}{}&								% Dental
            \multicolumn{3}{l|}{\ipa{\*r}} &					% Alveolar
                                                                        % Post-alveolar
            & \ipa{\:R} &											% Retroflex
            & j &														% Palatal
            & \textturnmrleg &									% Velar
            & &														% Uvular
            & &														% Pharyngeal
            \BlankCell        & \BlankCell         \\		% Glottal

        \hline Lat. appr. & 					% Lat. Approx
            \BlankCell        & \BlankCell        &		% Bilabial
            \BlankCell        & \BlankCell        &		% Labiodental
            \multicolumn{3}{|r}{}&								% Dental
            \multicolumn{3}{l|}{l}&								% Alveolar
                                                                        % Post-alveolar
            & \textipa{\:l} &											% Retroflex
            & \textipa{L} &												% Palatal
            & \textipa{\;L} &											% Velar
            & &														% Uvular
            \BlankCell        & \BlankCell        &		% Pharyngeal
            \BlankCell        & \BlankCell         \\		% Glottal
        \hline
    \end{tabular}
}%scalebox
\captionof{table}{Consoantes AFI}\label{tab:ipa1}
\end{center}

\begin{center}
    \begin{vowel}
        %    \putcvowel[l]{i}{1}
        \putvowel[l]{i}{0pt}{0pt}
        \putcvowel[r]{y}{1}
        \putcvowel[l]{e}{2}
        \putcvowel[r]{\o}{2}
        \putcvowel[l]{\textepsilon}{3}
        \putcvowel[r]{\oe}{3}
        \putcvowel[l]{a}{4}
        \putcvowel[r]{\textscoelig}{4}
        \putcvowel[l]{\textscripta}{5}
        \putcvowel[r]{\textturnscripta}{5}
        \putcvowel[l]{\textturnv}{6}
        \putcvowel[r]{\textopeno}{6}
        \putcvowel[l]{\textramshorns}{7}
        \putcvowel[r]{o}{7}
        \putcvowel[l]{\textturnm}{8}
        \putcvowel[r]{u}{8}
        \putcvowel[l]{\textbari}{9}
        \putcvowel[r]{\textbaru}{9}
        \putcvowel[l]{\textreve}{10}
        \putcvowel[r]{\textbaro}{10}
        \putcvowel{\textschwa}{11}
        \putcvowel[l]{\textrevepsilon}{12}
        \putcvowel[r]{\textcloserevepsilon}{12}
        \putcvowel{\textsci\ \textscy}{13}
        \putcvowel{\textupsilon}{14}
        \putcvowel{\textturna}{15}
        \putcvowel{\ae}{16}
    \end{vowel}
\captionof{table}{Vogais AFI}\label{tab:ipa2}   
\end{center} 

O AFI poderia ter sido utilizado para a construção das representações vetoriais, mas apenas um subconjunto do mesmo seria aproveitado já que o português não contém todos os fones possíveis. Desse modo, optou-se por uma simplificação do AFI que resultou na Tabela \ref{tab:new_rep}. Nessa nova representação, além da exclusão de alguns pontos e modos de articulação, apresenta-se também uma chave de transcrição alternativa que engloba apenas caracteres pertencentes ao código ASCII. A opção pela realização dessas modificações se deu em razão de facilitar o pré-processamento do corpus uma vez que o código fonético não é interpretável pela maioria das linguagens de programação.






\begin{center}
\scalebox{0.9}{
    \begin{tabular}{|l|cc|cc|cc|cc|cc|cc|}
        \hline & 
            \multicolumn{2}{|c|}{\footnotesize{Bilabial}} &					% Bilabial
            \multicolumn{2}{|c|}{\footnotesize{Lab. dent.}} & 			% Labiodental
            \multicolumn{2}{|c|}{\footnotesize{Alveolar}} & 				% Alveolar
            \multicolumn{2}{|c|}{\footnotesize{P-alveo.}} & 		% Post-alveolar

            \multicolumn{2}{|c|}{\footnotesize{Velar}} & 					% Velar
            \multicolumn{2}{|c|}{\footnotesize{Glottal}}  \\					% Glottal

        \hline Plosive &  							% Plosive
            p & b &	% Bilabial
            &&	% Labiodental
            t & d	% Alveolar
            & &% Post-alveolar
            & k & g 											& 		% Velar

            & \BlankCell         \\								% Glottal

        \hline Nasal & 							% Nasal
            & m 	% Bilabial
            &  &  & % Labiodental
            & n 	% Alveolar
            & & % Post-alveolar
            & & N 	% Velar
            &  &          \\		% Glottal

        \hline Tap/Flap &  						% Tap /Flap
            &													% Bilabial
            & &														% Labiodental
           && r &					% Alveolar
            &&                 % Post-alveolar

            \BlankCell        & \BlankCell        &		% Velar
            \BlankCell        & \BlankCell         \\		% Glottal

        \hline Fricative & 						% Fricative
            &  &									% Bilabial
            f & v &													% Labiodental
            s & z &													% Alveolar
            x & j &									% Post-alveolar
            &  &% Velar
            h &  \\										% Glottal

        \hline Lat. appr. & 					% Lat. Approx
            \BlankCell        & \BlankCell        &		% Bilabial
            \BlankCell        & \BlankCell        &		% Labiodental

           & l &							% Alveolar
                                                      &&                  % Post-alveolar
  
             & L & 												% Palatal
            											% Velar
            \BlankCell        & \BlankCell         \\		% Glottal
        \hline
    \end{tabular}
}%scalebox
\captionof{table}{Consoantes na nova representação}\label{tab:new_rep}
\end{center}

\begin{center}
\begin{table}[H]
\begin{center}
    \begin{tabular}{lll}
        \hline
         & Anterior & Posterior \\
         \hline
        Fechada & i & u \\
        \hline
        Meia-fechada & e & o \\
        \hline
        Meia-aberta & E & O \\
        \hline
        Aberta & a &  \\
        \hline
        Nasais & A (ã) &\\ & 3 ($\tilde{e}$) &\\ 
        \hline
    \end{tabular}
\end{center}
\caption{Vogais na nova representação}
\label{tab:new_vocals}
\end{table}
\end{center}

Além de tais modificações, também não foram incluídas representações para variantes, portanto os sons de \textbf{[x]-[h]-[r]} (variantes possíveis na pronúncia da palavra 'te\textbf{rr}a') foram resumidos apenas a \textbf{[h]}. Da mesma forma, também não há um símbolo equivalente para \textbf{[\ipa{S}]}, som empregado normalmente entre \textbf{t} e \textbf{i}, como na palavra "time". A representação para as vogais foi simplicada também para facilitar as transcrições e restringir o número de traços fonéticos possíveis no modelo. A Tabela \ref{tab:chave} mostra a chave de transcrição proposta para as consoantes e vogais. A Tabela \ref{tab:transc} exibe alguns exemplos de transcrição na nova representação sugerida.


\begin{table}[H]
\begin{center}
\begin{tabular}{lc}
\textbf{AFI} & \multicolumn{1}{l}{\textbf{Transcrição Proposta}} \\ \hline

\textbf{{[}p{]}} - \textbf{p}arar & p \\
\textbf{{[}b{]}} - \textbf{b}otar & b \\
\textbf{{[}t{]}} - \textbf{t}ocar & t \\
\textbf{{[}d{]}} - \textbf{d}ançar & d \\
\textbf{{[}k{]}} - \textbf{c}asar & k \\
\textbf{{[}g{]}} - \textbf{g}ostar & g \\
\textbf{{[}f{]}} - \textbf{f}ugir & f \\
\textbf{{[}v{]}} - \textbf{v}oltar & v \\
\textbf{{[}s{]}} - \textbf{s}oltar & s \\
\textbf{{[}z{]}} - pre\textbf{s}enciar & z \\
\textbf{{[}\ipa{S}{]}} - \textbf{ch}amar & x \\
\textbf{{[}\ipa{Z}{]}} - \textbf{j}antar & j \\
\textbf{{[}t\ipa{S}{]}} - sen\textbf{t}ir & t \\
\textbf{{[}d\ipa{Z}{]}} - \textbf{d}izer & d \\
\textbf{{[}h{]}} - e\textbf{rr}ar & h \\
\textbf{{[}\ipa{\:r}{]}} - enca\textbf{r}ar & r \\
\textbf{{[}l{]}} - pu\textbf{l}ar & l \\
\textbf{{[}\textipa{L}{]}} - espa\textbf{lh}ar & L \\
\textbf{{[}m{]}} - \textbf{m}orar & m \\
\textbf{{[}n{]} }- \textbf{n}adar & n \\
\textbf{{[}\ipa{N}{]}} - so\textbf{nh}ar & N\\
\textbf{{[}a{]}} - p\textbf{a}rar & a \\
\textbf{{[}e{]}} - l\textbf{e}r & e \\
\textbf{{[}{]}} - esp\textbf{e}ro & E \\
\textbf{{[}i{]}} - r\textbf{i}r & i \\
\textbf{{[}o{]}} - pr\textbf{o}por & o \\
\textbf{{[}{]} }- col\textbf{o}co & O \\
\textbf{{[}u{]}} - c\textbf{u}rtir & u \\
\textbf{{[}{]}} - \textbf{e}ntreter & 3 \\
\textbf{{[}ã{]}} - pl\textbf{a}ntar & A \\
\textbf{{[}{]}} - comp\textbf{o}nho & o \\
\textbf{{[}{]}} - cas\textbf{o} & u \\
\textbf{{[}{]}} - sa\textbf{i}o & i \\
\textbf{{[}{]}} - vo\textbf{l}to & u
\end{tabular}
\end{center}
\caption{Chave de Transcrição Proposta}
\label{tab:chave}
\end{table}




dar exemplos de transcrição
% 




