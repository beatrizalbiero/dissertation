\chapter{Encoder-Decoder}
\label{ch:05}

Neste capítulo, será introduzida a arquitetura do modelo \textit{Encoder-Decoder}. Além disso, falaremos sobre a aplicação da arquitetura aplicada ao problema em questão, ou seja, a proposição de um modelo para a predição de formas verbais flexionadas (no paradigma já estabelecido). Por último, será apresentado o algoritmo de \textit{Pós-Processamento} adotado nesta pesquisa.

\section{Introdução ao modelo Enconder-Decoder}
\label{sec:intro-sec-sec}
Um modelo de mapeamento \textit{Encoder-Decoder} (\cite{enc-dec:2014}) é um sistema composto por duas RNR's cuja principal função é mapear uma sequência à outra. Modelos do tipo \textit{Encoder-Decoder} (também conhecidos como \textit{Seq2Seq} (\cite{seq2seq:2014}) têm sido bastante utilizados em tarefas linguísticas, especialmente no desenvolvimento de sistemas de diálogo e em tradução automática.

Em um contexto de tradução, por exemplo, o modelo recebe como \textit{input }uma sequência de uma língua de origem e produz como \textit{output} uma sequência em uma língua alvo. A sequência gerada precisa, além de preservar o conteúdo semântico da sequência de origem, apresentar uma sintaxe aceita pelos falantes da língua alvo. 

\begin{figure}[ht!]
\centering
\begin{tikzpicture}
\node[punkt] (seq1) {Sequência 1};
\node[punkt, right=40pt of seq1] (seq2) {Sequência 2};

\path[tedge] (seq1) -- (seq2);

\node[punkt, below=40pt of seq2] (seq21) {O menino pequeno gosta do seu cachorro.};

\node[punkt, left=40 pt of seq21] (seq12) {The small boy likes his dog.};

\path[tedge] (seq12) -- (seq21);

\node[text, below=10 pt of seq21] (nada) {};

\end{tikzpicture}
\caption{Objetivo do Modelo Encoder-Decoder: Mapeamento de uma Sequência à Outra} 
\label{fig:seq2seq_simple}
\end{figure}
Observa-se no caso da tradução, que não há uma correspondência exata entre os termos de cada uma das línguas. A sentença na língua inglesa possui um termo a menos, sendo que o pronome “his” corresponde aos termos “do seu” no português. Além disso, observa-se que cada uma das línguas possui uma sintaxe diferente. No primeiro caso, o adjetivo \textit{small} se posiciona antes do substantivo \textit{boy}. No segundo, essa ordem é invertida (\textit{menino pequeno}). 

As duas redes recorrentes funcionam da seguinte maneira (ver Fig. \ref{fig:seq2seq}): uma primeira rede, a denominada \textit{Encoder}, é alimentada com uma sentença da língua de origem (em inglês, por exemplo). Os termos dessa sentença entram um após o outro na rede e alimentam os \textit{estados} ($\vect{h}^{(t)}$), porém nesse caso não é necessário relacionar cada termo a um correspondente ($y_t$) como no caso do modelo de linguagem. A rede \textit{Encoder}, portanto, não possui \textit{alvos}, ela serve apenas para acumular os dados da língua de origem. Assim, o resultado do último estado, gerado após a inserção de todos os termos da sequência de origem, servirá como estado inicial ($\vect{h}^{(0)}$) para uma segunda rede recorrente, denominada de \textit{Decoder}. A rede \textit{Decoder}, por sua vez, recebe como seu primeiro \textit{input}, um marcador que representa o início da sequência alvo (\textbf{<beg>}), inicializando o modelo de linguagem dessa língua e com ordem de parada ao predizer o marcador de final de palavra (\textbf{<eos>}), como no modelo de rede neural recorrente apresentado na Seção \ref{sec:RNN}. 

 \begin{figure}[ht!]
\centering

\scalebox{0.9}{
\begin{tikzpicture}[
  hid/.style 2 args={
    rectangle split,
    rectangle split horizontal,
    draw=#2,
    rectangle split parts=#1,
    fill=#2!20,
    outer sep=1mm}]
  % draw input nodes
  \foreach \i [count=\step from 1] in {the,small,boy,{{$<$beg$>$}}}
    \node (i\step) at (2*\step, -2) {\i};

%write states    
\node[text, left = 2pt of i#1] (ht) {$\vect{h}^{(t)}$};

  % draw output nodes
  \foreach \t [count=\step from 4] in {o,menino,pequeno,{{$<$eos$>$}}} {
    \node[align=center] (o\step) at (2*\step, +2.75) {\t};
  }
  % draw embedding and hidden layers for text input
  \foreach \step in {1,...,3} {
    \node[hid={3}{red}] (h\step) at (2*\step, 0) {};
    \node[hid={3}{red}] (e\step) at (2*\step, -1) {};    
    \draw[->] (i\step.north) -> (e\step.south);
    \draw[->] (e\step.north) -> (h\step.south);
  }
  
  %write inputs and outpus
\node[text, below = 15pt of ht] {$x_{t}$};
\node[text, above = 20pt of ht] {$y_{t}$};
  
  % draw embedding and hidden layers for label input
  \foreach \step in {4,...,7} {
    \node[hid={3}{yellow}] (s\step) at (2*\step, 1.25) {};
    \node[hid={3}{blue}] (h\step) at (2*\step, 0) {};
    \node[hid={3}{blue}] (e\step) at (2*\step, -1) {};    
    \draw[->] (e\step.north) -> (h\step.south);
    \draw[->] (h\step.north) -> (s\step.south);
    \draw[->] (s\step.north) -> (o\step.south);
  }  
  % edge case: draw edge for special input token
  \draw[->] (i4.north) -> (e4.south);
  % draw recurrent links
  \foreach \step in {1,...,6} {
    \pgfmathtruncatemacro{\next}{add(\step,1)}
    \draw[->] (h\step.east) -> (h\next.west);
  }
  % draw predicted-labels-as-inputs links
  \foreach \step in {4,...,6} {
    \pgfmathtruncatemacro{\next}{add(\step,1)}
    \path (o\step.north) edge[->,out=45,in=225] (e\next.south);
  }
\end{tikzpicture}
}\caption{Esquema de Encoder-Decoder para Tradução Automática} 
\label{fig:seq2seq}
\end{figure}
 
 Por fim, o aprendizado do modelo ocorre como já mencionado no Capítulo \ref{ch:03}: os \textit{outputs} do modelo são comparados com os \textit{alvos}, o erro é propagado de volta através do algoritmo de \textit{backpropagation}, os pesos da rede são atualizados de modo a diminuir esse erro e o processo se repete até a conclusão de todas as \textit{épocas}. É importante ressaltar também que, embora a figura ilustre como exemplo duas sequências de três palavras, o modelo comporta sequências de tamanhos quaisquer e que não precisam coincidir entre si.
 
\section{Encoder-Decoder e flexão verbal}

Assim como no problema da tradução automática, pode-se retratar a questão do aprendizado de flexão dos verbos através de uma relação entre duas sequências. No caso desta pesquisa, pensaremos nessa relação como partindo de uma forma base para a forma flexionada escolhida (como mencionado na Seç. \ref{sec:escopo}).

\begin{figure}[ht!]
\centering
\begin{tikzpicture}
\node[punkt] (seq1) {Verbo no Infinitivo};
\node[punkt, right=40pt of seq1] (seq2) {Verbo Flexionado};

\path[tedge] (seq1) -- (seq2);

\node[punkt, below=40pt of seq2] (seq21) {estudo};

\node[punkt, left=40 pt of seq21] (seq12) {estudar};

\path[tedge] (seq12) -- (seq21);

\node[text, below=10 pt of seq21] (nada) {};
\end{tikzpicture}
\caption{Relação entre um Verbo no Infinitivo e o Mesmo Flexionado} 
\label{fig:verbs}
\end{figure}

Diferentemente do caso de tradução, em que se busca o aprendizado das relações entre uma sentença e outra das línguas de origem e de alvo, no caso do aprendizado de flexão de verbos busca-se aprender as relações existentes entre uma forma verbal básica e uma flexionada a nível fonético. Desse modo, resta explicar como se deu a transformação dos verbos em vetores adequados para que o modelo fosse capaz de aprender tais relações.


\subsection{A rede Encoder desenvolvida}

Relembrando a Seç \ref{sec:inputs}, temos que cada fone é representado através de uma forma vetorial com 20 dimensões (correspondentes ao número de traços fonéticos do estudo). Desse modo, a camada de \textit{inputs} da rede \textit{Encoder} tem dimensão 20, e é responsável por receber os verbos na forma base (RAD + VT). Em seguida, o \textit{input} é direcionado para uma rede de tipo LSTM \textit{unidirecional}. Unidirecional refere-se a um mecanismo que regula o foco de atenção da rede. Nessa configuração, os \textit{inputs} são inseridos sequencialmente seguindo o fluxo padrão (do primeiro ao último termo da sequência). Existe também o tipo \textit{bidirecional}, em que os \textit{inputs} são inseridos de forma dupla, seguindo o fluxo padrão e também o inverso (\cite{schuster1997bidirectional}).


\subsection{A rede Decoder desenvolvida}

A rede \textit{Decoder}, por sua vez, funciona como um modelo de linguagem. O estado inicial desse modelo ($\vect{h}^{(0)}$) é o estado final gerado pela rede \textit{Encoder}. Além disso, também entram como \textit{input} na rede os respectivos verbos flexionados. A rede recorrente utilizada também é de tipo LSTM unidirecional.

A Figura \ref{fig:seq2seq} apresenta um esquema do modelo desenvolvido utilizando como exemplo o verbo irregular “\textit{ler}” (já com a exclusão da desinência de infinitivo).

\begin{figure}[ht!]
\centering

\scalebox{1.0}{
\begin{tikzpicture}[
  hid/.style 2 args={
    rectangle split,
    rectangle split horizontal,
    draw=#2,
    rectangle split parts=#1,
    fill=#2!20,
    outer sep=1mm}]
  % draw input nodes
  \foreach \i [count=\step from 1] in {l,e,r,{{$<$eos$>$}}}
    \node (i\step) at (2*\step, -2) {\i};
  % draw output nodes
  \foreach \t [count=\step from 4] in {l,e,i,{{o}}} {
    \node[align=center] (o\step) at (2*\step, +2.75) {\t};
  }
  % draw embedding and hidden layers for text input
  \foreach \step in {1,...,3} {
    \node[hid={3}{red}] (h\step) at (2*\step, 0) {};
    \node[hid={3}{red}] (e\step) at (2*\step, -1) {};    
    \draw[->] (i\step.north) -> (e\step.south);
    \draw[->] (e\step.north) -> (h\step.south);
  }
  % draw embedding and hidden layers for label input
  \foreach \step in {4,...,7} {
    \node[hid={3}{yellow}] (s\step) at (2*\step, 1.25) {};
    \node[hid={3}{blue}] (h\step) at (2*\step, 0) {};
    \node[hid={3}{blue}] (e\step) at (2*\step, -1) {};    
    \draw[->] (e\step.north) -> (h\step.south);
    \draw[->] (h\step.north) -> (s\step.south);
    \draw[->] (s\step.north) -> (o\step.south);
  }  
  % edge case: draw edge for special input token
  \draw[->] (i4.north) -> (e4.south);
  % draw recurrent links
  \foreach \step in {1,...,6} {
    \pgfmathtruncatemacro{\next}{add(\step,1)}
    \draw[->] (h\step.east) -> (h\next.west);
  }
  % draw predicted-labels-as-inputs links
  \foreach \step in {4,...,6} {
    \pgfmathtruncatemacro{\next}{add(\step,1)}
    \path (o\step.north) edge[->,out=45,in=225] (e\next.south);
  }
\end{tikzpicture}
}\caption{Esquema de Encoder-Decoder Utilizado} 
\label{fig:seq2seq}
\end{figure}


 \subsection{Pós-processamento}

Assim como é necessária uma etapa de pré-processamento para transformar os verbos em vetores numéricos para a alimentação do modelo, também é necessário traduzir a saída do modelo (que neste momento corresponde a uma sequência de vetores numéricos) para uma sequência de caracteres e para a reconstrução do verbo flexionado predito.

Primeiramente, temos que a predição do verbo flexionado é o resultado da concatenação das predições do modelo de linguagem aprendido durante o treinamento. Dessa forma, para realizar uma predição, o algoritmo libera um vetor de cada vez. O algoritmo finaliza a predição de fones de um verbo assim que prevê um marcador de final. Tais predições não constituem mais vetores binários, ao invés disso, o modelo solta vetores com valores variando de 0 a 1 (conforme a função de ativação utilizada). Por essa razão, para o pós-processamento dos vetores de saída são feitas duas aproximações: (i) valores abaixo de 0.5 foram substituídos por 0 e valores acima são substituídos por 1; (ii) o resultado dessa operação pode não resultar em um fone válido, com por exemplo um vetor com presença de dois traços contraditórios (ex: fricativo e oclusivo) e portanto é substituído (se necessário) pela representação vetorial cuja distância seja mínima quando comparada a todos os fones. Aqui, portanto, vemos que a restrição quanto à validade dos fones envolve um conhecimento externo ao modelo proposto, por parte da linguística. De todo modo, esse critério teve de ser adotado para viabilizar o algoritmo de pós-processamento. 

Na álgebra linear existem várias maneiras para se medir a distância entre dois vetores. Nesta pesquisa, optou-se arbitrariamente pela \textit{distância euclidiana} (ver \cite{boulos2009geometria} para maiores detalhes). Entretanto, é importante atentar para o fato de que há um custo computacional embutido nessa operação: calculam-se as distâncias entre cada vetor predito pelo modelo e as 30 representações vetoriais possíveis. 

A distância euclidiana entre os pontos \textbf{p} e \textbf{q} é equivalente ao comprimento do segmento de reta que os conecta (${\displaystyle {\overline {\mathbf {p} \mathbf {q} }}}$). Utilizando coordenadas cartesianas, e sejam \textbf{q} = ($q_1, q_2, ..., q_n$) e \textbf{p} = ($p_1, p_2, ..., p_n$) dois pontos em um espaço n-dimensional, temos:

\begin{equation}
    d(\textbf{q}, \textbf{p)} = \sqrt{(q_1 - p_1)^2 + (q_2 - p_2)^2 + ... + (q_n - p_n)^2}\notag
\end{equation}
\begin{equation}
    = \sqrt{\sum_{i=1}^n (q_i-p_i)^2}.    
\end{equation}

Desse modo, o esquema apresentado na Fig. \ref{fig:pos-process} apresenta um resumo do processo de pós-processamento utilizado neste trabalho.
\begin{figure}[H]
\centering
\begin{tikzpicture}
\node[punkt] (pred) {Predição\\$\hat{y}$};
\node[punkt, right=40pt of pred] (arr) {Aproximação\\$\hat{y}_{ap}$};
\node[punkt, right=40pt of arr] (dist) {Distância Euclidiana com Fones Candidatos};
\node[punkt, right=40pt of dist] (subs) {Substituição pelo Vetor Candidato mais Próximo};

\path[tedge] (pred) -- (arr);
\path[tedge] (arr) -- (dist);
\path[tedge] (dist) -- (subs);


\node[punkt, below=40 pt of pred] (pred1) {[0.8734, 0.235, 0.003, ...]};
\node[punkt, below=40pt of arr] (arr1) {[1.0, 0.0, 0.0, ...]};
\node[punkt, right=160pt of arr1] (subs1) {[1.0, 0.0, 0.0, ...]};


\path[tedge] (pred1) -- (arr1);
\path[tedge] (arr1) -- (subs1);

\node[text, below=10 pt of arr1] (nada) {};



\end{tikzpicture}
\caption{Decodificação dos Outputs do Modelo} 
\label{fig:outputs}
\end{figure}

 





