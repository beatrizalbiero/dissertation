\chapter{Encoder-Decoder}
\label{ch:05}

Agora que já temos o conhecimento de RNR e modelo de linguagem...



\section{Definição}
Um modelo de mapeamento Encoder-Decoder é um sistema composto por duas RNRs cuja principal função é mapear sequências. A primeira rede, chamada \textit{Encoder}, mapeia os dados de uma determinada sequência de origem e devolve uma representação condensada do treinamento (um estado), este alimenta então a segunda rede, \textit{Decoder}, que, por sua vez, tem como objetivo gerar uma nova sequência alvo.  %ref
Modelos do tipo Encoder-Decoder (também conhecidos como \textit{Seq2Seq}) tem sido bastante utilizados em tarefas linguísticas, especialmente no desenvolvimento de sistemas de diálogo e em tradução automática. Em um contexto de tradução do inglês para o português, por exemplo, uma rede encoder seria treinada com um corpus em inglês, o qual apresenta determinadas probabilidades de arranjo sintático e valores semânticos, ou seja, uma gestalt sequencial armazenada em uma matriz. A rede decoder, ao ser treinada na língua alvo, isto é, português, implicitamente leva em consideração tanto as características sintático-semânticas desta quanto a gestalt da rede encoder, gerando desta forma uma nova sequência que contempla ambas as línguas e assemelha-se a uma tradução humana. Tecnicamente, a gestalt gerada pela rede encoder nada mais é do que o último estado consolidado alcançado a partir dos treinamentos dessa rede. Por fim, obtem-se um sistema capaz de predizer os resultados de uma sequência alvo a partir de uma sequência de origem.

Em função de uma similaridade conceitual, este modelo foi escolhido, uma vez que possibilita uma interação mediada pela essência de duas qualidades de padrões sequencias e tem como objetivo a produção de um resultado orgânico, semelhante, por exemplo, a intuição de um falante nativo que produz a flexão de um verbo na primeira pessoa do presente a partir de um verbo no infinitivo. 

incluir referencias

\section{Modelo Desenvolvido}

\subsection{Inputs e Outputs}
\begin{itemize}
    \item dar exemplos de codificação
    \item são duas LSTMs, então falar sobre o que entra e sai de cada uma delas.
    \item exemplo de input e output
    \item acrescentar um desenho com o modelo verdadeiro utilizado
\end{itemize}

\input{tikz/seq2seq.tex} %reduzir esse desenho

