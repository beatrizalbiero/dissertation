\chapter{O Experimento}
\label{ch:07}

Neste capítulo falaremos sobre o modelo \textit{Encoder-Decoder} aplicado, bem como os hiperparâmetros utilizados. Além disso, será exibida a metodologia adotada para avaliação dos resultados e discussões sobre os resultados obtidos.

\section{Hiperparâmetros}
\label{sec:treinamento}

Os hiperparâmetros selecionados para a realização do experimento foram principalmente baseados em uma tarefa similar de \cite{cholletseq2seq}, em que textos são traduzidos de uma língua para outra utilizando o nível de caracter como referência.

Após alguns experimentos, o número de \textit{épocas} do modelo foi definido como 300. A escolha se deu após a verificação de que após esse valor o modelo tinha seu aprendizado estacionado.

Sem entrar em detalhes teóricos, também vale comentar os demais hiperparâmetros utilizados. As redes \textit{Encoder} e \textit{Decoder} são ambas arquiteturas do tipo LSTM com dimensão latente de 256 unidades cada (ou seja, apresentam uma camada intermediária com 256 nós). Para a otimização do modelo, foi utilizado o algoritmo \textit{Adam} (\cite{adam:2014}), disponível na API do \textit{Keras} (\cite{chollet2015keras}) com hiperparâmetros pré-definidos (taxa de aprendizado = 0.001, $\beta_{1}$=0.9, $\beta_{2}$=0.999, $\epsilon$ = Vazio, Decaimento = 0.0, amsgrad = Falso). A função de custo utilizada foi a de Entropia Cruzada Binária (ref), já que temos que classificar uma série de valores como sendo \textbf{0}'s ou \textbf{1}'s. Além disso, o treinamento foi realizado em lotes (\textit{batches} de 128 verbos com uma taxa de 20\% para validação cruzada. Por último, ainda baseado na tarefa de \cite{cholletseq2seq}, uma camada intermediária foi acrescentada entre o \textit{Decoder} e a camada de \textit{output}.

A Fig. \ref{fig:encoder-decoder} exibe a arquitetura completa do modelo.

\begin{figure}[!htb]
\center{\includegraphics[width=0.4\textwidth]
{img/encoder-decoder.png}}
\caption{\label{fig:encoder-decoder} Arquitetura Final Utilizada.}
\end{figure}

\section{Metodologia para Avaliação}
\label{sec:method}

 Para a avaliação dos resultados do modelo, foi utilizada uma técnica de validação cruzada chamada \textit{K-Fold} (\cite{kfold:2018}). A análise K-Fold permite que todos os verbos do corpus sejam testados. A técnica consiste, primeiramente, na formação de K subconjuntos de verbos mutuamente exclusivos de tamanhos próximos (idênticos caso a divisão por K seja exata). Em seguida, um desses subconjuntos é escolhido como o conjunto de teste enquanto que os K-1 restantes são utilizados como treino. O tamanho de K é definido a partir da escolha de proporção em que se deseja realizar a segmentação entre treino e teste, ou seja, para sempre manter a proporção de testes em 20\% de modo que estes verbos sejam sempre diferentes, o corpus precisa ser segmentado em 5 subconjuntos distintos. O desenho \ref{fig:kfold} ilustra o algoritmo. 

\input{definitions/colors}
\input{definitions/styles}
\begin{figure}[ht!]
\centering

\scalebox{1.0}{
\begin{tikzpicture}[H]

%first
\node[box3] (box1) {};
\node[box2, right=0pt of box1] (box2) {};
\node[box2, right=0pt of box2] (box3) {};
\node[box2, right=0pt of box3] (box4) {};
\node[box2, right=0pt of box4] (box5);

%second
\node[box2, below=15pt of box1] (box6) {};
\node[box3, right=0pt of box6] (box7) {};
\node[box2, right=0pt of box7] (box8) {};
\node[box2, right=0pt of box8] (box9) {};
\node[box2, right=0pt of box9] (box10);

%third
\node[box2, below=15pt of box6] (box11) {};
\node[box2, right=0pt of box11] (box12) {};
\node[box3, right=0pt of box12] (box13) {};
\node[box2, right=0pt of box13] (box14) {};
\node[box2, right=0pt of box14] (box15);

%fourth
\node[box2, below=15pt of box11] (box16) {};
\node[box2, right=0pt of box16] (box17) {};
\node[box2, right=0pt of box17] (box18) {};
\node[box3, right=0pt of box18] (box19) {};
\node[box2, right=0pt of box19] (box20);

%fifth
\node[box2, below=15pt of box16] (box21) {};
\node[box2, right=0pt of box21] (box22) {};
\node[box2, right=0pt of box22] (box23) {};
\node[box2, right=0pt of box23] (box24) {};
\node[box3, right=0pt of box24] (box25);

%legenda
\node[box2, right=35pt of box10] (box26) {};
\node[textonly, right=5pt of box26] (box27) {Treino};
\node[box3, below=10pt of box26] (box28) {};
\node[textonly, right=5pt of box28] (box29) {Teste};



\end{tikzpicture}
}\caption{Técnica de segmentação do Corpus via K-Fold} 
\label{fig:kfold}
\end{figure}

Além disso, há ainda a vantagem do algoritmo de estratificação, ou seja, a certeza de que, em cada um dos treinamentos, as proporções das classes de verbos se mantenham no treinamento. Isso significa que, por exemplo, a classe de irregularidades do verbo “\textit{testar}” (testar $\rightarrow$ testo) que possui 20 verbos, terá sempre 16 verbos verbos no treinamento e 4 no teste. Assim, após os 5 treinamentos diferentes, todos os verbos da classe foram testados e está garantido de que havia verbos dessa classe no treinamento.

\subsection{Métrica de Avaliação}

Para a avaliação dos resultados, considerou-se a métrica de \textit{Acurácia}. A acurácia pode ser obtida através da fórmula:

\begin{equation}
    Acurácia = \frac{contagem(acertos)}{contagem(total)}
\end{equation}

Ainda, define-se como um acerto um verbo predito exatamente igual ao esperado.

\section{Resultados}
\label{sec:resultado}
Como há um fator de aleatoriedade nas inicializações dos pesos, e também como os verbos de treino e teste são sorteados pelo algoritmo no momento da segmentação do K-Fold, os resultados da rede podem variar. Para a verificação dos resultados levando em consideração as variações, foram realizadas 30 análises K-Fold. Isso significa que cada verbo foi testado 30 vezes. A Fig. \ref{fig:acc} mostra um gráfico do tipo \textit{boxplot} (\cite{2004:bussab}) que permite verificarmos a variação da acurácia. Na figura, podemos notar que a taxa de acerto do modelo desenvolvido se concentra entre 13-14\%, mas chega até 17\%.

\begin{figure}[H]
  \centering
  \includegraphics[width=0.6\linewidth]{img/mean_accuracy.png}
  \caption{Acurácia Para Todos os Verbos}
  \label{fig:acc}
\end{figure}

Ainda, podemos avaliar os resultados de acordo com a segmentação entre regulares e irregulares e verificar se o modelo obteve vantagem em alguma das classes. 

\begin{figure}[]
\begin{floatrow}
\ffigbox{%
  \includegraphics[width=0.8\linewidth]{img/boxplot_irregular_vs_regular.png}%
}{%
  \caption{Boxplots Para Acurácias por Classe}%
  \label{fig:boxplotsclasses}
}
\capbtabbox{%
\begin{tabular}{lll}
 & \textbf{Regulares} & \textbf{Irregulares} \\ \hline
\textbf{Acurácia Média} & 17.88 \% & 9.23 \% \\
\textbf{Acurácia DP} & 2.30 \% & 1.15 \% \\
\textbf{Acurácia Min}& 12.65 \% & 6.70 \%\\
\textbf{Acurácia Max}& 23.83 \% &  11.00 \%\\ \hline\\
& & \\ 
& & \\
& & \\
& & \\
\end{tabular}
}{%
  \caption{Acurácias Médias por Classe com Desvio Padrão}%
  \label{tab:acuraciamedia}
}
\end{floatrow}
\end{figure}

Com a Fig. \ref{fig:boxplotsclasses} e a Tab. \ref{tab:acuraciamedia} podemos observar que o modelo obteve resultados melhores para a classe dos verbos \textit{Regulares} em relação à classe de \textit{Irregulares}. 


\subsection{Acurácia em Cada Classe}
\label{sec:prop}

Também é interessante avaliar o desempenho do modelo em cada classe irregular. Na Fig. \ref{fig:kfoldprop} observamos as acurácias de cada classe ordenadas pela proporção das mesmas no corpus. Nota-se que apesar da classe regular apresentar a maior proporção no corpus, as classes com maior acurácia são as classes dos verbos “botar” e “seguir”. Para avaliar a correlação entre a proporção no corpus e a acurácia foi utilizado o coeficiente de correlação de Pearson (\cite{2004:bussab}). Esse coeficiente assume valores entre -1 e 1, sendo que valores próximos às extremidades indicam maior correlação e valores próximos a zero indicam que as variáveis independem linearmente uma da outra. Para a interpretação do valor obtido utilizaremos a seguinte escala (\cite{pearson:1989}):

\begin{itemize}
    \item 0.9 para mais ou para menos indica uma correlação muito forte.
    \item0.7 a 0.9 positivo ou negativo indica uma correlação forte.
    \item0.5 a 0.7 positivo ou negativo indica uma correlação moderada.
    \item0.3 a 0.5 positivo ou negativo indica uma correlação fraca.
    \item0 a 0.3 positivo ou negativo indica uma correlação desprezível.
\end{itemize}

O coeficiente de correlação observado entre essas variáveis foi de \textbf{0.42}, o que indica uma correlação \textit{fraca}. 

\begin{figure}[H]
  \centering
  \includegraphics[width=0.8\linewidth]{img/proporxacc.png}
  \caption{Acurácia Para Cada Classe}
  \label{fig:kfoldprop}
\end{figure}

Para a análise dos resultados obtidos, utilizaremos o experimento K-Fold de acurácia total mais alta como referência (17\%, vide apêndice \ref{ch09-appendice}). Como se pode observar, diferentes tipos de erros podem acontecer, sendo que alguns são piores do que outros. Vejamos, por exemplo, o verbo “parecer” (com transcrição fonética e RAD + VT = “parese”). Nesse caso vemos que o modelo produziu a forma “peresu”. Nota-se que essa transformação em que o modelo errou alguns traços de um único fone, é diferente do erro que ocorre com o verbo “retrair” (com transcrição fonética e RAD + VT = “hetrai”), o qual o modelo produziu a forma “setEruu”. Vemos que no segundo caso, a forma produzida é praticamente incompreensível. Desse modo, ao avaliarmos as diferentes classes de verbos, falaremos em \textit{erros incompreensíveis} para descrever os verbos cujas formas preditas apresentam erros de difícil interpretabilidade, ou seja, mais próximos de terem sido gerados aleatoriamente do que através de um modelo. 

Ao analisar os erros ocorridos na classe do verbo “Pôr”, verifica-se que praticamente um terço (8/26) dos erros cometidos foram erros de \textit{regularização}, ou seja, o modelo se confundiu com a classe mais presente (a classe dos verbos regulares) e manteve o padrão de flexão regular. 

Em seguida, analisamos os verbos sem agrupamento (na figura, “none”). A acurácia baixa nesta classe era esperada pois não havia outros verbos no corpus para que o modelo conseguisse detectar possíveis padrões para fazer predições corretas. 

A classe do verbo “testar” apresentou muitas predições incoerentes, porém também houve erros de regularização (4/19). Chama a atenção que, para o verbo “pegar”, o modelo acertou a flexão irregular transformando a vogal anterior meio-fechada em meio-aberta (e $\rightarrow$ \textepsilon), porém não acertou o traço de vozeamento para caracterizar o fone “g” ao invés de “k”, o que resultou na flexão equivocada de pegar $\rightarrow$ pEku.

A classe do verbo “vir”, por sua vez, apresentou 5/11 de erros por regularização, os demais parecem erros incompreensíveis.

A classe do verbo “ansiar” apresentou 7 erros incompreensíveis e errou a flexão do verbo “passear“ (com transcrição fonética e RAD + VT = “pasea“), ao substituir a consoante “s” por “r“, ambas alveolares. 

A classe do verbo “dizer” apresentou apenas erros incompreensíveis e a classe do verbo “pedir” acertou a flexão irregular no verbo “pedir” porém não transformou o fone “d” em “s” (alveolar - surda - fricativa). Ao invés disso, apresentou “t” (alveolar - surda - oclusiva).

\subsection{Acurácia e Comprimento Médio das Classes}

Como o coeficiente de \textit{Pearson} observado entre a proporção das classes no corpus e a acurácia indicou uma correlação fraca, podemos buscar por outras variáveis que também possam influenciar no desempenho do modelo. Uma delas é o comprimento dos verbos. O verbo “3ntret3Nu” (entretenho), por exemplo, contém 9 fones para serem preditos. Como cada fone contém 20 traços, no total o modelo tem que conseguir prever 180 números. O verbo “falu” (falo), em contrapartida, possui 4 fones, ou seja, 80 números a serem preditos. Dessa forma, é natural pensar que quanto maior o número de predições necessárias, maior a probabilidade do modelo cometer algum erro. Com isso, a Figura \ref{fig:kfoldprop} exibe as acurácias das classes ordenadas de acordo com os comprimentos médios das mesmas. Na figura vemos que os grupos apresentam comprimentos médios que variam de 5 a 8 fones, mas com barras de confiança com limites próximos, indicando uma certa homogeneidade nesse aspecto.

No caso desse estudo, o coeficiente de \textit{Pearson} obtido foi de \textbf{0.39}, o que indica uma correlação também fraca entre estas variáveis. Entretanto, é interessante notar que a classe em que o modelo apresentou melhor desempenho (a classe de “botar”) é simultaneamente uma classe com alta proporção no corpus e também apresenta o comprimento médio mais baixo do grupo.

\begin{figure}[H]
  \centering
  \includegraphics[width=0.8\linewidth]{img/comp_acc.png}
  \caption{Acurácia por Comprimento Médio}
  \label{fig:kfoldprop}
\end{figure}


% colocar no apendice os resultados
\section{Outros Erros Relevantes}
\label{sec:interesting}

Alguns erros interessantes como troca de classes e regularizações já foram apresentados na Seção \ref{sec:prop}, mas ainda há outros erros que merecem destaque. (Ver Apêndice para tabela completa com Classe, Input, Output e Alvo)

Na classe de verbos sem agrupamento, um erro notável foi a flexão realizada para o verbo trazer (“trasu”), o que mostra que o modelo identificou o padrão de flexão da classe do verbo “fazer”. 

Na classe do verbo “pedir”, apresentou a flexão “espidu” para o verbo “espedir”, o que denota uma possível confusão com a classe do verbo “conseguir”.

Nos verbos regulares nota-se a presença de vários erros devido a troca de apenas um traço fonético. Como por exemplo para o verbo “convidar”, o output resultante foi “konviru”. Para o verbo “convencer”, “konfensu”.

Para os verbos da classe do verbo “seguir“, três verbos foram flexionados de acordo com a família do verbo “testar”. São eles: “ferir” (fEru), “vestir” (vEstu) e “repetir” (hepEtu).

A classe do verbo “ver” também apresentou erros interessantes: a confusão com a classe do verbo “vir” nos verbos “prever” (“preveNu”) e “entrever” (“entreveNu”). Também não acertou o alvo do verbo “rever” (hefexu), ficou faltando o traço de vozeamento nas duas últimas consoantes.

\section{Discussão}
\label{sec:discuss}

Durante a execução deste presente trabalho, foi divulgado um estudo semelhante realizado pelos pesquisadores \cite{kirov:2018}. Nesse estudo, Kirov e Cotterell revisitam a questão dos verbos irregulares do inglês utilizando a arquitetura \textit{Encoder-Decoder}. Em razão da similaridade entre os estudos, é interessante pautar a discussão dos resultados obtidos através de uma comparação entre os mesmos. 

\cite{kirov:2018} afirmam ter obtido uma acurácia de 100\% durante o treinamento considerando verbos regulares e irregulares. Durante a fase de teste, o desempenho no grupo dos verbos regulares se mantém próximo ao anterior (99.7\%), entretanto, ao observarmos exclusivamente o desempenho no grupo dos verbos irregulares, a acurácia cai para 28.6\%. \cite{kirov:2018} explicam que os erros obtidos nessa circunstância foram erros de \textit{regularização}, e que em nenhum momento o modelo misturou regulares e irregulares como em “\textit{gaved}” (erro observado e criticado no modelo de \cite{rumelhart:1986}). Em números absolutos, os autores apresentaram um corpus composto por 4039 verbos (aproximadamente 10 vezes maior que o utilizado nesta pesquisa), sendo 168 destes considerados irregulares. Eles ainda afirmam ter realizado uma partição aleatória tripla no corpus com proporções 80-10-10 para treino, desenvolvimento e teste. Desse modo, a base de teste dos pesquisadores continha apenas ~17 verbos irregulares. Com uma acurácia de 28\%, isso significa que o modelo \textit{Encoder-Decoder} apresentado pelos autores acertou apenas cinco verbos irregulares durante o treinamento. Sobre estes verbos, três deles eram verbos derivados de outros através de prefixação: \textit{retell}, \textit{partake} e \textit{withdraw}. Outro era o verbo \textit{sling}, similar a outros verbos presentes no treino (\textit{fling}, por exemplo). O último foi o verbo \textit{forsake}, cuja terminação se assemelha bastante ao verbo \textit{take}.

Além do modelo \textit{Encoder-Decoder} apresentado, \cite{kirov:2018} também reproduziram o modelo de regras de \cite{Albright2003RulesVA} utilizando o mesmo corpus. Segundo os autores, para este modelo a acurácia dentre os verbos regulares ficou em torno de 95\%, porém não foi capaz de acertar nenhum verbo irregular em nenhuma das três partições realizadas. 

Em termos de pré-processamento, \cite{kirov:2018}
não explicitam a codificação utilizada, mas explicam que não utilizaram traços fonéticos como dados de entrada para a rede. A base de dados utilizada pelos autores foi a CELEX, retirada de \cite{Baayen1993TheCL}, e era composta pelos verbos transcritos com a notação do AFI. 

Voltando ao modelo desenvolvido nesta pesquisa, observamos que a acurácia obtida no grupo dos verbos regulares foi a mais baixa em comparação aos demais trabalhos aqui referidos. Entretanto, isto pode ser explicado em razão do tamanho do corpus necessário para o aprendizado em arquiteturas mais complexas, como é o caso do \textit{Encoder-Decoder}. Como comparação, o modelo de \cite{cholletseq2seq} para tradução automática, o qual foi utilizado como referência para este trabalho, possui mais de 170 mil duplas de sentenças (francês - inglês). Entretanto, como a ordem de grandeza obtida (423 verbos) estava próxima à ordem do conjunto de \cite{rumelhart:1986} (506 verbos), e por se tratarem de verbos e não sentenças inteiras, acreditava-se que o corpus obtido seria suficiente para a obtenção de acurácias mais altas. Por outro lado, nota-se também uma grande diferença entre as proporções de verbos regulares e irregulares dos corpus obtidos. No caso de \cite{rumelhart:1986}, os verbos irregulares constituíam cerca de 20\% no corpus, no caso de \cite{kirov:2018}, 5\% e nesta pesquisa, aproximadamente 50\%. Desse modo, pode-se pensar que a maior proporção de verbos irregulares no corpus também pode ter influenciado na baixa acurácia obtida nos verbos regulares. 

Com relação aos verbos irregulares, apesar da acurácia média obtida ter sido 9.23\%, em números relativos isso representa em torno de 20 verbos. Dessa forma, é difícil comparar os dois modelos nessa questão. Se por um lado o montante de verbos regulares obtidos por \cite{kirov:2018} os levou a quase 100\% de acurácia nesse grupo, também foi um fator desfavorável para o desempenho do grupo dos verbos irregulares. Também pode-se questionar se os cinco verbos corretos obtidos por \cite{kirov:2018} refletem de fato o potencial de 23\% de acurácia do modelo. Certamente a discussão seria mais interessante caso tivéssemos uma acurácia média com desvio-padrões, como foi apresentado neste trabalho. Em comparação com o modelo de \cite{Albright2003RulesVA}, o modelo apresentado nesta pesquisa apresentou desempenho melhor nesse quesito. 

No que diz respeito aos tipos de erros encontrados, pode-se dizer que os erros observados nesta pesquisa tem caráter menos comportado do que os apresentados por \cite{kirov:2018}. Entretanto, como os processos de codificação e decodificação, bem como os objetos dos estudos são diferentes (traços fonéticos x fonemas), também seria injusta uma comparação entre os modelos nesse sentido.

Sobre o desempenho obtido pelo modelo comparando-se as classes de verbos irregulares, chama a atenção que a classe do verbo “botar” tenha apresentado a maior acurácia. Como apresentado na Seç. \ref{sec:resultado}, a classe desse verbo, além de apresentar uma proporção no corpus relativamente alta e ser também a classe com comprimento médio mais baixo, também foi comentado na Seç.\ref{sec:corpus} sobre o fato desta ser uma classe com exemplares diversificados. Desse modo, supõe-se que estes fatores combinados tenham levado a esse resultado. 

Para concluir a discussão, pode-se dizer que o desempenho do modelo \textit{Encoder-Decoder} está bastante condicionado à qualidade do corpus disponível. De certo que para qualquer modelo de Rede Neural, quanto maior o número de exemplos, melhor. Entretanto, vemos também que quanto mais profunda a arquitetura do modelo de Rede Neural, maior a demanda por mais exemplos. No que diz respeito ao aprendizado de verbos irregulares, especialmente no caso do Português Brasileiro, dificilmente conseguiríamos um corpus com dezenas de milhares de exemplos, quem dirá centenas de milhares. Ainda, vimos que a questão da proporção da classe de verbos irregulares mostrou-se relevante. Idealmente, um modelo computacional que fosse capaz de simular o aprendizado humano deveria ter seu desempenho independente das especificidades das línguas, já que o cérebro humano funciona dessa forma. Desse modo, os resultados obtidos nesta pesquisa não permitem dizer que o modelo \textit{Encoder-Decoder} seja adequado para esta tarefa. Contudo, pode-se argumentar que, como os processos de codificação e decodificação são módulos independentes da arquitetura, essa conclusão a respeito do modelo \textit{Encoder-Decoder} seria injusta. E que para que pudéssemos chegar a conclusões mais contundentes, mais experimentos deveriam ser realizados. Dito isto, direções para possíveis trabalhos futuros serão abordadas no próximo capítulo (Cap. \ref{ch:08}). 

  