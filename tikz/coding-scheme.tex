\definecolor{blue}{RGB}{159, 192, 176}
\definecolor{green}{RGB}{160, 227, 127}
\definecolor{orange}{RGB}{243, 188, 125}
\definecolor{red}{RGB}{253, 123, 84}
\definecolor{nephritis}{RGB}{39, 174, 96}
\definecolor{emerald}{RGB}{46, 204, 113}
\definecolor{turquoise}{RGB}{39, 174, 96}
\definecolor{green-sea}{RGB}{22, 160, 133}
\definecolor{purple}{RGB}{181, 124, 215}
% Tikzstyles for Computation Graphs

% nodes
\tikzstyle{noop} = [circle, draw=none, fill=red, minimum size = 10pt]
\tikzstyle{op} = [circle, draw=red, line width=1.5pt, fill=red!70, text=black, text centered, font=\bf \normalsize, minimum size = 25pt]

\tikzstyle{opintense} = [circle, draw=red, line width=1.5pt, fill=red!150, text=black, text centered, font=\bf \normalsize, minimum size = 25pt]


%new style
\tikzstyle{gp} = [circle, draw=red, line width=4pt, text=black, text centered, font=\bf \normalsize, minimum size = 4.cm]

\tikzstyle{box} = [rectangle, draw=red, line width=1.5pt, fill=red!70, text=black, align=center, font=\bf \normalsize, minimum size = 45pt]

\tikzstyle{box2} = [rectangle, draw=black, line width=0.9pt, text=black, align=center, font=\bf \normalsize, minimum size = 20pt]

\tikzstyle{box3} = [rectangle, draw=black, line width=0.9pt, fill=black, text=black, align=center, font=\bf \normalsize, minimum size = 20pt]

\tikzstyle{state} = [circle, draw=blue, line width=1.5pt, fill=blue!70, text=black, text centered, font=\bf \normalsize, minimum size = 25pt]

\tikzstyle{output} = [circle, draw=purple, line width=1.5pt, fill=purple!70, text=black, text centered, font=\bf \normalsize, minimum size = 25pt]


\tikzstyle{gradient} = [circle, draw=nephritis, line width=1.5pt, fill=nephritis!60, text=black, text centered, font=\bf \normalsize, minimum size = 25pt]
\tikzstyle{textonly} = [draw=none, fill=none, text centered, font=\bf \normalsize]
\tikzstyle{boxtextonly} = [draw=none, fill=none, align=center, font=\bf \normalsize]

\tikzstyle{normal} = [circle, draw=black, line width=1.0pt, fill=none, text=black, text centered, font=\bf \normalsize, minimum size = 20pt]


% edges
\tikzstyle{tedge}  = [draw, thick, >=latex, ->]
\tikzstyle{tedge_dashed}  = [draw, thick, >=latex, ->, dashed]
\tikzstyle{nedge}  = [draw, thick, >=latex]
\tikzstyle{nedge_dashed}  = [draw, thick, >=latex, dashed]


% namedscope
\tikzstyle{namedscope} = [circle, draw=orange, line width=1.5pt, fill=orange!60, align=center, inner sep=0pt]
\begin{figure}[ht!]
\centering

\scalebox{0.8}{
\begin{tikzpicture}[H]

%vetor
\node[box2] (box1) {0};
\node[box2, below=0pt of box1] (box2) {0};
\node[box3, below=0pt of box2] (box3) {1};
\node[box3, below=0pt of box3] (box4) {1};
\node[textonly, below=0pt of box4] (box5) {\reflectbox{$\vdots$}};
\node[box3, below=0pt of box5] (box6) {1};
\node[box3, below=0pt of box6] (box7) {1};
\node[box2, below=0pt of box7] (box8) {0};
\node[box3, below=0pt of box8] (box9) {1};
\node[box2, below=0pt of box9] (box10) {0};
% \node[textonly, below=0pt of box10] (dim) {460x1};
\node[textonly, below=0pt of dim] (space) {};
%wickelfeatures
\node[textonly, right=20pt of box1] (wi1) {\#, oclusiva, média};
\node[textonly, right=20pt of box2] (wi2) {\#, anterior, fricativa};
\node[textonly, right=20pt of box3] (wi3) {\#, interrompida, vogal};
\node[textonly, right=20pt of box4] (wi4) {\#, anterior, vogal};
\node[textonly, right=20pt of box5] (wi5) {\reflectbox{$\vdots$}};
\node[textonly, right=20pt of box6] (wi6) {\#, surda, baixa};
\node[textonly, right=20pt of box7] (wi7) {\#, surda, frontal};
\node[textonly, right=20pt of box8] (wi8) {contínua, vogal, vogal };
\node[textonly, right=20pt of box9] (wi9) {interrompida, vogal, nasal };
\node[textonly, right=20pt of box10] (wi10) {fricativa, posterior, \#};

%trigramas
\node[textonly, left=300pt of wi3] (tri1) {\#,k,A};
\node[textonly, below=10pt of tri1] (tri2) {k,A,m};
\node[textonly, below=10pt of tri2] (tri3) {A,m,\#};


%legenda
\node[textonly, right=40pt of wi3] (tit1) {ausente};
\node[textonly, below=10pt of tit1] (tit2) {presente};

\node[box2, right=15pt of wi3] (fig1) {};
\node[box3, below=10pt of fig1] (fig2) {};
\node[text, below=10pt of wi10] (nada) {};

\draw [decorate,decoration={brace,amplitude=10pt},xshift=-10pt,yshift=0pt]
(-8.5,-5) -- (-8.5,-1) node [black,midway,xshift=-1cm] 
{\footnotesize };


%features
\node[textonly, right=20pt of tri1] (f1) {\#, interrompida, vogal};
\node[textonly, below=0pt of f1] (f2) {\#, interrompida, baixa};
\node[textonly, below=0pt of f2] (f3) {\#, surda, longa};
\node[textonly, below=0pt of f3] (f4) {\#, surda, posterior};
\node[textonly, below=0pt of f4] (f5) {\reflectbox{$\vdots$}};
\end{tikzpicture}
}\caption{Esquema de Codificação de Rumelhart e McClelland} 
\label{fig:wick}
\end{figure}