\input{definitions/colors}
\input{definitions/styles}
\begin{figure}[h]
\centering

\scalebox{1.0}{
\begin{tikzpicture}[auto]

% operations =========
% phon features 1
\node[textonly] (out1) {Output};
\node[textonly, right=25pt of out1] (gab) {Gabarito};


% FNN output
\node[normal, below=40pt of out1] (x1) {};
\node[normal, below=35pt of x1] (x2) {};
\node[normal, below=35pt of x2] (x3) {};

% from input
\node[normal_dashed, left=45pt of x1] (in1) {};
\node[normal_dashed, below=5pt of in1] (in2) {};
\node[normal_dashed, above=5pt of in1] (in3) {};

\node[normal_dashed, left=45pt of x2] (in12) {};
\node[normal_dashed, below=5pt of in12] (in22) {};
\node[normal_dashed, above=5pt of in12] (in32) {};

\node[normal_dashed, left=45pt of x3] (in13) {};
\node[normal_dashed, below=5pt of in13] (in23) {};
\node[normal_dashed, above=5pt of in13] (in33) {};

% FNN target
\node[normal, right=65pt of x1] (y1) {};
\node[normal, right=65pt of x2] (y2) {};
\node[normal, right=65pt of x3] (y3) {};
\node[text, below=15pt of x3] (nada) {};



% edges FNN
\path[arrows_dashed] (x1) -- (y1);
\path[arrows_dashed] (x2) -- (y2);
\path[arrows_dashed] (x3) -- (y3);



\end{tikzpicture}
}\caption{Comparações entre o \textit{Output} e o Gabarito} 
\label{fig:gabarito}
\end{figure}