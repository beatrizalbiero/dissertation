\definecolor{blue}{RGB}{159, 192, 176}
\definecolor{green}{RGB}{160, 227, 127}
\definecolor{orange}{RGB}{243, 188, 125}
\definecolor{red}{RGB}{253, 123, 84}
\definecolor{nephritis}{RGB}{39, 174, 96}
\definecolor{emerald}{RGB}{46, 204, 113}
\definecolor{turquoise}{RGB}{39, 174, 96}
\definecolor{green-sea}{RGB}{22, 160, 133}
\definecolor{purple}{RGB}{181, 124, 215}
% Tikzstyles for Computation Graphs

% nodes
\tikzstyle{noop} = [circle, draw=none, fill=red, minimum size = 10pt]
\tikzstyle{op} = [circle, draw=red, line width=1.5pt, fill=red!70, text=black, text centered, font=\bf \normalsize, minimum size = 25pt]

\tikzstyle{opintense} = [circle, draw=red, line width=1.5pt, fill=red!150, text=black, text centered, font=\bf \normalsize, minimum size = 25pt]


%new style
\tikzstyle{gp} = [circle, draw=red, line width=4pt, text=black, text centered, font=\bf \normalsize, minimum size = 4.cm]

\tikzstyle{box} = [rectangle, draw=red, line width=1.5pt, fill=red!70, text=black, align=center, font=\bf \normalsize, minimum size = 45pt]

\tikzstyle{box2} = [rectangle, draw=black, line width=0.9pt, text=black, align=center, font=\bf \normalsize, minimum size = 20pt]

\tikzstyle{box3} = [rectangle, draw=black, line width=0.9pt, fill=black, text=black, align=center, font=\bf \normalsize, minimum size = 20pt]

\tikzstyle{state} = [circle, draw=blue, line width=1.5pt, fill=blue!70, text=black, text centered, font=\bf \normalsize, minimum size = 25pt]

\tikzstyle{output} = [circle, draw=purple, line width=1.5pt, fill=purple!70, text=black, text centered, font=\bf \normalsize, minimum size = 25pt]


\tikzstyle{gradient} = [circle, draw=nephritis, line width=1.5pt, fill=nephritis!60, text=black, text centered, font=\bf \normalsize, minimum size = 25pt]
\tikzstyle{textonly} = [draw=none, fill=none, text centered, font=\bf \normalsize]
\tikzstyle{boxtextonly} = [draw=none, fill=none, align=center, font=\bf \normalsize]

\tikzstyle{normal} = [circle, draw=black, line width=1.0pt, fill=none, text=black, text centered, font=\bf \normalsize, minimum size = 20pt]


% edges
\tikzstyle{tedge}  = [draw, thick, >=latex, ->]
\tikzstyle{tedge_dashed}  = [draw, thick, >=latex, ->, dashed]
\tikzstyle{nedge}  = [draw, thick, >=latex]
\tikzstyle{nedge_dashed}  = [draw, thick, >=latex, dashed]


% namedscope
\tikzstyle{namedscope} = [circle, draw=orange, line width=1.5pt, fill=orange!60, align=center, inner sep=0pt]

% RNN STATE CELL ====================================

\begin{figure}[ht!]
\centering
\hspace*{-1.0cm}
\scalebox{0.9}{
\begin{tikzpicture}[auto]

% timestep 1
\rnnSimple{(1)}{(0)}{t1}{}

% % timestep 0
\node[normal, minimum size=40pt,left=50pt of ht1] (ht0) {$\vect{h}^{(0)}$};

% % timestep 2
\rnnSimple{(2)}{(1)}{t2}{right=50pt of ht1};
\node[textonly, below= 80pt of ht1] (p) {p};
\node[textonly, above= 80pt of ht1] (â) {a};

% % timestep 2
\rnnSimple{(3)}{(1)}{t3}{right=50pt of ht2};
\node[textonly, below= 80pt of ht2] (a) {a};
\node[textonly, above= 80pt of ht2] (r_o) {r};

% timestep3
\node[textonly, below= 80pt of ht3] (r) {r};
\node[textonly, above= 80pt of ht3] (end) {\textupsilon};

% % state transfers
\path[tedge] (ht0) edge node[above right = 2pt] {} (ht1);
\path[tedge] (ht1) edge node[above right = 2pt] {} (ht2);
\path[tedge] (ht2) edge node[above right = 2pt] {} (ht3);



\end{tikzpicture}
}%\scalebox
\caption{RNR - Modelo de Linguagem}
\label{fig:rnnpractice}
\end{figure}


