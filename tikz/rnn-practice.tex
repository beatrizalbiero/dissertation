\input{definitions/colors}
\input{definitions/styles}

% RNN STATE CELL ====================================

\begin{figure}[ht!]
\centering
\hspace*{-1.0cm}
\scalebox{0.9}{
\begin{tikzpicture}[auto]

% timestep 1
\rnnSimple{(1)}{(0)}{t1}{}

% % timestep 0
\node[normal, minimum size=40pt,left=50pt of ht1] (ht0) {$\vect{h}^{(0)}$};

% % timestep 2
\rnnSimple{(2)}{(1)}{t2}{right=50pt of ht1};
\node[textonly, below= 80pt of ht1] (p) {p};
\node[textonly, above= 80pt of ht1] (â) {a};

% % timestep 2
\rnnSimple{(3)}{(1)}{t3}{right=50pt of ht2};
\node[textonly, below= 80pt of ht2] (a) {a};
\node[textonly, above= 80pt of ht2] (r_o) {r};

% % state transfers
\path[tedge] (ht0) edge node[above right = 2pt] {} (ht1);
\path[tedge] (ht1) edge node[above right = 2pt] {} (ht2);
\path[tedge] (ht2) edge node[above right = 2pt] {} (ht3);

% timestep3
\node[textonly, below= 80pt of ht3] (r) {r};
\node[textonly, above= 80pt of ht3] (end) {\textupsilon};

\end{tikzpicture}
}%\scalebox
\caption{RNR - Modelo de Linguagem}
\label{fig:rnnpractice}
\end{figure}


