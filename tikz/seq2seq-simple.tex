\begin{figure}[ht!]
\centering
\begin{tikzpicture}
\node[punkt] (seq1) {Sequência 1};
\node[punkt, right=40pt of seq1] (seq2) {Sequência 2};

\path[tedge] (seq1) -- (seq2);

\node[punkt, below=40pt of seq2] (seq21) {O menino pequeno gosta do seu cachorro.};

\node[punkt, left=40 pt of seq21] (seq12) {The small boy likes his dog.};

\path[tedge] (seq12) -- (seq21);

\node[text, below=10 pt of seq21] (nada) {};

\end{tikzpicture}
\caption{Objetivo do Modelo Encoder-Decoder: Mapeamento de uma Sequência à Outra} 
\label{fig:seq2seq_simple}
\end{figure}