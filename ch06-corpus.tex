\chapter{Corpus}
\label{ch:06}

Para a execução dos experimentos, o escopo dos experimentos foi restringido a um único tempo, modo, pessoa e número, já que o português apresenta formas irregulares em múltiplas combinações destes elementos e seria necessário construir uma rede diferente para cada uma destas combinações. Optou-se por estudar as irregularidades presentes nas flexões de primeira pessoa do singular no tempo presente, modo indicativo.  
O corpus utilizado para o treinamento dessa rede foi baseado nas informações contidas em \url{https://www.conjugacao.com.br/verbos-irregulares/} e \url{https://www.conjugacao.com.br/verbos-regulares/}.\\
Primeiramente, foi realizada uma etapa de extração dos verbos e suas respectivas conjugações para um arquivo \textit{.csv} via técnica de \textit{webscraping}, uma técnica que extrai informações contidas nas páginas da web [incluir ref]. Em seguida, os verbos irregulares foram selecionados manualmente para diferentes famílias de verbos, ou seja, grupos que continham o mesmo padrão de flexão. Alguns dos verbos irregulares listados na fonte de referência não eram irregulares no processo de flexão de interesse e portanto foram realocados para o grupo de verbos regulares. Como um exemplo disto, um dos motivos para o verbo \textit{acudir} ser considerado irregular é a flexão que ocorre na terceira pessoa do singular no tempo presente e modo indicativo (acode), porém como na primeira pessoa do singular o padrão regular se mantém, esse verbo foi realocado para o grupo dos verbos regulares. Em seguida, foi realizado um experimento na tentativa de utilizar o transcritor fonético automático dispobilizado por \cite{guide:2016}, porém o transcritor falhou na tentativa de transcrever verbos cuja escrita coincide com substantivos, como por exemplo "apoio", "peso", "toco", "posto", "jogo", entre outros. Dessa forma, como haviam problemas na utilização do algoritmo, optou-se por fazer uma transcrição manual. No total, foram obtidos 423 verbos, 83 verbos a menos que no experimento realizado por \cite{rumelhart:1986}.

Dos 423 verbos extraídos, 20 foram considerados verbos sem possível agrupamento e foram alocados para uma classe chamada \textit{None} (verbos como "ir", "trazer" e "saber"), totalizando uma base de 214 verbos regulares e 209 irregulares (50.6\% e 49.4\% respectivamente). A Tabela \ref{tab:classes} associa os nomes dados aos tipos de irregularidades encontradas e apresenta um exemplo de cada uma delas, além do número de exemplos de cada e a proporção desse número no total. 


Explicar o que formam as classes, regularidades 

\begin{table}[H]
\begin{center}
\begin{tabular}{|l|c|c|c|c|}
\toprule
& Classes & Exemplos & Contagem & Proporção\\
\midrule
1 & iar:eiu & ansiar, anseio & 9 & 2.13\%\\
2 & o\_ar:O\_u & tocar, toco & 30 & 7.09\%\\
3 & o\_ir:u\_o & cobrir, cubro & 7 & 1.65\%\\
4 & izer:igu & dizer, digo & 7 & 1.65\%\\
5 & fazer:fasu & fazer, faço & 15 & 3.55\%\\
6 & crer:eiu & crer, creio & 5 & 1.18\%\\
7 & entir:intu & sentir, sinto & 8 & 1.89\% \\
8 & edir:Esu & pedir, peço & 7 & 1.65\%\\
9 & or:oNu & pôr, ponho & 27 & 6.38\%\\
10 & e\_ir:i\_u & seguir, sigo & 27 & 6.38\%\\
11 & ter:teNu & ter, tenho & 10 & 2.36\%\\
12 & e\_ar:E\_u & testar, testo & 20 & 4.73\%\\
13 & ver:veju & ver, vejo & 6 & 1.42\%\\
14 & vir:veniu & vir, venho & 10 & 2.60\%\\
15 & None & saber, sei & 20 & 4.73\%\\
16 & Regulares & falar, falo & 214 & 50.59\%\\
\bottomrule
\end{tabular}
\end{center}
\captionof{table}{Organização do corpus}
\label{tab:classes}
\end{table}