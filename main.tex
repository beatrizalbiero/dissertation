\documentclass[12pt,twoside,a4paper]{book}

% ---------------------------------------------------------------------------- %
% Pacotes 
\usepackage[T1]{fontenc}
\usepackage[portuguese]{babel}
\usepackage{floatrow}
\newfloatcommand{capbtabbox}{table}[][\FBwidth]
\usepackage{blindtext}

% \usepackage[latin1]{inputenc}
\usepackage[utf8x]{inputenc}
\usepackage[pdftex]{graphicx}           % usamos arquivos pdf/png como figuras
\usepackage[toc,page]{appendix}
\usepackage{setspace}                   % espaçamento flexível
\usepackage{float}
\usepackage{indentfirst}                % indentação do primeiro parágrafo
\usepackage{makeidx}                    % índice remissivo
\usepackage[nottoc]{tocbibind}          % acrescentamos a bibliografia/indice/conteudo no Table of Contents
\usepackage{courier}  % usa o Adobe Courier no lugar de Computer Modern Typewriter
\usepackage{vowel}
\usepackage{type1cm}                    % fontes realmente escaláveis
\usepackage{pdfpages}
\usepackage{tipa}                       % para utilizar simbolos foneticos
\usepackage{listings}                   % para formatar código-fonte (ex. em Java)
\usepackage{titletoc}
%\usepackage[landscape,dvips]{geometry}
\usepackage{tabularx} 
\addto\captionsportuguese{% Replace "english" with the language you use
  \renewcommand{\contentsname}%
    {Sumário}%
}
\newcommand{\aup}{\textsuperscript}

\usepackage{lastpage}
\usepackage{tipa}
\let\ipa\textipa
\usepackage[utf8]{inputenc}
\newcommand{\BlankCell}{}
\usepackage{lmodern}
\usepackage{amsmath}
\usepackage{amsfonts}
\usepackage{booktabs}
\usepackage{bm}
\usepackage{epigraph}
\setlength\epigraphwidth{.6\textwidth}
\setlength\epigraphrule{0pt}

%\usepackage[bf,small,compact]{titlesec} % cabeçalhos dos títulos: menores e compactos
\usepackage[fixlanguage]{babelbib}
\usepackage[font=normalsize,format=plain,labelfont=bf,up,textfont=it,up,skip=0pt]{caption}
\usepackage[usenames,svgnames,dvipsnames]{xcolor}
\usepackage[a4paper,top=2.54cm,bottom=2.0cm,left=2.0cm,right=2.54cm]{geometry} % margens
%\usepackage[pdftex,plainpages=false,pdfpagelabels,pagebackref,colorlinks=true,citecolor=black,linkcolor=black,urlcolor=black,filecolor=black,bookmarksopen=true]{hyperref} % links em preto
\usepackage[pdftex,plainpages=false,pdfpagelabels,pagebackref,colorlinks=true,citecolor=black,linkcolor=black,urlcolor=black,filecolor=black,bookmarksopen=true]{hyperref} % links coloridos
\usepackage[all]{hypcap}                    % soluciona o problema com o hyperref e capitulos
\usepackage[round,sort,nonamebreak]{natbib} % citação bibliográfica textual(plainnat-ime.bst)
%\setcitestyle{numbers}
\usepackage{pgfplots}
\usepackage{hyperref}
\fontsize{70}{72}\usefont{OT1}{cmr}{m}{n}
{\selectfont}
\nocite{*}

% ---------------------------------------------------------------------------- %
% Cabeçalhos similares ao TAOCP de Donald E. Knuth
\usepackage{fancyhdr}
\pagestyle{fancy}
\fancyhf{}

% new commands------------------------
\usepackage{subcaption}
\newcommand{\vect}[1]{\bm{#1}}
\newcommand{\myprime}[1]{{#1}^{\prime}}
\newcommand{\grad}[2]{\nabla_{#1} {#2}}
\newcommand{\dotp}[2]{{#1}^{\top}{#2}}
\newcommand{\dotpPright}[2]{{#1}^{\top}\left({#2}\right)}
\newcommand{\outerp}[2]{\left({#1}\right){#2}^{\top}}
\newcommand{\Jacobian}[2]{\frac{\partial #1}{\partial #2}}
\newcommand{\Vocab}{\mathbb{V}}
\newcommand{\corpus}{\mathbb{C}}
\newcommand{\A}{\mathcal{A}}
\DeclareMathOperator*{\argmin}{arg\,min}
 \DeclareMathOperator*{\argmax}{arg\,max}
\DeclareMathOperator{\E}{\mathbb{E}}

% new commands------------------------

\renewcommand{\chaptermark}[1]{\markboth{\MakeUppercase{#1}}{}}
\renewcommand{\sectionmark}[1]{\markright{\MakeUppercase{#1}}{}}
\renewcommand{\headrulewidth}{0pt}

% ---------------------------------------------------------------------------- %
\graphicspath{{./figuras/}}             % caminho das figuras (recomendável)
\frenchspacing                          % arruma o espaço: id est (i.e.) e exempli gratia (e.g.) 
\urlstyle{same}                         % URL com o mesmo estilo do texto e não mono-spaced
\makeindex                              % para o índice remissivo
\raggedbottom                           % para não permitir espaços extra no texto
\fontsize{70}{72}\usefont{OT1}{cmr}{m}{n}{\selectfont}
\cleardoublepage
\large

% ---------------------------------------------------------------------------- %
% Opções de listing usados para o código fonte
% Ref: http://en.wikibooks.org/wiki/LaTeX/Packages/Listings
\lstset{ %
language=Python,                  % choose the language of the code
basicstyle=\footnotesize,       % the size of the fonts that are used for the code
numbers=left,                   % where to put the line-numbers
numberstyle=\footnotesize,      % the size of the fonts that are used for the line-numbers
stepnumber=1,                   % the step between two line-numbers. If it's 1 each line will be numbered
numbersep=5pt,                  % how far the line-numbers are from the code
showspaces=false,               % show spaces adding particular underscores
showstringspaces=false,         % underline spaces within strings
showtabs=false,                 % show tabs within strings adding particular underscores
frame=single,	                % adds a frame around the code
framerule=0.6pt,
tabsize=2,	                    % sets default tabsize to 2 spaces
captionpos=b,                   % sets the caption-position to bottom
breaklines=true,                % sets automatic line breaking
breakatwhitespace=false,        % sets if automatic breaks should only happen at whitespace
escapeinside={\%*}{*)},         % if you want to add a comment within your code
backgroundcolor=\color[rgb]{1.0,1.0,1.0}, % choose the background color.
rulecolor=\color[rgb]{0.8,0.8,0.8},
extendedchars=true,
xleftmargin=10pt,
xrightmargin=10pt,
framexleftmargin=10pt,
framexrightmargin=10pt
}

% Additional packages
\usepackage{lscape}

\usepackage{booktabs}
\usepackage{graphicx}
% \usepackage[table,xcdraw]{xcolor}
\usepackage{colortbl}
\usepackage{algpseudocode}
\usepackage{soul}
\usetikzlibrary{matrix,chains,positioning,decorations.pathreplacing,arrows}

%circle table
\newcommand*\circled[1]{\tikz[baseline=(char.base)]{% <---- BEWARE
            \node[shape=circle,draw,inner sep=2pt] (char) {#1};}}

% Declaracoes em Português
\algrenewcommand\algorithmicend{\textbf{fim}}
\algrenewcommand\algorithmicdo{\textbf{faça}}
\algrenewcommand\algorithmicwhile{\textbf{enquanto}}
\algrenewcommand\algorithmicfor{\textbf{para}}
\algrenewcommand\algorithmicif{\textbf{se}}
\algrenewcommand\algorithmicthen{\textbf{então}}
\algrenewcommand\algorithmicelse{\textbf{senão}}
\algrenewcommand\algorithmicreturn{\textbf{devolve}}
\algrenewcommand\algorithmicfunction{\textbf{função}}

% Rearranja os finais de cada estrutura
\algrenewtext{EndWhile}{\algorithmicend\ \algorithmicwhile}
\algrenewtext{EndFor}{\algorithmicend\ \algorithmicfor}
\algrenewtext{EndIf}{\algorithmicend\ \algorithmicif}
\algrenewtext{EndFunction}{\algorithmicend\ \algorithmicfunction}
% O comando For, a seguir, retorna 'para #1 -- #2 até #3 faça'
\algnewcommand\algorithmicto{\textbf{até}}
\algrenewtext{For}[3]%
{\algorithmicfor\ #1 $\gets$ #2 \algorithmicto\ #3 \algorithmicdo}

\usepackage{adjustbox}
\usepackage{fancyvrb}
\usepackage{amsmath}
\usepackage{amsthm}
\usepackage{amssymb}

\usepackage{cases}
\newcommand\tab[1][1cm]{\hspace*{#1}}
\usepackage[ruled,vlined,linesnumbered]{algorithm2e}
\SetKwComment{Comment}{$\triangleright$\ }{}
\usepackage{tikz}
\usetikzlibrary{shapes,arrows,positioning,fit,backgrounds, arrows.meta}
\newcommand{\empt}[2]{$#1^{\langle #2 \rangle}$}
\include{tikz/styles}
\usepackage{tkz-euclide}
\usepackage{tikz-3dplot}

\newtheorem{example}{Example}[section]
\newtheorem{definition}{Definition}[section]
\newtheorem{theorem}{Theorem}[section]
\newtheorem{corollary}{Corollary}[theorem]
\newtheorem{lemma}{Lemma}[theorem]
\newtheorem*{remark}{Remark}


% ---------------------------------------------------------------------------- %
% Corpo do texto
\begin{document}
\frontmatter 
% cabeçalho para as páginas das seções anteriores ao capítulo 1 (frontmatter)
\fancyhead[RO]{{\footnotesize\rightmark}\hspace{2em}\thepage}
\setcounter{tocdepth}{2}
\fancyhead[LE]{\thepage\hspace{2em}\footnotesize{\leftmark}}
\fancyhead[RE,LO]{}
\fancyhead[RO]{{\footnotesize\rightmark}\hspace{2em}\thepage}


\onehalfspacing  % espaçamento

% ---------------------------------------------------------------------------- %
% CAPA
% Nota: O título para as dissertações/teses do IME-USP devem caber em um 
% orifício de 10,7cm de largura x 6,0cm de altura que há na capa fornecida pela SPG.
\thispagestyle{empty}
\begin{center}
    \vspace*{2.3cm}
    \textbf{\Large{O modelo Encoder-Decoder aplicado em irregularidades verbais do Português Brasileiro
}}\\
    
    
    \vskip 2cm
    \textsc{
    Dissertação de Mestrado\\[+0.5cm]
    Programa de Mestrado em Semiótica e Linguística Geral\\[+0.5cm]
    Universidade de São Paulo\\[+0.5cm]
   }
    
    \vspace*{6.2cm}
    Beatriz Albiero\\
    Orientador: Prof. Dr. Marcelo Barra Ferreira\\
    Departamento de Linguística FFLCH-USP
    
    \vskip 0.5cm
    \normalsize{São Paulo\\ 2019}
\end{center}


\pagenumbering{roman}     % começamos a numerar

\thispagestyle{empty}

\begin{center}
    Universidade de São Paulo\\
    Faculdade de Filosofia, Letras e Ciências Humanas\\
    Departamento de Linguística\\
    Programa de Pós-Graduação em Linguística 
\end{center}
\begin{center}
    \vspace*{1.3cm}
    \textbf{\Large{O modelo Encoder-Decoder aplicado em irregularidades verbais do Português Brasileiro
}}\\
\end{center}
\vspace{1.0 cm}

\begin{center}
\Large{Beatriz Albiero}
\end{center}

\vspace*{4.3cm}  
\hfill\begin{minipage}{0.5\linewidth}
  Dissertação apresentada ao Programa de Pós-Graduação em Linguística do Departamento de Linguística da Faculdade de Filosofia, Letras e Ciências Humanas da Universidade de São Paulo, para obtenção do título de Mestre em Letras. \\
  
  De acordo:\\
  
\_\_\_\_\_\_\_\_\_\_\_\_\_\_\_\_\_\_\_\_
  
 
\end{minipage}

 \begin{center}   
    \vskip 5.5cm
    \normalsize{São Paulo\\ 2019}
\end{center}
\begin{folhadeaprovacao}
  \textbf{\Large{Banca Examinadora}}
  \vskip 2cm
    \_\_\_\_\_\_\_\_\_\_\_\_\_\_\_\_\_\_\_
  
    Prof. Dr. Marcelo Barra Ferreira
    
    (USP) Presidente
    \vskip 2cm
    \_\_\_\_\_\_\_\_\_\_\_\_\_\_\_\_\_\_\_
      
    Prof. Dr. Pablo Picasso Feliciano de Faria
    
    (UniCamp) Membro Titular
    \vskip 2cm
    \_\_\_\_\_\_\_\_\_\_\_\_\_\_\_\_\_\_\_
    
    Prof. Dr. Marcos Fernando Lopes

    (USP) Membro Titular
    \vskip 2cm
    \_\_\_\_\_\_\_\_\_\_\_\_\_\_\_\_\_\_\_
    
    Profa. Dra. Livy Maria Real Coelho
    
    Membro Titular
  
    

\end{folhadeaprovacao}
% ---------------------------------------------------------------------------- %
% Dedicatória
% ---------------------------------------------------------------------------- %
% Agradecimentos
\chapter*{Agradecimentos}

%Começar o mestrado na Linguística foi uma das melhores coisas que eu já fiz por mim. 


% ---------------------------------------------------------------------------- %
% Abstract
\chapter*{Resumo}
\noindent Albiero, B. \textbf{O modelo Encoder-Decoder aplicado em irregularidades verbais do Português Brasileiro}. 
2019.
Dissertação (Mestrado) - Faculdade de Filosofia e Ciências Humanas,
Universidade de São Paulo, São Paulo, 2019.
\\

Inspirada na controversa questão da aquisição de verbos irregulares na língua inglesa (\cite{chomsky:1968}, \cite{Pinker:1999}, \cite{Pinker:1988}, \cite{Albright2003RulesVA}, \cite{kirov:2018}), esta pesquisa tem como objetivo estudar a questão do aprendizado de verbos irregulares do Português Brasileiro sob a ótica do modelo computacional \textit{Encoder-Decoder}. Desse modo, a tarefa proposta ao modelo era a de predizer uma forma verbal flexionada dada uma forma básica (\textit{Radical + Vogal Temática}). Para tanto, o escopo da pesquisa restringiu-se ao estudo do paradigma de 1\aup{a} Pessoa do Singular no Modo Indicativo e Tempo Presente. O modelo utilizado, por sua vez, é um modelo de caráter associativo que pertence ao grupo dos modelos de Redes Neurais Artificiais. Também, fez-se necessária a construção de um \textit{corpus} linguístico composto pelo paradigma selecionado e em seguida transcrito em notação fonética especial para viabilizar a interpretação pelo modelo escolhido. O \textit{corpus} produzido é composto por 423 verbos que foram marcados como pertencendo às famílias de verbos regulares (51\%) ou irregulares (49\%). Ainda, dentro do escopo da família de verbos irregulares, foi possível identificar 15 subgrupos irregulares baseado nos padrões de flexão encontrados. A partir da notação fonética fornecida, os verbos puderam ser associados a novas representações que continham informações relativas aos traços fonéticos presentes, tanto em suas formas básicas, quanto nas  respectivas formas flexionadas. Assim, o modelo proposto tenta predizer as formas flexionadas a partir da identificação das relações fonéticas envolvidas. O modelo apresentado foi submetido a múltiplos treinamentos e testes e apresentou uma acurácia média que se concentrou em torno de 13\%-14\%, mas chegou a acertar 17\% em um dos experimentos. Considerando a segmentação entre verbos regulares e irregulares, o modelo performou melhor na classe dos regulares. Entretanto, considerando-se todas as 16 classes individualmente (15 irregulares + 1 regular), pôde-se observar que as duas primeiras classes em que o modelo performou melhor eram classes irregulares, deixando a classe regular como a terceira com os melhores resultados.

\\
 \textbf{Palavras-chave:}
linguística computacional, fonética, redes neurais, conexionismo, aquisição.

\chapter*{Abstract}

% ---------------------------------------------------------------------------- %
% Sumário
\tableofcontents    % imprime o sumário

% ---------------------------------------------------------------------------- %
\chapter{Lista de Abreviações}
\begin{tabular}{ll}
\vspace{3mm}
% \textbf{Acc} 		 & Accuracy\\ \vspace{3mm}
\textbf{FFD} 		 & Feedforward\\ 
\vspace{3mm}
\textbf{AFI} 		 & Alfabeto Fonético Internacional\\ \vspace{3mm}
\textbf{LSTM} 		 & Long-Short Term Memory\\ \vspace{3mm}
\textbf{PLN} 		 & Processamento de Linguagem Natural\\ 
\vspace{3mm}
\textbf{ReLU} 		 & Rectifier Linear Unit\\ \vspace{3mm}
\textbf{RNR} 		 & Rede Neural Recorrent\\ \vspace{3mm}
\textbf{GRU} 		 & Gated Recurrent Unit\\ 
\vspace{3mm}


% \textbf{seq2seq} 	 & Sequence-to-Sequence\\ \vspace{3mm}
% \textbf{Val\_Acc} 	 & Validation Accuracy\\ \vspace{3mm}
% \textbf{Val\_Loss} 	 & Validation Loss\\ \vspace{3mm}

\end{tabular}


% ---------------------------------------------------------------------------- %
\chapter{Lista de Símbolos}
\begin{tabular}{ll}
\vspace{2mm}
\textbf{\#}   &Indicador de Fronteira (Início ou final de palavra)\\ \vspace{2mm}
\textbf{<eos>}   &Token End of Sentence (Final de Sentença)\\ \vspace{2mm}
% $b$    &Bias\\ \vspace{2mm}
% $\vect{c}$    &Matriz de Estados (Cell States)\\ \vspace{2mm}
% $\vect{h}$    &Matriz de Estados\\ \vspace{2mm}
$\vect{x}$    &Vetor de Inputs\\ \vspace{2mm}
$\vect{\hat{y}}$    &Vetor de Outpus\\ \vspace{2mm}
$\vect{w}$    &Vetor\\ \vspace{2mm}
$\vect{W}$    &Matriz\\ \vspace{2mm}
%$\corpus$     &Corpus\\ \vspace{2mm}
%$\Vocab$      &Vocabulary \\ \vspace{2mm}
%$[\vect{a}; \vect{b}]$ &Concatenation of the vectors $\vect{a}$ and $\vect{b}$ \\ \vspace{2mm}
$\sigma$ & Função Sigmoid \\ \vspace{2mm}
% $\tanh$ & Tangente Hiperbólica \\ \vspace{2mm}
\end{tabular}


% ---------------------------------------------------------------------------- %
% Listas de figuras e tabelas criadas automaticamente
\listoffigures            
% \listoftables            

% ---------------------------------------------------------------------------- %
% Capítulos do trabalho
\mainmatter

% cabeçalho para as páginas de todos os capítulos
\fancyhead[RE,LO]{\thesection}

% \singlespacing              % espaçamento simples
\onehalfspacing            % espaçamento um e meio

\chapter{Introdução}
\label{ch:01}

A questão do aprendizado infantil, referente ao processo de flexão de verbos irregulares na língua inglesa, está certamente entre um dos temas de debate mais controversos %entre as principais correntes teóricas no estudo da 
do campo da Linguística (Pinker, \citeyear{Pinker:1999}). O cerne do debate está na exata caracterização dos mecanismos que possibilitam que um falante seja capaz de relacionar um verbo na forma não flexionada (\textit{walk}, por exemplo) à sua forma flexionada no \textit{Simple Past} (\textit{walked}).

Os verbos no tempo passado do inglês podem ser subdivididos em uma variedade de famílias. Um primeiro grupo é a forma aceita como a \textit{regular}, cuja forma ortográfica corresponde à aplicação da regra \textit{\text{stem} + ed}, como em \textit{walk}. %Entretanto, sob uma perspectiva fonológica, ainda é possível dividir esse grupo em três menores seguindo as variações possíveis do segmento \textit{ed}: (i) [-\textsci d], (ii) [-d] e (iii) [-t]. Em (i), observa-se que o segmento [-\textsci d] é utilizado sempre nos casos em que o último fonema do \textit{stem} for um [t] ou [d], como por exemplo o verbo \textit{pad} $\rightarrow$ \textit{padded} ([p\ae d] $\rightarrow$ [p\ae d\textsci d]). A situação (ii) é utilizada sempre que último fonema do \textit{stem} for uma vogal ou uma consoante sonora, como por exemplo em \textit{drag}$\rightarrow$ dragged ([dr\ae g]) $\rightarrow$ [dr\ae gd] ou show$\rightarrow$ showed ([\textesh o\textupsilon]$\rightarrow$[\textesh o\textupsilon d]). Por sua vez, o caso (iii) é aplicado sempre após uma consoante surda (\textit{sack}$\rightarrow$ sacked ([s\ae k] $\rightarrow$ [s\ae kt]). 
% ocorrendo não somente a formação de um padrão regular (que ortograficamente corresponde à composição verbo + \textit{ed}), mas também à formação de subgrupos de verbos irregulares, como por exemplo:
Dentre os verbos irregulares, estes podem ser considerados ou supletivos, como por exemplo \textit{go} $\rightarrow$ \textit{went}, ou podem se conglomerar seguindo padrões fonéticos de flexão similares (\cite{Nelson:2010}):

\begin{enumerate}
    \item blow – blew, grow – grew, know – knew, throw – threw
    \item bear – bore, swear – swore, tear – tore, wear – wore
    \item drink – drank, shrink – shrank, sink – sank, stink – stank 
\end{enumerate}

É possível pensar que o aprendizado de tais padrões dependeria de uma memorização caso a caso. No entanto, a pesquisa de \cite{Bybee:1983} mostra um estudo psicolinguístico em que indivíduos são apresentados a diversos verbos inventados (hipoteticamente em uma forma não flexionada). A pesquisa revelou que, ao invés de aplicarem sistematicamente a regra regular (verbo + \textit{ed}), os indivíduos apresentaram tendências à alocação de alguns verbos em alguns subgrupos irregulares. % um estudo psicolinguístico conduzido a partir da  verbos inventados revelou que os indivíduos apresentam tendências com relação à alocação dos verbos também em grupos de verbos irregulares, ao invés de sistematicamente aplicar a regra regular ( + \textit{ed}). 
Por exemplo, para o verbo inventado “\textit{spling}”, a maioria dos indivíduos optou pela forma “\textit{splang}”  ou “\textit{splung}”. Este exemplo contradiz a ideia de que os falantes poderiam estar apenas reproduzindo formas memorizadas e sugere que eles estejam ativamente identificando padrões, e mais: possuem uma intuição natural sobre a adequabilidade da alocação de um verbo a um grupo de verbos ou a outro. 

A partir do exemplo dado, é razoável deduzir que a motivação por de trás de tais tendências ocorra a partir das similaridades entre as unidades fonéticas dos verbos inventados e os verbos reais que já apresentam uma flexão de caráter irregular. Entretanto, as circunstâncias que levam à aquisição dessa \textit{intuição} linguística são indeterminadas. Por um lado, faz sentido dizer que para que um ser humano seja capaz de introduzir-se ao mundo dos falantes, é necessário que ele seja dotado de algumas pré-disposições para tal, caso contrário seria possível ensinar essa forma de comunicação para outras espécies. Em contrapartida, estudos mostram que crianças privadas do contato com uma sociedade falante se tornam permanentemente incapazes de dominar integralmente a gramática de uma língua (\cite{Pinker:languageinstinct}), o que nos leva a concluir que a experiência das crianças com a sociedade, assim como as suas próprias pré-disposições genéticas são parcialmente responsáveis pelo processo de desenvolvimento da linguagem. A dificuldade, está portanto, na tentativa de se quantificar, delimitar e apontar os conhecimentos adquiridos a partir do contato cultural, bem como os conhecimentos linguísticos ditos \textit{inatos}. É, portanto, em torno desta questão que tem início o debate a respeito do aprendizado dos verbos irregulares da língua inglesa.

%https://plato.stanford.edu/entries/rationalism-empiricism/


De um lado do debate, encontra-se a teoria da Fonologia Gerativa de Chomsky e Halle (\citeyear{chomsky:1968}). Nesta teoria, os indivíduos seriam portadores de um dispositivo de aquisição de linguagem (\textit{LAD} - Language Acquisition Device) responsável pela \textit{formulação} e \textit{manipulação} de estruturas fonológicas abstratas em um sistema intrincado de regras. De modo simplificado, a teoria propõe que o falante seja capaz de identificar e formular regras intuitivamente para dar conta do aprendizado das formas irregulares da língua. Um exemplo disso é a família dos verbos terminados em “-ind”.

\begin{center}
bind – bound, find – found, grind – ground, wind – wound
\end{center}

Vemos que, de modo simplificado, pode-se propor uma regra baseada em uma gramática sensível a contextos (Context-Sensitive Grammar (CSG) (ref)) 
%verificar essa formula fonetica

\begin{center}
a\textsci $\rightarrow$ a\textupsilon / \textbf{X}  \_\_nd]+past
\end{center}

A regra proposta sugere que o segmento [a\textsci] se transforma em [a\textupsilon] quando terminado em [nd] e flexionado para o passado. O símbolo \_\_ representa o local aonde ocorre tal transformação e \textbf{X} representa uma unidade fonológica arbitrária. 

Em outras palavras, pode-se dizer que o conhecimento dito \textit{inato} defendido por Chomsky e Halle refere-se a uma certa capacidade cognitiva de formulação de regras a partir da identificação de alguns elementos fundamentais (como por exemplo os elementos apontados na regra proposta). Uma estrutura como essa permitiria ao falante construir generalizações e, eventualmente, abstrair as regras fonológicas de sua língua. \\

Do outro lado do debate, os pesquisadores \cite{rumelhart:1986} confrontam a teoria anterior ao argumentar que comportamentos de caráter regrado podem ser reproduzidos por mecanismos que não dependam de nenhuma manipulação simbólica. Ao invés disso, os pesquisadores sugerem que os mecanismos envolvidos no processo de flexão verbal possam ser construídos de tal forma que a sua performance possa ser caracterizada por regras, mas que as regras em si não estejam representadas explicitamente em nenhuma parte do processo. Para sustentar essa ideia, \cite{rumelhart:1986} apresentam um modelo computacional baseado em padrões associativos que não fazem uso de construções com regras desse tipo. Posteriormente, o modelo construído foi fundamental para o surgimento de uma nova escola dentro das ciências cognitivas: o conexionismo.\\

\input{definitions/colors}
\input{definitions/styles}
\begin{figure}[ht!]
\centering

\scalebox{1.0}{
\begin{tikzpicture}[auto]

% operations =========
% phon features 1
\node[textonly] (1pho1) {int-vogal-int};

% Legenda
\node[textonly, above=10pt of 1pho1] (leg1) {Unidades de Input};


% FNN input
\node[normal, right=5pt of 1pho1] (x1) {};
\node[normal, below=25pt of x1] (x2) {};
\node[normal, below=25pt of x2] (x3) {};
\node[normal, below=25pt of x3] (x4) {};
\node[normal, below=25pt of x4] (x5) {};
\node[normal, below=25pt of x5] (x6) {};
\node[text, below=10pt of x6] (nada) {};

% FNN output
\node[normal, right=45pt of x1] (y1) {};
\node[normal, right=45pt of x2] (y2) {};
\node[normal, right=45pt of x3] (y3) {};
\node[normal, right=45pt of x4] (y4) {};
\node[normal, right=45pt of x5] (y5) {};
\node[normal, right=45pt of x6] (y6) {};

% phon features 2
\node[textonly, right=5pt of y1] (2pho1) {int-vogal-int};
\node[textonly, above=10pt of 2pho1] (leg2) {Unidades de Output};
\node[textonly, left=25pt of x2] (1pho2) {anterior-nasal-posterior};
\node[textonly, right=25pt of y2] (2pho2) {anterior-nasal-posterior};
\node[textonly, left=25pt of x3] (3pho1) {...};
\node[textonly, right=25pt of y3] (1pho3) {...};
\node[textonly, left=25pt of x4] (4pho1) {nasal-cont-ocl};
\node[textonly, right=25pt of y4] (1pho4) {nasal-cont-ocl};
\node[textonly, left=25pt of x5] (5pho1) {médio-cont-baixa};
\node[textonly, right=25pt of y5] (1pho5) {médio-cont-baixa};
\node[textonly, left=25pt of x6] (6pho1) {vogal-fric-\#};
\node[textonly, right=25pt of y6] (1pho6) {vogal-fric-\#};
% edges FNN
\path[nedge] (x1) -- (y1);
\path[nedge] (x1) -- (y2);
\path[nedge] (x1) -- (y3);
\path[nedge] (x1) -- (y4);
\path[nedge] (x1) -- (y5);
\path[nedge] (x1) -- (y6);
\path[nedge] (x2) -- (y1);
\path[nedge] (x2) -- (y2);
\path[nedge] (x2) -- (y3);
\path[nedge] (x2) -- (y4);
\path[nedge] (x2) -- (y5);
\path[nedge] (x2) -- (y6);
\path[nedge] (x3) -- (y1);
\path[nedge] (x3) -- (y2);
\path[nedge] (x3) -- (y3);
\path[nedge] (x3) -- (y4);
\path[nedge] (x3) -- (y5);
\path[nedge] (x3) -- (y6);
\path[nedge] (x4) -- (y1);
\path[nedge] (x4) -- (y2);
\path[nedge] (x4) -- (y3);
\path[nedge] (x4) -- (y4);
\path[nedge] (x4) -- (y5);
\path[nedge] (x4) -- (y6);
\path[nedge] (x5) -- (y1);
\path[nedge] (x5) -- (y2);
\path[nedge] (x5) -- (y3);
\path[nedge] (x5) -- (y4);
\path[nedge] (x5) -- (y5);
\path[nedge] (x5) -- (y6);
\path[nedge] (x6) -- (y1);
\path[nedge] (x6) -- (y2);
\path[nedge] (x6) -- (y3);
\path[nedge] (x6) -- (y4);
\path[nedge] (x6) -- (y5);
\path[nedge] (x6) -- (y6);


\end{tikzpicture}
}\caption{Esquema do Modelo Apresentado pelos Pesquisadores Rumelhart e McClelland} 
\label{fig:esquemafdd}
\end{figure}


O modelo desenvolvido foi criado por analogia à estrutura em que se relacionam os neurônios no cérebro. Ele é composto basicamente por uma rede artificial de nódulos interconectados paralelamente (Fig. \ref{fig:esquemafdd}).

A primeira camada de nódulos do modelo é responsável por receber os dados de entrada (os \textit{inputs}), que são os dados referentes aos traços fonéticos distintivos que caracterizam os sons de um verbo não flexionado. Traços fonéticos podem ser caracterizados como propriedades distintivas das unidades fônicas (\cite{paraconhecer:2015}). Tais propriedades podem ser baseadas em critérios acústicos, articulatórios ou perceptuais. Na figura, cada nódulo é apresentado ao lado de uma sequência de três traços. O primeiro nódulo, por exemplo, refere-se à sequência \textbf{int-vogal-int}. Nesse caso, \textbf{int} (uma abreviação para \textit{interrompida}) indica uma propriedade comum entre algumas consoantes, referente à interrupção do fluxo de ar (como no fone [k], por exemplo). Na figura temos ainda \textbf{fric} para fricativas; \textbf{ocl} para oclusivas; vogais; \textbf{nasal} para nasalidade; locais da execução (anterior e posterior); traços de corpo da língua (média, baixa); entre outros. 

Ainda sobre a camada de \textit{input}, é possível observar que cada nódulo é representado por uma tríade de traços fonéticos. Esta foi uma solução encontrada pelos autores para realizar o mapeamento entre os traços dos verbos (da forma não flexionada para \textit{Past Simple}). A ideia era contornar a questão da inserção de dados de natureza sequencial variável (como é o caso de um verbo - composto por uma sequência de sons) de uma só vez no modelo. Nesse caso, cada tríade representa três traços fonéticos, cada um referente a um fone. Por exemplo, para o verbo \textit{came}, temos que cada um dos fones possui múltiplos traços. O fone /\textit{k}/, por exemplo, é uma consoante oclusiva, surda, anterior\footnote{O tema dos traços fonéticos utilizados será abordado em maior profundidade no Cap. \ref{ch:02}.}. Os fones subsequentes também são constituídos a partir de seus respectivos traços fonéticos. Desse modo, cada traço está associado a cada fone de uma sequência de três. Em seguida, o \textit{input} recebido é passado adiante para a próxima camada através de uma rede de conexões. 

A segunda camada é uma camada de resposta (\textit{output}) que tem como objetivo retornar dados referentes aos traços que caracterizam os sons do mesmo verbo fornecido no \textit{input}, porém no tempo passado. Concluída esta etapa, os dados de saída obtidos deverão ser então comparados à forma correta do verbo no tempo passado, através de uma espécie de gabarito, usualmente conhecido como \textit{alvo} ou também \textit{target} (Fig.\ref{fig:gabarito}). Feita essa comparação, é possível alterar a rede de conexões entre as camadas de \textit{input} e \textit{output} de modo a fortalecer (ou enfraquecer) as mesmas para atingir o objetivo proposto. Antes da primeira comparação, a rede é inicializada com conexões aleatórias. Conforme o número de comparações aumenta, a tendência é que as atualizações realizadas tenham servido como um aprendizado (uma espécie de \textit{treinamento}) e permitam que o modelo seja utilizado para flexionar os verbos.

\input{definitions/colors}
\input{definitions/styles}
\begin{figure}[h]
\centering

\scalebox{1.0}{
\begin{tikzpicture}[auto]

% operations =========
% phon features 1
\node[textonly] (out1) {Output};
\node[textonly, right=25pt of out1] (gab) {Target};


% FNN output
\node[normal, below=40pt of out1] (x1) {$y_{1}$};
\node[normal, below=35pt of x1] (x2) {$y_{2}$};
\node[normal, below=35pt of x2] (x3) {$y_{3}$};

% from input
\node[text, left=45pt of x1] (in1) {};
\node[text, left=45pt of x2] (in2) {};
\node[text, left=45pt of x3] (in3) {};


% comparison
\node[text, right=31pt of x1] (nada1) {};
\node[text, below=5pt of nada1] (nada2) {\small{Comparação}};
\node[text, right=31pt of x2] (nada6) {};
\node[text, right=31pt of x3] (nada7) {};

\node[text, left=31pt of x1] (nada3) {};
\node[text, left=31pt of x2] (nada4) {};
\node[text, left=31pt of x3] (nada5) {};


% FNN target
\node[normal, right=65pt of x1] (y1) {$\hat{y_{1}}$};
\node[normal, right=65pt of x2] (y2) {$\hat{y_{2}}$};
\node[normal, right=65pt of x3] (y3) {$\hat{y_{2}}$};
\node[text, below=15pt of x3] (nada) {};



% edges FNN
\path[arrows_dashed] (x1) -- (y1);
\path[arrows_dashed] (x2) -- (y2);
\path[arrows_dashed] (x3) -- (y3);

\draw[arrows_dashed, ->] (nada1) to [out=135,in=115] (nada3);
\draw[arrows_dashed, ->] (nada6) to [out=135,in=115] (nada4);
\draw[arrows_dashed, ->] (nada7) to [out=135,in=115] (nada5);

\path[tedge] (in1) -- (x1);
\path[tedge] (in2) -- (x1);
\path[tedge] (in3) -- (x1);

\path[tedge] (in1) -- (x2);
\path[tedge] (in2) -- (x2);
\path[tedge] (in3) -- (x2);

\path[tedge] (in1) -- (x3);
\path[tedge] (in2) -- (x3);
\path[tedge] (in3) -- (x3);



\end{tikzpicture}
}\caption{Comparações entre o \textit{Output} e o \textit{Target}} 
\label{fig:gabarito}
\end{figure}

Para realizar o treinamento, \cite{rumelhart:1986} introduzem 420 verbos no modelo repetidamente (200 vezes cada um, 84.000 inserções no total).  Após o treinamento, o modelo foi capaz de prever corretamente todos os 420 verbos inseridos. Além disso, em um novo conjunto com 86 verbos desconhecidos, acertou cerca de 3/4 dos verbos regulares presentes. Dentre os novos verbos irregulares, cometeu erros interessantes de \textit{regularização} (como \textit{catched} (ao invés de caught) e \textit{digged} (ao invés de dug). %(\cite{Pinker:1999}).  

Além desses resultados, \cite{rumelhart:1986} relatam que o processo de aprendizado do modelo apresentou um fenômeno interessante, reproduzindo um desempenho similar a comportamentos observáveis em crianças durante a fase de aquisição: a Curva de Desenvolvimento em U (U-shaped Development, \cite{marcus:1992}). A Curva de Desenvolvimento em U basicamente se refere a um processo de aprendizado que ocorre em três estágios: \\

(i) inicialmente, crianças aprendem a flexionar verbos corretamente (\textit{come}$\rightarrow$\textit{came});

(ii) em seguida passam por um processo de \textit{super-regularização} (em que produzem formas como \textit{comed}), conforme são capazes de assimilar uma quantidade maior de verbos e compreender que existe uma fórmula genérica que funciona quase sempre;

(iii) por fim, elas entendem (intuitivamente) que a fórmula regular não dá conta de todos os casos e passam a reproduzir corretamente tanto os verbos regulares quanto irregulares. \\
% micro e macro u shaped development


\cite{rumelhart:1986} descrevem como foi possível observar tal comportamento também no modelo computacional desenvolvido.
Na fase inicial do processo de treinamento, o modelo foi alimentado com uma quantidade pequena de verbos, como: \textit{come}, \textit{get}, \textit{give}, \textit{look}, \textit{take}, \textit{go}, \textit{have}, \textit{live} e \textit{feel}. A performance do modelo foi compatível com o primeiro estágio da curva, ou seja, para esses verbos foi capaz de identificar corretamente a forma correspondente no \textit{Simple Past}. Em um segundo momento, o modelo foi alimentado com uma quantidade muito maior de verbos. Nesse estágio é possível verificar que o modelo está passando por um processo de regularização sistemática dos verbos. Ele produziu resultados como: \textit{breaked}, \textit{comed}, \textit{gived}; e também combinações entre padrões regulares e irregulares (ex. \textit{gaved}). 
Após uma série de muitas inserções repetidas, o modelo finalmente foi capaz de responder corretamente a uma quantidade maior de verbos, assim como no último estágio do processo de aprendizado natural. 

Os resultados de \cite{rumelhart:1986} foram capazes de causar bastante alvoroço na comunidade científica da época. Muitos pesquisadores viam o novo modelo como uma completa mudança de paradigma, não apenas na Linguística, mas também como uma nova forma de se estudar aprendizado em geral (\cite{Schneider1987}). 

Apesar disso, \cite{Pinker:1988} dão continuidade ao debate ao apontar uma série de questões pertinentes em que o modelo falhou em explicar. Primeiramente, como o modelo recebe apenas uma representação fonética do verbo como \textit{input}, ele é incapaz de gerar duas respostas diferentes para verbos com sonoridade idêntica (por exemplo \textit{break}$\rightarrow$\textit{broke} e \textit{brake}$\rightarrow$\textit{braked}). Para realizar essas predições corretamente, o modelo precisaria de um módulo adicional para distinguir entre as duas palavras, o que o descaracterizaria como modelo puramente associativo. Em segundo lugar, o modelo é extremamente dependente dos padrões observados durante o treinamento, tendo uma capacidade baixa para generalizações. \cite{Pinker:1999} comenta que o modelo ficou mudo quando alimentado com os verbos \textit{jump}, \textit{pump}, \textit{warm, trail} e \textit{glare} (que dispõem de uma sonoridade razoavelmente incomum). Além disso, o modelo apresentou alguns resultados completamente distorcidos, como: \textit{squat – squakt, tour – toureder} e \textit{mail – membled}; associações inaceitáveis para qualquer falante nativo. 

Com relação ao padrão de aprendizado observado (a Curva de Desenvolvimento em U), \cite{Pinker:1999} explica que esse comportamento foi provocado segundo a forma em que os verbos foram inseridos no modelo durante o treinamento: Rumelhart & McClelland realizaram o treinamento \textit{em partes}. Na primeira parte do treinamento, alimentaram o modelo com alguns verbos de alta frequência na língua inglesa (muitos deles irregulares). Em seguida, treinaram o modelo com esses verbos, reintroduzindo-os múltiplas vezes até que o modelo conseguisse atingir um alto desempenho nesses verbos. Depois introduziram uma quantidade maior de verbos, sendo estes menos frequentes que os anteriores mas em sua maioria regulares. Dessa forma, o modelo começou a se ajustar para aplicar a regra regular e assim foi possível observar o comportamento pretendido. Ainda, segundo \cite{pluket:1991}, os estágios de desenvolvimento (i), (ii) e (iii) podem ser considerados parte de um comportamento \textit{macro U-shape}, mas ainda é possível observar a ocorrência de um comportamento \textit{micro U-shape}. \cite{pluket:1991} observam que a reprodução dos verbos irregulares oscila bastante entre flexões corretas e \textit{super-regularizadas}. Eles também notam que estas oscilações ocorrem em proporções diferentes para cada verbo e que crianças raramente “\textit{irregularizam}“ verbos regulares (como \textit{ping}$\rightarrow$\textit{pang}), fato que foi observado no modelo (\textit{gaved}).\\



\\
Pinker e Prince \citeyear{Pinker:1988} apresentam a formulação de uma nova teoria linguística para tal questão: uma teoria híbrida em que a fonologia gerativa se aplica ao processo de flexão regular e um modelo conexionista se aplica ao processo de flexão irregular. Os pesquisadores propõem que as formas regulares sejam computadas a partir de um mecanismo que deve abstrair o radical do verbo e combiná-lo ao sufixo –ed.  Tal mecanismo pode ser aplicado a qualquer palavra, em um processo independente da memória. As formas irregulares, por sua vez, passam por um processo diferente: verbos irregulares precisam passar por um processo de memorização, uma memorização que ocorre de maneira associativa, havendo não somente a associação entre um verbo e outro mas também entre as propriedades (traços fonéticos, rima, stem, núcleo, etc.) de um verbo e de outro, parecido com o que foi proposto por Rumelhart e McClelland.

\section{Motivação}
\label{sec:motivation}

\subsection{Motivação no campo da Linguística}

\subsubsection{Complexidade da Língua Portuguesa}

A morfologia verbal da língua Inglesa é bastante simples, se comparada à Portuguesa. Em primeiro lugar, os verbos do Português se distribuem em três classes módicas (\textit{conjugações}), sendo cada uma destas definida a partir de uma \textit{vogal temática} (\textit{/a/}, \textit{/e/} e \textit{/i/}). Dado um verbo em sua forma infinitiva, por exemplo \textit{Amar}, a vogal temática é a aquela que se encontra entre o morfema lexical do verbo (o radical) e o fone \textit{r}.

\begin{align*}
    \text{Am + a + r}\\
    \text{Radical + VT + r} 
\end{align*}

Com isto, os três possíveis tipos de conjugação são: 1\aup{a} - ar (amar, brigar), 2\aup{a} - er (beber, comer) e 3\aup{a} (rir, descobrir). Na língua Inglesa, essa distinção não existe. 

Outra diferença entre as línguas pode ser observada com relação às pessoais gramaticais das línguas. No Inglês, com exceção do grupo \textit{to be} que apresenta três formas possíveis (considerando o tempo presente): (i) I \textit{am}, (ii) he/she/it \textit{is} e (iii) they \textit{are}; o restante dos verbos apresenta apenas duas: a forma base para \textit{I, We} e \textit{They} (por exemplo, \textit{walk}) e base + s para \textit{he/she/it }(\textit{walks}). No \textit{Past Simple} apresenta maior número de formas também apenas para o grupo \textit{to be}: (i) I/he/she/it \textit{was} e (ii) They/We \textit{were}, os demais não apresentam marcação \cite{Nelson:2010}. 

No Português, a norma tradicional distingue seis pessoas: 1\aup{a}: Eu, 2\aup{a}: Tu, 3\aup{a}: Ele/Ela, 4\aup{a}: Nós, 5\aup{a}: Vós, 6\aup{a}: Eles. Mesmo com a exclusão da 2\aup{a} e da 5\aup{a} pessoa (cujo uso está em decadência (\cite{1999:camara}), o número de formas possíveis para cada verbo é o dobro do número de opções do Inglês.

Com relação às irregularidades, no Inglês os verbos irregulares encontram-se apenas no \textit{Simple Past} e \textit{Past Participle}, enquanto que o sistema verbal do Português é repleto de irregularidades em todos os tempos verbais (\cite{wuerges:2014}).

\subsubsection{Aprendizado de Verbos na Língua Portuguesa}
\label{sec:aprendizado_port}

Uma criança em processo de aquisição de linguagem no sistema do português brasileiro é posta a superar todas as complexidades mencionadas. Uma parte do processo é justamente perceber a relação entre a vogal temática e as possíveis conjugações verbais regulares. Nesse processo não é incomum observarmos o surgimento de trocas de conjugação. \cite{wuerges:2014} apresenta dados linguísticos produzidos por crianças com diversas destas trocas: 


\begin{table}[H]
\begin{center}
\begin{tabular}{cccc}
Verbo & Execução & Objetivo & Troca  \\ \hline
botar & “eu boti“ & botei & 1\aup{a} com 2\aup{a} ou 3\aup{a} \\
comer & “eu comei“ & comi & 2\aup{a} com 1\aup{a} \\
jantar & “eu janti“ & jantei & 1\aup{a} com 2\aup{a} ou 3\aup{a} \\ \hline
& & & 
\end{tabular}
\caption{Exemplos de Trocas de Conjugação Durante o Processo de Aquisição Verbal}
\label{tab:aquisicao}
\end{center}
\end{table}

As formas verbais irregulares apresentam-se como uma dificuldade adicional nesse processo para as crianças falantes da língua portuguesa. \cite{wuerges:2014} também aponta exemplos observados de \textit{regularização} de verbos irregulares: “eu \textit{consego}” (regularização do verbo conseguir) e “eu \textit{podo}” (regularização do verbo poder).

Um verbo é dito irregular se apresentar alterações no radical (em relação ao radical da forma infinitiva) e/ou no sufixo flexional (em relação ao padrão regular imposto por cada conjugação) (\cite{wuerges:2014}). Os sufixos flexionais (SF) são basicamente os segmentos acrescentados após o radical do verbo. Eles podem ser divididos em dois tipos: (i) sufixo modo-temporal (SMT) e (ii) sufixo número-pessoal (SNP). Para o verbo “gostaremos”, por exemplo, considera-se o segmento \textit{/gost/} como o radical do verbo, \textit{/o/} como o sufixo flexional, que neste caso marca simultaneamente modo indicativo, 1\aup{a} pessoa do singular e tempo Presente.


% \cite{1972:camara} define a estrutura dos verbos da seguinte forma:

% \begin{equation}
%     T (R + VT) + SF (SMT + SNP)
% \end{equation}

% Em que T representa o tema do verbo (composto pelo radical (R) e seguido por uma vogal temática (VT)). SF representa o sufixo flexional do verbo, composto pelos sufixos modo-temporal e número-pessoal. Nessa fórmula, leva-se em conta a alomorfia de cada um dos sufixos flexionais e a possibilidade de zero (ø) para um deles ou ambos. Para o verbo “gostaremos”, por exemplo, considera-se o segmento “\textit{gost}” como o radical do verbo, “\textit{a}” como vogal temática  e “\textit{emos}” como o sufixo flexional, que neste caso marca simultaneamente modo indicativo, 1\aup{a} pessoa do plural e tempo Futuro do Presente.

Seguindo a definição proposta, é necessário reforçar que o interesse deste estudo está em capturar irregularidades no nível fonético, portanto verbos como: “gosto”, “boto” e “coloco”, cuja ortografia apresenta o padrão regular; serão classificados como irregulares. A Tabela \ref{tab:irreg} exibe alguns exemplos das classificações realizadas.

\begin{center}
\begin{table}[H]
\centering
\begin{tabular}{ccc}
\multicolumn{1}{l}{\textbf{Verbo Infinitivo}} & \multicolumn{1}{l}{\textbf{Verbo Flexionado}} & \multicolumn{1}{l}{\textbf{Classificação}} \\ \hline
Falar & Falo & Regular \\
Gostar & Gosto & Irregular \\
Testar & Testo & Irregular \\
Ansiar & Anseio & Irregular \\
Pedir & Peço & Irregular \\
Mentir & Minto & Irregular \\
Por & Ponho & Irregular
\end{tabular}
\caption{Exemplos de Classificações de Verbos Quanto a Presença de Irregularidades}
\label{tab:irreg}
\end{table}
\end{center}

Uma análise sobre a disposição das irregularidades presentes no português brasileiro (levando em consideração apenas a 1\aup{a} Pessoa do Singular (tempo Presente e modo Indicativo) nos permite observar algumas regularidades (padrões) dentre os verbos irregulares:\\

\begin{center}

Bobear – Bobeio, Bloquear – Bloqueio, Chatear – Chateio, Clarear – Clareio, Golpear – Golpeio;\\
\\
Agredir – Agrido, Conseguir – Consigo, Inserir – Insiro, Perseguir – Persigo, Preferir – Prefiro, Proferir – Profiro, Repetir – Repito, Servir –  Sirvo, Vestir – Visto;\\
\\
Cobrir – Cubro, Dormir – Durmo, Engolir – Engulo;\\
\\
 Al[e]gar – Al[ε]go, C[e]gar – C[ε]go, Compl[e]tar – Compl[ε]to,  Col[e]tar – Col[ε]to, Entr[e]gar – Entr[ε]go, Pr[e]gar – Pr[ε]go;\\
\\
Ad[o]rar – Ad[\textopeno]ro, Ad[o]tar – Ad[\textopeno]to, B[o]tar – B[\textopeno]to, C[o]lar – C[\textopeno]lo, F[o]car – F[\textopeno]co, M[o]rar – M[\textopeno]ro, S[o]ltar – S[\textopeno]lto, S[o]lar – S[\textopeno]lo, T[o]car – T[\textopeno]co, M[o]strar – M[\textopeno]stro;\\
\\
Mentir - Minto, Sentir - Sintu;

\end{center}



Os padrões observados a partir da exposição de algumas classes irregulares, permitem, assim como no inglês, a proposição de fórmulas, ou regras fonéticas, que explicam as flexões realizadas em cada classe. É possível notar, por exemplo, que um verbo da mesma família de \textit{conseguir} segue a regra:

% Inserir regrinha formal
\begin{center}
e $\rightarrow$ i/\_C*]ir 
\end{center}

A regra proposta indica que /e/ se transforma em /i/ quando em um contexto de terceira conjugação (ir). No caso, C* indica uma sequência de consoantes. 

As previsibilidades encontradas sugerem não somente a possibilidade de elaboração de regras, como também a possibilidade do desenvolvimento de redes capazes de capturar tais dependências. 

\subsubsection{Delimitação de Escopo}

Dada a maior complexidade do Português e a existência de irregularidades verbais em múltiplos paradigmas de tempos, pessoas e modos, o presente estudo se limitará a estudar apenas os padrões irregulares encontrados no paradigma da 1\aup{a} Pessoa do Singular no tempo Presente e modo Indicativo (com exemplos já explorados na Seção \ref{sec:aprendizado_port}). Desse modo, verbos que apresentem irregularidade em outro tempo, modo ou pessoa que não 1\aup{a} pessoa do singular no tempo presente e modo indicativo, serão tratados como pertencentes à classe dos regulares. Como exemplo, considere o verbo \textit{correr}. Esse verbo apresenta flexão regular para a 1\aup{a} Pessoa (\textit{corro}), mas é irregular para a 3\aup{a} Pessoa (\textit{corre}). Desse modo, apesar de \textit{correr} ser um verbo irregular, por apresentar flexão regular no paradigma escolhido, será tratado como regular para os fins dessa pesquisa.

Mesmo com a delimitação escolhida, podemos dizer que a complexidade da tarefa ainda é maior do que o exercício de aprendizado dos verbos irregulares do passado do inglês. 

 


% \subsubsection{Verbos Irregulares}

% %Em princípio, a «irregularidade» pode-se referir ao sufixo flexional, como vimos em nota ao capítulo anterior para SNP = -des, em credes, ledes, etc. Muito mais relevante há a mudança no radical, que passa a contribuir para as noções gramaticais de modo-tempo e número-pessoa. A mudança no radical é que é verdadeiramente importante e cria uma série de padrões morfológicos verbais, que vamos apreciar no presente capítulo. pg 111 camara jr
% Como foi introduzido no Capítulo \ref{ch:01}, o cerne desta pesquisa está na construção de um modelo de redes neurais que consiga capturar os processos flexionais dos verbos irregulares do português brasileiro. Para tanto, fazem-se necessárias definições e delimitações a respeito do objeto de estudo para que se possa construir uma representação vetorial que permita ao modelo capturar os padrões esperados.

% A primeira definição que deve ser feita diz respeito à escolha do paradigma conjugacional de estudo. Diferentemente do inglês que apresenta irregularidades apenas no \textit{Simple Past}, os verbos irregulares do português brasileiro se distribuem livremente entre diferentes tempos, modos e pessoas. Como observado no capítulo introdutório, o paradigma conjugacional de 1\aup{a} pessoa do singular no tempo presente e modo indicativo apresenta uma diversidade de verbos irregulares que podem ser agrupados em classes de acordo com os mesmos processos flexionais e foi portanto escolhido como o paradigma conjugacional desta pesquisa.
% Feita essa escolha, resta definir o que será considerado como verbo irregular dentro desse escopo. 


%Outra dificuldade é ter de lidar com o fato de que os verbos irregulares no português apresentam, em pelo menos uma forma verbal de seu paradigma, alterações no radical e/ou na sua desinência. Isto fica evidente quando observamos a enunciação de formas como: “eu consego*” ou “eu podo*” (poder). É interessante também notar enunciações criativas para verbos de natureza um pouco mais complicada, como o verbo por:  puso* (eu), ponhei* (eu) (\cite{wuerges:2014}).

% Uma análise sobre a disposição das irregularidades presentes no português brasileiro (levando em consideração apenas a 1\aup{a} pessoa do singular (tempo presente - modo indicativo) nos permite observar algumas regularidades (padrões) dentre os verbos irregulares:\\

% \begin{center}

% Bobear – Bobeio, Bloquear – Bloqueio, Chatear – Chateio, Clarear – Clareio, Golpear – Golpeio;\\

% Agredir – Agrido, Conseguir – Consigo, Inserir – Insiro, Perseguir – Persigo, Preferir – Prefiro, Proferir – Profiro, Repetir – Repito, Servir –  Sirvo, Vestir – Visto;\\

% Cobrir – Cubro, Dormir – Durmo, Engolir – Engulo;\\

%  Al[e]gar – Al[ε]go, C[e]gar – C[ε]go, Compl[e]tar – Compl[ε]to,  Col[e]tar – Col[ε]to, Entr[e]gar – Entr[ε]go, Pr[e]gar – Pr[ε]go;\\

% Ad[o]rar – Ad[\textopeno]ro, Ad[o]tar – Ad[\textopeno]to, B[o]tar – B[\textopeno]to, C[o]lar – C[\textopeno]lo, F[o]car – F[\textopeno]co, M[o]rar – M[\textopeno]ro, S[o]ltar – S[\textopeno]lto, S[o]lar – S[\textopeno]lo, T[o]car – T[\textopeno]co, M[o]strar – M[\textopeno]stro;\\

% Mentir - Minto, Sentir - Sintu;

% \end{center}

% Os padrões observados a partir da exposição de algumas classes irregulares, permitem, assim como no inglês, a proposição de fórmulas, ou regras fonéticas, que explicam as flexões realizadas em cada classe. É possível notar, por exemplo, que um verbo da mesma família de \textit{conseguir} segue a regra:

% % Inserir regrinha formal
% \begin{center}
% e $\rightarrow$ i/\_C*]ir \footnote{C* indica uma sequência de uma ou mais consoantes}
% \end{center}


% As previsibilidades encontradas sugerem não somente a possibilidade de elaboração das regras, como também a possibilidade do desenvolvimento de redes capazes de capturar tais dependências. Apesar disso, deve-se destacar que a língua portuguesa apresenta um comportamento diferente da língua inglesa. %Uma análise foi realizada a partir de uma base de verbos irregulares do inglês em notação fonética (retirados de \url{https://www.apronus.com/learn_english/irregularverbs.htm}) e concluiu-se que a média de número de fonemas por verbo é de 4.16 fonemas. Em contrapartida, construiu-se uma base de verbos da línguas portuguesa e concluiu-se que a média de número de fonemas por verbo é de 6.39 fonemas.

\subsection{Motivação no campo da Computação}

Desde a apresentação da pesquisa de \cite{rumelhart:1986}, o modelo associativo utilizado pelos autores já passou por diversos avanços. Na realidade, essa modelagem hoje é chamada de Rede Neural Artificial (ou também \textit{Deep Learning}) e passou a ser utilizada em uma variedade de tarefas computacionais, como classificação de imagens, classificação de texto, tradução automática, agentes conversacionais, entre outros. 

Ao longo dos últimos anos, o poder computacional foi aumentando, e com isso arquiteturas mais robustas foram desenvolvidas. Inclusive, o nome \textit{Deep Learning} surgiu em decorrência da adição de camadas intermediárias de nódulos entre as camadas de \textit{input} e \textit{output}. Tal incremento possibilita que as informações de entrada sejam distribuídas ao longo de mais conexões e com isso, o aprendizado se torna mais \textit{profundo}. Isso acontece pois o tipo de arquitetura sem camadas intermediárias consegue encontrar apenas funções com fronteiras de decisão lineares. Ao aumentar o número de camadas intermediárias, é possível ampliar o universo de soluções para a resolução de problemas mais complexos. Esse tipo de modelagem que segue um fluxo com um sentido único (do \textit{input} ao \textit{output}) é chamado de \textit{Feedforward}. Entretanto, existem muitos tipos de arquiteturas cujos fluxos não seguem essa configuração. Nesse âmbito, um tipo de arquitetura que ficou muito famoso é o \textit{Convolucional} (\textit{Convolutional Neural Networks} - CNN's) (bastante utilizado na área de visão computacional (\cite{Krizhevsky:2012}, por exemplo). No campo de tarefas linguísticas, as redes do tipo \textit{Recorrente} (\textit{Recurrent Neural Networks} - RNN's) são bastante utilizadas, uma vez que a arquitetura viabiliza a ingestão de dados sequenciais  (\cite{pengfei:2016}, por exemplo). O tema dos modelos de Redes Neurais, em especial a arquitetura de RNN's, será abordado com mais detalhes no Capítulo \ref{ch:03}.

No que diz respeito à questão do aprendizado dos verbos do \textit{Past Simple} do Inglês, uma série de novos experimentos deram sequência após as críticas de (\cite{Pinker:1988}). (\cite{pluket:1991}, \citeyear{PLUNKETT:1993}) simplificam a questão ao considerar apenas verbos de tamanho fixo (3 sílabas) e abordam o problema fazendo uso de uma arquitetura com adição de camadas intermediárias (\textit{Multi-Layered Perceptron - MLP}). Outros trabalhos transformaram a questão em um problema de \textit{classificação}, desse modo o modelo não teria mais como objetivo encontrar uma forma flexionada. Ao invés disso, teriam um conjunto finito e pré determinado de formas possíveis. \cite{Nakisa1996WhereDD}, por exemplo, classificam os plurais dos substantivos da língua Alemã. \cite{plunkett:1997} atacam o mesmo problema, porém para a língua Árabe. \cite{wetermann:1997} apresentam um modelo construído para mapear verbos não flexionados da língua Alemã para a forma no particípio. O modelo apresentado é capaz de lidar com dados de sequência variável e utiliza uma arquitetura baseada em Redes Neurais Recorrentes. Entretanto, o modelo foi construído a partir de um mecanismo de rota dupla, de modo que verbos irregulares passavam por uma rota específica de memorização. Apesar disso, o modelo falhou em capturar completamente os padrões irregulares. Segundo os autores, a explicação pode estar no fato de terem utilizado uma arquitetura de Rede Neural Recorrente muito simples, cujo treinamento apresentou problemas durante a atualização das conexões. Uma arquitetura RNN mais robusta poderia ter apresentad  


. Hoje, com o avanço nas arquiteturas e acréscimo de camadas intermediárias (entre as camadas de \textit{input} e \textit{output}, também são conhecidos como modelos de \textit{Deep Learning} (\cite{Goodfellow-et-al-2016}). Os modelos de redes neurais 
As melhorias nos poderes computacionais em conjunto com o acesso a grandes quantidades de dados impulsionaram as pesquisas na área e com isso o surgimento de novas arquiteturas. Entre elas, destaca-se o modelo Encoder-Decoder (\cite{enc-dec:2014}). O Encoder-Decoder é uma arquitetura especialmente conhecida pelo seu bom desempenho em tarefas, em especial no ramo da tradução automática. Sendo assim, parte da motivação desta pesquisa consiste na avaliação desse modelo na tarefa de aprendizado de flexão de verbos irregulares. 


% ------------------------------------------------------------------------
\section{Objetivo}
\label{sec:objectives}

Este trabalho tem como principal objetivo estudar a questão de aprendizado de verbos irregulares do Português Brasileiro através de um novo tipo de modelagem de rede neural, o Encoder-Decoder. 

Por se tratar de um trabalho de caráter experimental no campo da Linguística Computacional, em nenhum momento se prestará à defesa de uma corrente teórica em aquisição, porém apresenta novos resultados que poderão fomentar futuras discussões sobre o assunto. 

O desenvolvimento dessa pesquisa também resultou na construção de um Corpus que compõe 423 verbos flexionados na primeira pessoa do singular do modo indicativo, suas respectivas formas no infinitivo e suas respectivas transcrições fonéticas\footnote{Realizadas com base na chave de transcrição a ser apresentada no Capítulo \ref{ch:02}}.


% ------------------------------------------------------------------------
\section{Organização}
\label{sec:organization}

O Capítulo \ref{ch:02} expõe o embasamento teórico linguístico necessário para a construção dos dados de treinamento do modelo. 

O Capítulo \ref{ch:03} apresenta uma introdução aos modelos de redes neurais e também introduz os conceitos de Modelo de Linguagem e Redes Neurais Recorrentes - conceitos imprescindíveis para o entendimento do modelo final desenvolvido.

O Capítulo \ref{ch:05} exibe a arquitetura conhecida como \textit{Encoder-Decoder} e em seguida a aplicação do mesmo para a questão do aprendizado de flexão dos verbos. Também apresenta uma seção dedicada à formação do Corpus utilizado.

O Capítulo \ref{ch:07} apresenta os resultados obtidos pelo modelo.

O Capítulo \ref{ch:08} expõe uma discussão sobre os resultados obtidos e as dificuldades encontradas. Também destaca sugestões para pesquisas futuras sobre o assunto.

No Apêndice encontram-se disponíveis todas as predições do modelo e o Corpus completo utilizado.


\chapter{Pré-Processamento dos Verbos para Redes Neurais}
\label{ch:02}

\section{Verbos Irregulares}

Como foi introduzido no Capítulo \ref{ch:01}, o cerne desta pesquisa está na construção de um modelo de redes neurais que consiga capturar os processos flexionais dos verbos irregulares do português brasileiro. Para tanto, fazem-se necessárias definições e delimitações a respeito do objeto de estudo para que se possa construir uma representação vetorial que permita ao modelo capturar os padrões esperados.

A primeira definição que deve ser feita diz respeito à escolha do paradigma conjugacional de estudo. Diferentemente do inglês que apresenta irregularidades apenas no \textit{Simple Past}, os verbos irregulares do português brasileiro se distribuem livremente entre diferentes tempos, modos e pessoas. Como observado no capítulo introdutório, o paradigma conjugacional de 1\aup{a} pessoa do singular no tempo presente e modo indicativo apresenta uma diversidade de verbos irregulares que podem ser agrupados em classes de acordo com os mesmos processos flexionais e foi portanto escolhido como o paradigma conjugacional desta pesquisa.
Feita essa escolha, resta definir o que será considerado como verbo irregular dentro desse escopo. 

\cite{1972:camara} define os verbos regulares como os verbos que apresentam o seguinte padrão:

\begin{equation}
    T (R + VT) + SF (SMT + SNP)
\end{equation}

Em que T representa o tema do verbo (composto pelo radical (R) e seguido por uma vogal temática (VT)). SF representa o sufixo flexional do verbo, composto pelos sufixos modo-temporal e número-pessoal. Por essa definição, serão considerados verbos irregulares aqueles cujas formas flexionadas não apresentem o padrão regular proposto. Ainda, entende-se que verbos que apresentem irregularidade em outro tempo, modo ou pessoa que não 1\aup{a} pessoa do singular no tempo presente e modo indicativo, serão tratados como pertencentes à classe dos regulares. Também é necessário reforçar que o interesse deste estudo está em capturar irregularidades no nível fonético, portanto verbos como "gosto", "boto", "coloco"; serão classificados como irregulares. A Tabela \ref{tab:irreg} exibe alguns exemplos das classificações realizadas.

\begin{center}
\begin{table}[H]
\centering
\begin{tabular}{ccc}
\multicolumn{1}{l}{\textbf{Verbo Infinitivo}} & \multicolumn{1}{l}{\textbf{Verbo Flexionado}} & \multicolumn{1}{l}{\textbf{Classificação}} \\ \hline
Falar & Falo & Regular \\
Gostar & Gosto & Irregular \\
Testar & Testo & Irregular \\
Ansiar & Anseio & Irregular \\
Pedir & Peço & Irregular \\
Mentir & Minto & Irregular \\
Por & Ponho & Irregular
\end{tabular}
\caption{Exemplos de Classificações de Verbos Quanto a Presença de Irregularidades}
\label{tab:irreg}
\end{table}
\end{center}

\section{Representações Fonéticas}
\label{sec:transcr}
Na década de 60, quando foi realizado o experimento dos pesquisadores Rumelhart e McClelland, o poder computacional para a construção de modelos de redes neurais era muito baixo e forçou os pesquisadores a desenvolverem um sistema de codificação composto de várias etapas. De maneira simplificada, o sistema apresentado consistia, primeiramente, na transformação dos verbos (sequências de fones) em vetores de trigramas de fones somados a uma demarcação de início e final de palavra, representada pelo símbolo \#. Em seguida, cada um dos fones era substituído por uma lista de traços distintivos %\ref{} colocar aqui a referencia do chomsky
. Ao final, todas as possíveis combinações de três traços distintivos subsequentes foram mapeados como chaves em um dicionário de tamanho fixo em que os valores poderiam assumir somente 0 ou 1; 1 caso aquela sequência de traços estivesse presente no verbo e 0 caso contrário. A Figura \ref{fig:wick} ilustra o esquema de codificação para o verbo "came" ([kAm] segundo a chave de transcrição apresentada pelos pesquisadores). 

\input{definitions/colors}
\input{definitions/styles}
\begin{figure}[ht!]
\centering

\scalebox{1.0}{
\begin{tikzpicture}[H]

%vetor
\node[box2] (box1) {};
\node[box2, below=0pt of box1] (box2) {};
\node[box3, below=0pt of box2] (box3) {};
\node[box3, below=0pt of box3] (box4) {};
\node[textonly, below=0pt of box4] (box5) {\reflectbox{$\vdots$}};
\node[box3, below=0pt of box5] (box6) {};
\node[box3, below=0pt of box6] (box7) {};
\node[box2, below=0pt of box7] (box8) {};
\node[box3, below=0pt of box8] (box9) {};
\node[box2, below=0pt of box9] (box10) {};
\node[textonly, below=0pt of box10] (dim) {460x1};
\node[textonly, below=0pt of dim] (space) {};
%wickelfeatures
\node[textonly, right=20pt of box1] (wi1) {\#, oclusiva, média};
\node[textonly, right=20pt of box2] (wi2) {\#, anterior, fricativa};
\node[textonly, right=20pt of box3] (wi3) {\#, fricativa, vogal};
\node[textonly, right=20pt of box4] (wi4) {\#, contínua, anterior};
\node[textonly, right=20pt of box5] (wi5) {\reflectbox{$\vdots$}};
\node[textonly, right=20pt of box6] (wi6) {\#, não vozeada, aberta};
\node[textonly, right=20pt of box7] (wi7) {\#, contínua, posterior};
\node[textonly, right=20pt of box8] (wi8) {contínua, vogal, vogal };
\node[textonly, right=20pt of box9] (wi9) {contínua, vogal, líquida};
\node[textonly, right=20pt of box10] (wi10) {fricativa, posterior, \#};

%pré processamento
\node[textonly, left=230pt of box4] (verb1) {falar};
\node[textonly, below=10pt of verb1] (verb2) {fala};
\node[textonly, below=10pt of verb2] (verb3) {\#fala\#};

%trigramas
\node[textonly, right=20pt of verb1] (tri1) {\#,f,a};
\node[textonly, below=10pt of tri1] (tri2) {f,a,l};
\node[textonly, below=10pt of tri2] (tri3) {a,l,a};
\node[textonly, below=10pt of tri3] (tri4) {l,a,\#};

%features
\node[textonly, left=50pt of box2] (f1) {\#,cont,vogal};
\node[textonly, below=0pt of f1] (f2) {\#,cont,aberta};
\node[textonly, below=0pt of f2] (f3) {\#,cont,anterior};
\node[textonly, below=0pt of f3] (f4) {\#,cont,baixa};
\node[textonly, below=0pt of f4] (f5) {\reflectbox{$\vdots$}};
\end{tikzpicture}
}\caption{Esquema de Codificação de Wickelfeatures} 
\label{fig:wick}
\end{figure}

Na época, a dificuldade de se representar uma sequência de itens em um único vetor numérico era muito grande. O modelo apresentou problemas especialmente em casos em que uma mesma sequência de traços distintivos aparecia mais de uma vez no mesmo verbo, visto que o modelo marca apenas presença ou ausência da sequência. Hoje, novas arquiteturas de redes neurais (a ver no Cap \ref{ch:03}) estão disponíveis para lidar com dados sequenciais com a utilização de representações vetoriais menos complicadas. 

Uma possível escolha para tal representação é o tradicional Alfabeto Fonético Internacional (AFI), apresentado nas Tabelas \ref{tab:ipa1} e \ref{tab:ipa2}. O AFI é um sistema de notação fonética, criado pela Associação Fonética Internacoional para promover uma padronização na transcrição de dados de diferentes idiomas. Ele organiza símbolos que representam unidades sonoras presentes nas línguas humanas a partir de características de execução dessas unidades. A Tabela \ref{tab:ipa1} reune o conjunto do sons consoantes e exibe na dimensão das colunas o ponto de articulação dos sons. As linhas compõe os diferentes modos de articulação possíveis. Por fim, dentro de uma mesma célula pode ocorrer um som com ou sem a vibração das cordas vocais, é o caso do par \textbf{[p b]}. O símbolo da esquerda representa o som surdo e o símbolo da direita, sonoro. As vogais estão organizadas na Tabela \ref{tab:ipa2}. As colunas dessa tabela se referem ao local de reprodução dos sons e as linhas à abertura da boca durante a execução. Quando os símbolos aparecem em pares, aquele da direita representa uma vogal arredondada. %\cite{para conhecer} 
\begin{center}
\scalebox{0.8}{
    \begin{tabular}{|l|cc|cc|cc|cc|cc|cc|cc|cc|cc|cc|cc|}
%\begin{tabular}{|l|cc|}
        \hline & 
            \multicolumn{2}{|c|}{\footnotesize{Bilabial}} &					% Bilabial
            \multicolumn{2}{|c|}{\footnotesize{Lab. dent.}} & 			% Labiodental
            \multicolumn{2}{|c|}{\footnotesize{Dental}} & 					% Dental
            \multicolumn{2}{|c|}{\footnotesize{Alveolar}} & 				% Alveolar
            \multicolumn{2}{|c|}{\footnotesize{P-alveo.}} & 		% Post-alveolar
            \multicolumn{2}{|c|}{\footnotesize{Retroflex}} & 				% Retroflex
            \multicolumn{2}{|c|}{\footnotesize{Palatal}} & 					% Palatal
            \multicolumn{2}{|c|}{\footnotesize{Velar}} & 					% Velar
            \multicolumn{2}{|c|}{\footnotesize{Uvular}} & 					% Uvular
            \multicolumn{2}{|c|}{\footnotesize{Pharyng.}} & 			% Pharyngeal
            \multicolumn{2}{|c|}{\footnotesize{Glottal}}  \\% Glottal

        \hline 
        Plosive &	% Plosive
            p & b &													% Bilabial
            &&														% Labiodental
            \multicolumn{3}{|r}{t}&							% Dental
            \multicolumn{3}{l|}{d}&							% Alveolar
                                                                        % Post-alveolar
            \ipa{\:t} & \ipa{\:d}&									% Retroflex
            c & \textbardotlessj &														% Palatal
            k & g &													% Velar
            q & \ipa{\;G} &										% Uvular
            & \BlankCell        &								% Pharyngeal
            \ipa{P}& \BlankCell         \\								% Glottal

        \hline Nasal & 							% Nasal
            & m &													% Bilabial
            & \ipa{M} &											% Labiodental
            \multicolumn{3}{|r}{}&								% Dental
            \multicolumn{3}{l|}{n}&							% Alveolar
                                                                        % Post-alveolar
            & \ipa{\:n} &														% Retroflex
            & \textltailn &														% Palatal
            & \ipa{N} &														% Velar
            & N &														% Uvular
            \BlankCell        & \BlankCell        &		% Pharyngeal
            \BlankCell        & \BlankCell         \\		% Glottal

        \hline Trill &  								% Trill
            & \ipa{\;B}&											% Bilabial
            & &														% Labiodental
            \multicolumn{3}{|r}{}&								% Dental
            \multicolumn{3}{l|}{r}&								% Alveolar
                                                                        % Post-alveolar
            & &														% Retroflex
            & &														% Palatal
            \BlankCell        & \BlankCell        &		% Velar
            & \ipa{\;R}&											% Uvular
            & &														% Pharyngeal
            \BlankCell        & \BlankCell         \\		% Glottal

        \hline Tap/Flap &  						% Tap /Flap
            & &													% Bilabial
            & &														% Labiodental
            \multicolumn{3}{|r}{} &					% Dental
            \multicolumn{3}{l|}{\ipa{R}} &					% Alveolar
                                                                        % Post-alveolar
            & \ipa{\:r} &														% Retroflex
            & &														% Palatal
            \BlankCell        & \BlankCell        &		% Velar
            & &														% Uvular
            & &														% Pharyngeal
            \BlankCell        & \BlankCell         \\		% Glottal

        \hline Fricative & 						% Fricative
            \ipa{F} & \ipa{B} &									% Bilabial
            f & v &													% Labiodental
            \ipa{T} & \ipa{D} &									% Dental
            s & z &													% Alveolar
            \ipa{S} & \ipa{Z} &									% Post-alveolar
            \ipa{\:s} & \ipa{\:z} &								% Retroflex
            \ipa{\c{c}} & \ipa{J} &								% Palatal
            x & \ipa{G} &											% Velar
            \ipa{X} & \ipa{K} &									% Uvular
            \textcrh & \ipa{Q} &								% Pharyngeal
            h & \texthth \\										% Glottal

        \hline Lat. Fric. & 					% Lat. Fricative
            \BlankCell        & \BlankCell        &		% Bilabial
            \BlankCell        & \BlankCell        &		% Labiodental
            \multicolumn{3}{|r}{\textbeltl} &				% Dental
            \multicolumn{3}{l|}{\textlyoghlig} &			% Alveolar
                                                                        % Post-alveolar
            & &														% Retroflex
            & &														% Palatal
            & &														% Velar
            & &														% Uvular
            \BlankCell        & \BlankCell        			% Pharyngeal
            & \BlankCell        & \BlankCell         \\   % Glottal

        \hline Approx & 							% Approx.
            & &														% Bilabial
            & \ipa{V} &											% Labiodental
            \multicolumn{3}{|r}{}&								% Dental
            \multicolumn{3}{l|}{\ipa{\*r}} &					% Alveolar
                                                                        % Post-alveolar
            & \ipa{\:R} &											% Retroflex
            & j &														% Palatal
            & \textturnmrleg &									% Velar
            & &														% Uvular
            & &														% Pharyngeal
            \BlankCell        & \BlankCell         \\		% Glottal

        \hline Lat. appr. & 					% Lat. Approx
            \BlankCell        & \BlankCell        &		% Bilabial
            \BlankCell        & \BlankCell        &		% Labiodental
            \multicolumn{3}{|r}{}&								% Dental
            \multicolumn{3}{l|}{l}&								% Alveolar
                                                                        % Post-alveolar
            & \textipa{\:l} &											% Retroflex
            & \textipa{L} &												% Palatal
            & \textipa{\;L} &											% Velar
            & &														% Uvular
            \BlankCell        & \BlankCell        &		% Pharyngeal
            \BlankCell        & \BlankCell         \\		% Glottal
        \hline
    \end{tabular}
}%scalebox
\captionof{table}{Consoantes AFI}\label{tab:ipa1}
\end{center}

\begin{center}
    \begin{vowel}
        %    \putcvowel[l]{i}{1}
        \putvowel[l]{i}{0pt}{0pt}
        \putcvowel[r]{y}{1}
        \putcvowel[l]{e}{2}
        \putcvowel[r]{\o}{2}
        \putcvowel[l]{\textepsilon}{3}
        \putcvowel[r]{\oe}{3}
        \putcvowel[l]{a}{4}
        \putcvowel[r]{\textscoelig}{4}
        \putcvowel[l]{\textscripta}{5}
        \putcvowel[r]{\textturnscripta}{5}
        \putcvowel[l]{\textturnv}{6}
        \putcvowel[r]{\textopeno}{6}
        \putcvowel[l]{\textramshorns}{7}
        \putcvowel[r]{o}{7}
        \putcvowel[l]{\textturnm}{8}
        \putcvowel[r]{u}{8}
        \putcvowel[l]{\textbari}{9}
        \putcvowel[r]{\textbaru}{9}
        \putcvowel[l]{\textreve}{10}
        \putcvowel[r]{\textbaro}{10}
        \putcvowel{\textschwa}{11}
        \putcvowel[l]{\textrevepsilon}{12}
        \putcvowel[r]{\textcloserevepsilon}{12}
        \putcvowel{\textsci\ \textscy}{13}
        \putcvowel{\textupsilon}{14}
        \putcvowel{\textturna}{15}
        \putcvowel{\ae}{16}
    \end{vowel}
\captionof{table}{Vogais IPA}\label{tab:ipa2}   
\end{center} 

O AFI poderia ter sido utilizado para a construção das representações vetoriais, mas apenas um subconjunto do mesmo seria aproveitado já que o português não contém todos os fones possíveis. Desse modo, optou-se por uma simplificação do AFI que resultou na Tabela \ref{tab:new_rep}. Nessa nova representação, além da exclusão de alguns pontos e modos de articulação, apresenta-se também uma chave de transcrição alternativa que engloba apenas caracteres pertencentes ao código ASCII. A opção pela realização dessas modificações se deu em razão de facilitar o pré-processamento do corpus uma vez que o código fonético não é interpretável pela maioria das linguagens de programação.






\begin{center}
\scalebox{0.9}{
    \begin{tabular}{|l|cc|cc|cc|cc|cc|cc|}
        \hline & 
            \multicolumn{2}{|c|}{\footnotesize{Bilabial}} &					% Bilabial
            \multicolumn{2}{|c|}{\footnotesize{Lab. dent.}} & 			% Labiodental
            \multicolumn{2}{|c|}{\footnotesize{Alveolar}} & 				% Alveolar
            \multicolumn{2}{|c|}{\footnotesize{P-alveo.}} & 		% Post-alveolar

            \multicolumn{2}{|c|}{\footnotesize{Velar}} & 					% Velar
            \multicolumn{2}{|c|}{\footnotesize{Glottal}}  \\					% Glottal

        \hline Plosive &  							% Plosive
            p & b &	% Bilabial
            &&	% Labiodental
            t & d	% Alveolar
            & &% Post-alveolar
            & k & g 											& 		% Velar

            & \BlankCell         \\								% Glottal

        \hline Nasal & 							% Nasal
            & m 	% Bilabial
            &  &  & % Labiodental
            & n 	% Alveolar
            & & % Post-alveolar
            & & N 	% Velar
            &  &          \\		% Glottal

        \hline Tap/Flap &  						% Tap /Flap
            &													% Bilabial
            & &														% Labiodental
           && r &					% Alveolar
            &&                 % Post-alveolar

            \BlankCell        & \BlankCell        &		% Velar
            \BlankCell        & \BlankCell         \\		% Glottal

        \hline Fricative & 						% Fricative
            &  &									% Bilabial
            f & v &													% Labiodental
            s & z &													% Alveolar
            x & j &									% Post-alveolar
            &  &% Velar
            h &  \\										% Glottal

        \hline Lat. appr. & 					% Lat. Approx
            \BlankCell        & \BlankCell        &		% Bilabial
            \BlankCell        & \BlankCell        &		% Labiodental

           & l &							% Alveolar
                                                      &&                  % Post-alveolar
  
             & L & 												% Palatal
            											% Velar
            \BlankCell        & \BlankCell         \\		% Glottal
        \hline
    \end{tabular}
}%scalebox
\captionof{table}{Consoantes na nova representação}\label{tab:new_rep}
\end{center}

\begin{center}
\begin{table}[H]
\begin{center}
    \begin{tabular}{lll}
        \hline
         & Anterior & Posterior \\
         \hline
        Fechada & i & u \\
        \hline
        Meia-fechada & e & o \\
        \hline
        Meia-aberta & E & O \\
        \hline
        Aberta & a &  \\
        \hline
        Nasais & A (ã) &\\ & 3 ($\tilde{e}$) &\\ 
        \hline
    \end{tabular}
\end{center}
\caption{Vogais na nova representação}
\label{tab:new_vocals}
\end{table}
\end{center}

Além de tais modificações, também não foram incluídas representações para variantes, portanto os sons de \textbf{[x]-[h]-[r]} (variantes possíveis na pronúncia da palavra 'te\textbf{rr}a') foram resumidos apenas a \textbf{[h]}. Da mesma forma, também não há um símbolo equivalente para \textbf{[\ipa{S}]}, som empregado normalmente entre \textbf{t} e \textbf{i}, como na palavra "time". A representação para as vogais foi simplicada também para facilitar as transcrições e restringir o número de traços fonéticos possíveis no modelo. A Tabela \ref{tab:chave} mostra a chave de transcrição proposta para as consoantes e vogais. A Tabela \ref{tab:transc} exibe alguns exemplos de transcrição na nova representação sugerida. O Corpus completo ecnontra-se no Apêndice.

\begin{table}[H]
\begin{center}
\begin{tabular}{cc}
\hline
\textbf{Verbo} & \textbf{Transcrição} \\ \hline
ressentir & hes3ntir \\
paro & paru \\
possuo & posuu \\
olha & oLa \\
sacudir & sakudir \\
voltar & voutar \\ \hline
\end{tabular}
\end{center}
\caption{Exemplos de Transcrições}
\label{tab:transc}
\end{table}

\begin{center}
\begin{table}[H]
\begin{center}

\end{center}
\caption{Vogais na nova representação}
\label{tab:transc}
\end{table}
\end{center}


\begin{table}[H]
\begin{center}
\begin{tabular}{lc}
\textbf{AFI} & \multicolumn{1}{l}{\textbf{Transcrição Proposta}} \\ \hline

\textbf{{[}p{]}} - \textbf{p}arar & p \\
\textbf{{[}b{]}} - \textbf{b}otar & b \\
\textbf{{[}t{]}} - \textbf{t}ocar & t \\
\textbf{{[}d{]}} - \textbf{d}ançar & d \\
\textbf{{[}k{]}} - \textbf{c}asar & k \\
\textbf{{[}g{]}} - \textbf{g}ostar & g \\
\textbf{{[}f{]}} - \textbf{f}ugir & f \\
\textbf{{[}v{]}} - \textbf{v}oltar & v \\
\textbf{{[}s{]}} - \textbf{s}oltar & s \\
\textbf{{[}z{]}} - pre\textbf{s}enciar & z \\
\textbf{{[}\ipa{S}{]}} - \textbf{ch}amar & x \\
\textbf{{[}\ipa{Z}{]}} - \textbf{j}antar & j \\
\textbf{{[}t\ipa{S}{]}} - sen\textbf{t}ir & t \\
\textbf{{[}d\ipa{Z}{]}} - \textbf{d}izer & d \\
\textbf{{[}h{]}} - e\textbf{rr}ar & h \\
\textbf{{[}\ipa{\:r}{]}} - enca\textbf{r}ar & r \\
\textbf{{[}l{]}} - pu\textbf{l}ar & l \\
\textbf{{[}\textipa{L}{]}} - espa\textbf{lh}ar & L \\
\textbf{{[}m{]}} - \textbf{m}orar & m \\
\textbf{{[}n{]} }- \textbf{n}adar & n \\
\textbf{{[}\ipa{N}{]}} - so\textbf{nh}ar & N\\
\textbf{{[}a{]}} - p\textbf{a}rar & a \\
\textbf{{[}e{]}} - l\textbf{e}r & e \\
\textbf{{[}\textepsilon{]}} - esp\textbf{e}ro & E \\
\textbf{{[}i{]}} - r\textbf{i}r & i \\
\textbf{{[}o{]}} - pr\textbf{o}por & o \\
\textbf{{[}\textopeno{]} }- col\textbf{o}co & O \\
\textbf{{[}u{]}} - c\textbf{u}rtir & u \\
\textbf{{[}\~e{]}} - \textbf{e}ntreter & 3 \\
\textbf{{[}ã{]}} - pl\textbf{a}ntar & A \\
\textbf{{[}\~o{]}} - comp\textbf{o}nho & o \\
\textbf{{[}\textupsilon{]}} - cas\textbf{o} & u \\
\textbf{{[}j{]}} - sa\textbf{i}o & i \\
\textbf{{[}w{]}} - vo\textbf{l}to & u
\end{tabular}
\end{center}
\caption{Chave de Transcrição Proposta}
\label{tab:chave}
\end{table}





\chapter{Modelos de Redes Neurais}
\label{ch:03}


% \epigraph{\itshape Those who cannot remember the past are condemned to compute it.''}{---Steven Pinker, \textit{Words and Rules}}

Este capítulo serve como uma introdução à teoria das redes neurais. Os conceitos aqui introduzidos servirão como embasamento teórico imprescindível para a apresentação do modelo final utilizado nesta pesquisa, o \textit{Encoder-Decoder}. 

\section{Introdução a Redes Neurais}

Um modelo de rede neural é, essencialmente, um modelo de aprendizado de máquina supervisionado %[ref] 
que está a procura de identificar padrões. Um modelo de aprendizado de máquina é uma tarefa computacional que explora algoritmos que podem aprender a partir de seus erros e fazer previsões sobre dados. Esse tipo de algoritmo é normalmente inicializado com nenhuma expectativa sobre a tarefa que deve realizar e busca informações exclusivamente a partir dos dados do problema. Os possíveis tipos de aprendizado de máquina são divididos entre: \textit{supervisionados}, \textit{não-supervisionados}
e por \textit{reforço}. (ref)
No caso das redes neurais, o aprendizado diz-se supervisionado, pois informa-se ao modelo o \textit{target} esperado  pelo treinamento (o \textit{alvo}) para cada informação de entrada (cada \textit{input}).

A inspiração para o desenvolvimento da modelagem em redes neurais artificiais surgiu a partir de estudos em neurosciência %[ref] 
que concluíram que, diante de múltiplas apresentações de um mesmo estímulo, um mesmo conjunto de neurônios sofre incitação e dispara. Um estímulo diferente resulta no disparo de um conjunto de neurônios diferente (\cite{hubel:1962}).  Analogamente, o modelo artificial é composto por uma camada denominada de \textit{input} responsável por receber diferentes estímulos (matematicamente retratados por vetores numéricos que representam o objeto do aprendizado). A informação recebida é distribuída ao longo de múltiplas conexões com a próxima camada através de uma multiplicação com uma matriz de pesos $\vect{W}$. A matriz de pesos funciona como uma analogia às conexões existentes entre neurônios de modo que um peso maior representa uma conexão reforçada e um peso menor representa uma conexão reprimida. Além disso, é importante que o sistema de aprendizado não seja demasiado sensível a todo \textit{input} que receber, pois nesse caso cada \textit{input} diferente recebido alteraria completamente o modelo impossibilitando um aprendizado generalizado. Como solução, o resultado obtido a partir da multiplicação dos pesos pelos \textit{inputs} entra como argumento em uma função, que nesse contexto é chamada de \textit{função de ativação}. A função de ativação é uma função que simula o potencial energético existente entre as conexões neurais e permite que uma unidade apenas seja ativada caso o resultado dessa função atinja um limiar mínimo. Isso permite ao sistema a produção de diferentes respostas para padrões diferentes utilizando a mesma rede e além disso, permite que o aprendizado ocorra de forma gradual, de modo que o efeito de estímulos passados ainda perdure por um longo período mesmo após a apresentação de novos estímulos. Uma das funções de ativação mais utilizadas na literatura (e inclusive utilizada pelos pesquisadores Rumelhart e McClelland) é a função \textit{Sigmoid}, uma função suave, diferenciável e facilmente interpretável. A Fig. \ref{fig:sigmoidplot} ilustra o processo de ativação dado um \textit{input} $\vect{x}$. Supõe-se que um neurônio seja ativado apenas se houver uma energia mínima para tal. Da mesma forma, representa-se uma ativação no eixo \textit{y} (com valores entre 0 (não ativação) e 1 (ativação)) de modo que apenas valores mais altos de $\vect{x}$ atingem valores próximos da ativação em \textit{y}.

\begin{figure}[H]
\centering
\scalebox{0.9}{
\begin{tikzpicture}
    \begin{axis}[
    	legend pos=north west,
        axis x line=middle,
        axis y line=middle,
        x tick label style={/pgf/number format/fixed,
                            /pgf/number format/fixed zerofill,
                            /pgf/number format/precision=1},
        y tick label style={/pgf/number format/fixed,
                            /pgf/number format/fixed zerofill,
                            /pgf/number format/precision=1},
        grid = major,
        width=16cm,
        height=8cm,
        grid style={dashed, gray!30},
        xmin=-1,     % start the diagram at this x-coordinate
        xmax= 1,    % end   the diagram at this x-coordinate
        ymin= 0,     % start the diagram at this y-coordinate
        ymax= 1,   % end   the diagram at this y-coordinate
        %axis background/.style={fill=white},
        xlabel=x,
        ylabel=y,
        tick align=outside,
        enlargelimits=false]
      % plot the stirling-formulae
      \addplot[domain=-1:1, black, thick,samples=500] {1/(1+exp(-5*x))};
      %\addplot[domain=-1:1, blue, ultra thick,samples=500] {1/(1+exp(-10*x))};
      \addlegendentry{$f(x)=\frac{1}{1+e^{-ax}}$}
      %\addlegendentry{$g(x)=\frac{1}{1+e^{-10x}}$}
    \end{axis} 
\end{tikzpicture}
}
\caption{A função \textit{Sigmoid}.}
\label{fig:sigmoidplot}
\end{figure} %diminuir esse desenho

Após essa passagem pela função de ativação, o resultado serve como novo input para a próxima camada e assim sucessivamente até a última, a camada de \textit{output}. Todas as camadas existentes entre as camadas de \textit{input} e de \textit{output} são chamadas de \textit{camadas escondidas} (Ver Fig. \ref{fig:ffd}.)

% \begin{align}\label{eq:sigmoid}
% p(w_{i} = 1) = \frac{1}{1+e^{\sum_{i} w_{i}x_{i}}}
% \end{align}

Algebricamente, pode-se representar o processo descrito através da composição de múltiplas funções, uma vez que o resultado das operações precedentes servirão como entrada para as próximas camadas.

\begin{align}
% f(\vect{x}) &= f^{(2)}(f^{(1)}(\vect{x}; \vect{W}_1); \vect{W}_2)\\
% &= 
\sigma(\vect{W}_2 (\sigma(\vect{W}_1\vect{x})))
\end{align}

\input{definitions/colors}
\input{definitions/styles}
\begin{figure}[ht!]
\centering

\scalebox{1.0}{
\begin{tikzpicture}[auto]

% operations =========

% FNN input
\node[normal] (x1) {};
\node[textonly, above=10pt of x1] (input) {Camada de \textit{inputs}};
\node[normal, below=10pt of x1] (x2) {};
\node[normal, below=10pt of x2] (x3) {};
\node[normal, below=10pt of x3] (x4) {};
\node[normal, below=10pt of x4] (x5) {};
\node[normal, below=10pt of x5] (x6) {};

% FNN output
\node[textonly, right=50pt of x1] (center) {Camada Escondida};
\node[normal, below=25pt of center] (y1) {};
\node[normal, below=10pt of y1] (y2) {};
\node[normal, below=10pt of y2] (y3) {};

% FNN output
\node[textonly, right=110pt of input] (output) {Camada de \textit{outputs}};
\node[normal, below=15pt of output] (z1) {};

\node[normal, below=10pt of z1] (z2) {};
\node[normal, below=10pt of z2] (z3) {};
\node[normal, below=10pt of z3] (z4) {};
\node[normal, below=10pt of z4] (z5) {};
\node[normal, below=10pt of z5] (z6) {};

% phon features 2

% edges FNN
\path[nedge] (x1) -- (y1);
\path[nedge] (x1) -- (y2);
\path[nedge] (x1) -- (y3);

\path[nedge] (x2) -- (y1);
\path[nedge] (x2) -- (y2);
\path[nedge] (x2) -- (y3);

\path[nedge] (x3) -- (y1);
\path[nedge] (x3) -- (y2);
\path[nedge] (x3) -- (y3);

\path[nedge] (x4) -- (y1);
\path[nedge] (x4) -- (y2);
\path[nedge] (x4) -- (y3);

\path[nedge] (x5) -- (y1);
\path[nedge] (x5) -- (y2);
\path[nedge] (x5) -- (y3);

\path[nedge] (x6) -- (y1);
\path[nedge] (x6) -- (y2);
\path[nedge] (x6) -- (y3);

% edges FNN
\path[nedge] (y1) -- (z1);
\path[nedge] (y1) -- (z2);
\path[nedge] (y1) -- (z3);
\path[nedge] (y1) -- (z4);
\path[nedge] (y1) -- (z5);
\path[nedge] (y1) -- (z6);
\path[nedge] (y2) -- (z1);
\path[nedge] (y2) -- (z2);
\path[nedge] (y2) -- (z3);
\path[nedge] (y2) -- (z4);
\path[nedge] (y2) -- (z5);
\path[nedge] (y2) -- (z6);
\path[nedge] (y3) -- (z1);
\path[nedge] (y3) -- (z2);
\path[nedge] (y3) -- (z3);
\path[nedge] (y3) -- (z4);
\path[nedge] (y3) -- (z5);
\path[nedge] (y3) -- (z6);

\end{tikzpicture}
}\caption{Esquema de uma Rede Neural com Camada Escondida} 
\label{fig:ffd}
\end{figure}

\section{Treinamento}

Para que a rede seja capaz de identificar os padrões desejados, é necessário alimentá-la também com o que se espera como resposta (\textit{targets} ou \textit{alvos}), pois o treinamento da mesma consiste, essencialmente, na atualização das matrizes de pesos que deve ocorrer a partir da comparação entre os valores previstos pela rede (os \textit{outputs}) e os \textit{targets}. A comparação entre esses valores se dá por meio de uma função de custo (\textit{Loss Function}), que representa uma forma de se quantificar o quão perto se está de uma rede ideal em que os resultados previstos correspondam exatamente aos \textit{targets}. O objetivo do aprendizado da rede é minimizar essa diferença, ou seja, encontrar os valores dos pesos na matriz $\vect{W}$ que minimizam a função de custo (\cite{josh:2017}). Para isso, faz-se uso de um algoritmo chamado de \textit{backpropagation} (ref). Após a inserção de cada input e a aplicação do algoritmo de backpropagation, todos os pesos de $\vect{W}$ são atualizados simultaneamente com os valores que em conjunto minimizam a função de custo e portanto aproximam as previsões da rede aos alvos. Entretanto, o ajuste de um modelo de redes neurais é bastante delicado e experimental. Ele depende não somente da busca pelos melhores parâmetros de $\vect{W}$ como também de uma busca pelos chamados \textit{hiperparâmetros}. 

Os hiperparâmetros são parâmetros do modelo que são escolhidos a priori para configurar o modelo. Um deles refere-se ao número de vezes em que um mesmo input é inserido, conhecido como o número de \textit{épocas}. Existem também outros hiperparâmetros importantes como a taxa de aprendizado $\eta$, o parâmetro de regularização $\lambda$, dimensões de camadas escondidas, número de camadas escondidas, entre outros (\cite{josh:2017}). Todos esses hiperparâmetros definem as configurações do treinamento e juntos podem levar ao sucesso ou insucesso do aprendizado. A escolha dos melhores hiperparâmetros para um modelo de redes neurais não é tarefa trivial. Múltiplos treinamentos com diferentes configurações de hiperparâmetros são realizados para que se possa fazer essa escolha. Na prática, porém, é mais comum que esses hiperparâmetros sejam configurados a partir de resultados de experimentos semelhantes publicados na literatura.   

Mesmo com uma extensa busca pelos hiperparâmetros do modelo, nem sempre é possível encontrar valores em $\vect{W}$ que possibilitem o aprendizado de forma satisfatória. Na prática, é bastante comum a ocorrência de problemas conhecidos como \textit{overfit} (quando o modelo se torna especialista nos dados de treino mas tem dificuldade de generalizar para dados nunca vistos antes) e \textit{underfit} (quando o modelo falha em capturar o padrão subjacente dos dados) (\cite{josh:2017}). Na literatura existem algumas técnicas disponíveis para tratar dessas questões, como Dropout, Data Augmentation e Over ou Under Sampling (ref). Ainda assim, o modelo de redes neurais é um modelo cuja interpretabilidade mostra-se ainda bastante inacessível, o que torna o resultado do treinamento bastante hermético. (ref)   %ref 

\section{Uma Outra Arquitetura}
\label{sec:arqFDD}
%juntar o cap 4 aqui dentro
O esquema apresentado pelos pesquisadores Rumelhart e McClelland represetado na Fig. \ref{fig:esquemafdd} é conhecido como uma arquitetura do tipo \textit{Feedforward-Network (FFD)} sem camadas escondidas. Nesse caso, todos os nódulos da camada de input se conectam diretamente aos nódulos da camada de output.

% Com o modelo de RM foi possível atingir uma acurácia de 91\% na tarefa de aprendizado de verbos do Simple Past da língua inglesa. Entretanto a arquitetura do tipo FFD não é a mais adequada para dados sequenciais. No Cap. \ref{ch:02} foi abordado o complicado esquema de codificação dos verbos através de trigramas de traços distintivos. O processo de decodificação dos mesmos era ainda mais complicado e envolvia um esquema de competição entre trigramas (\cite{rumelhart:1986}). \cite{Pinker:1999} inclusive comenta sobre a dificuldade de decodificação da palavra 'algalgal' (uma palavra da língua Oykangand). Como a representação do output da rede FFD aponta apenas a presença ou ausência dos traços, e não quantas vezes eles aparecem, o processo de decodificação apresenta problemas e a rede dificilmente acerta em casos de verbos maiores ou com repetições de trigramas de traços. No caso da língua portuguesa, essa tarefa é bastante difícil pois os verbos são em média maiores que os verbos da língua inglesa (como comentado no Capítulo \ref{ch:01}). 

Com isso, introduz-se um novo tipo de rede conhecida como Rede Neural Recorrente (RNN) cuja arquitetura foi pensada para esse caso de dados sequenciais. Para entender o funcionamento desse tipo de rede, antes faz-se necessária uma introdução ao conceito de \textit{Modelo de Linguagem}.

\subsection{Modelo de Linguagem}

Um modelo de linguagem é um modelo que propõe uma distribuição probabilística para uma sequência de termos em uma linguagem natural (\cite{manning99foundations}).
Em outras palavras, um modelo de linguagem responde à pergunta: “dados os termos $x_1,x_2,x_3,x_4$, qual a probabilidade deles ocorrerem em uma determinada sequência?”

A forma clássica de se abordar esse problema se dá através da assunção de independência entre os termos da sequência em combinação à Regra do Produto de Probabilidades, ou seja, considera-se que a probabilidade de uma determinada sequência com $T$ termos acontecer é igual à probabilidade da $palavra_{t}$ ocorrer dado que as outras $t-1$ palavras ocorreram:

\begin{equation}
P(x_1, \dots, x_T) = \prod_{t=1}^{T} P(x_t \vert x_1, \dots, x_{t-1}) 
\end{equation}

Essas considerações deram origem aos modelos de $N$-Gramas. Diferentes valores para N dão origem a diferentes modelos. O modelo de unigrama, por exemplo, leva em consideração apenas a probabilidade de cada termo da sequência:

\begin{equation}
P_{uni}(x_1, x_2, x_3, x_4) = P(x_1)P(x_2)P(x_3)P(x_4)
\end{equation}

Em que $P(x_i) = contagem(x_i)$ e $contagem$ é uma função que conta a ocorrência dessa palavra no corpus.\\

A partir de N=2, ou seja, do modelo de \textit{bigramas}, esse tipo de modelagem começa a fazer mais sentido. A ideia é calcular a probabilidade dos termos dada a contagem de vezes em que eles ocorreram um após o outro em um corpus de treinamento.

\begin{equation}
P_{bi}(x_1,x_2,x_3,x_4) = P(x_1)P(x_2\vert x_1)P(x_3\vert x_2)P(x_4\vert x_3)
\end{equation} 
em que
\begin{equation}
P(x_i\vert x_j) = \frac{contagem(x_i, x_j)}{contagem(x_j)}
\end{equation}


Esse tipo de modelo de linguagem, apesar de intuitivo, apresenta alguns obstáculos. O primeiro deles é o fato de atribuir probabilidade 0 se uma sequência de termos específica não existe no corpus de treinamento. Outro grande problema é o número alto de computações das probabilidades conforme o texto aumenta de tamanho ou o \textit{N} do \textit{N-Grama} aumenta. Desse modo fica difícil de se manter uma coerência dentro de uma sentença, o que se vê na prática são frases constituídas de “remendos”, pequenos grupos de (\textit{N}) palavras que produzem uma janela muito pequena de sentido.

\subsection{Redes Neurais Recorrentes}
\label{sec:RNN}

Os modelos de Redes Neurais Recorrentes (ref) surgem como uma alternativa ao tradicional modelo de linguagem de \textit{N-Gramas}. A ideia central desse modelo consiste na retroalimentação dos elementos sequenciais, de modo que o \textit{input} de cada um deles serve, não somente para a previsão do próximo item da sequência, mas também para a formação de um componente intermediário, um \textit{estado}. Esses estados, representados na Figura \ref{fig:unfoldedrnn} como $\vect{h}$'s são matrizes que funcionam como uma espécie de memória condensada dos elementos precedentes e servem como input para os estados posteriores. Essa é uma maneira de retransmitir a cada momento os efeitos dos \textit{inputs} anteriores para o restante da sequência (\cite{Goodfellow-et-al-2016}). 

%\input{definitions/colors}
\input{definitions/styles}
\begin{figure}[ht!]
\centering
\scalebox{1.40}{
\begin{tikzpicture}[auto]

% RNN state cell =============================
\node[normal] (h) {$\vect{h}$};
\node[normal, below=30pt of h] (x) {$\vect{x}$};
\node[normal, above=30pt of h] (yhat) {$\hat{\vect{y}}$};



% edges
\path[tedge] (x) edge node[below right= -4pt] {}  (h) ;
\path[tedge] (h) edge [out=-400,in=-320,looseness=8, distance=125pt] node[above right] {} (h);
\path[tedge] (h) edge node[below right = -4pt] {} (yhat);


\end{tikzpicture}

} % scalebox
\caption{Grafo Computacional de uma Rede Neural Recorrente}
\label{fig:rnn-cell}
\end{figure}
\input{definitions/colors}
\input{definitions/styles}

% RNN STATE CELL ====================================

\newcommand{\rnnSimple}[4]{

% operations
\node[normal, minimum size=40pt,#4] (h#3) {$\vect{h}^{#1}$};
\node[normal, minimum size=40pt,below=30pt of h#3] (x#3) {$\vect{x}^{#1}$};
\node[normal, minimum size=40pt, above=30pt of h#3] (yhat#3) {$\hat{\vect{y}}^{#1}$};

% edges
\path[tedge] (x#3) edge node[below right= -4pt] {} (h#3);
\path[tedge] (h#3) edge node[below right = -4pt] {} (yhat#3);
}

\begin{figure}[ht!]
\centering
\hspace*{-1.0cm}
\scalebox{0.9}{
\begin{tikzpicture}[auto]

% timestep 1
\rnnSimple{(1)}{(0)}{t1}{}

% % timestep 0
\node[normal, minimum size=40pt,left=50pt of ht1] (ht0) {$\vect{h}^{(0)}$};

% % timestep 2
\rnnSimple{(2)}{(1)}{t2}{right=50pt of ht1};
\node[textonly, below= 80pt of ht1] (ontem) {Ontem};
\node[textonly, above= 80pt of ht1] (eu) {João};
\path [pil, bend left=40] (ontem) edge node {} (eu);

% % timestep 3
\rnnSimple{(3)}{(1)}{t3}{right=50pt of ht2};
\node[textonly, below= 80pt of ht2] (eu2) {João};
\path [pil, bend left=3] (eu) edge node {} (eu2);
\node[textonly, above= 80pt of ht2] (fui) {foi};
\path [pil, bend left=40] (eu2) edge node {} (fui);

% timestep future
\node[textonly, below= 80pt of ht3] (fui2) {foi};
\path [pil, bend left=3] (fui) edge node {} (fui2);
\node[textonly, above= 80pt of ht3] (na) {na};
\path [pil, bend left=40] (fui2) edge node {} (na);
\node[textonly, right= 15pt of ht3] (future) {...};

%  \node[left=of dummy] (g) {Ultimate lender}
%   edge[pil, bend right=45] (market.west)
%   edge[pil, bend right=45] (formidler.west)
%   edge[pil,<->, bend left=45] node[auto] {Direct (a)} (t);

% % state transfers
\path[tedge] (ht0) edge node[above right = 2pt] {} (ht1);
\path[tedge] (ht1) edge node[above right = 2pt] {} (ht2);
\path[tedge] (ht2) edge node[above right = 2pt] {} (ht3);

\end{tikzpicture}
}%\scalebox
\caption{Modelo de Linguagem com Rede Neural Recorrente}
\label{fig:unfoldedrnn}
\end{figure}




Os estados indicados na Fig. \ref{fig:unfoldedrnn} são calculados a partir da equação recorrente:

\begin{equation}
\vect{h}^{(t)} = g(\vect{h}^{(t-1)}, \vect{x}^{(t)}; \vect{\theta})
\label{eq:rnn}
\end{equation}

Primeiramente, uma etapa de pré-processamento transforma todos os termos de um corpus de treinamento em vetores. A forma vetorial mais intuitiva para esse caso é conhecida como representação \textit{one-hot}, de modo que um vetor com o tamanho do vocabulário é inicializado com valores nulos e cada termo atribui o valor de 1 a sua dimensão correspondente nesse vetor (\cite{harris:2013}).\footnote{Uma alternativa a essa representação é conhecida como \textit{embedding} (\cite{word2vec:2013})}\\
O estado $\vect{h(0)}$ é normalmente inicializado de maneira aleatória e entra, em conjunto com o primeiro \textit{input} (o primeiro termo da sequência), no estado $\vect{h(1)}$. O alvo desse primeiro passo é o segundo termo da sequência ($\hat{\vect{y}}^{(1)}$). Em seguida, o segundo termo da sequência tem como seu respectivo alvo o próximo termo e assim por diante. Após o treinamento da rede e a aplicação do algoritmo de \textit{backpropagation}, espera-se que o último estado tenha incorporado uma certa memória de todos os estados anteriores e capturado as relações de dependência entre os termos, de modo que por fim seja possível a utilização desse modelo para gerar sequências de palavras. 
%\chapter{Modelos de Linguagem}
\label{ch:04}

Explicar que o meu modelo requer conhecimentos sobre o conceito de modelo de linguagem para ser compreendido




\section{Definição}

Um modelo de linguagem é um modelo que propõe uma distribuição probabilística para uma sequência de termos em uma linguagem natural. %Esses termos podem ser palavras ou caracteres e podem incluir contrações ("I'm, she's"), ênclises ("vende-se, aluga-se") 
Utilizando 
\[
P(x_1,x_2,x_3,x_4) = p
\]

% The classical approach to a language model was to use the chain rule of probability and a Markovian assumption, i.e., for a specific $n$ we assume that:

% \begin{equation}
% P(x_1, \dots, x_T) = \prod_{t=1}^{T} P(x_t \vert x_1, \dots, x_{t-1}) = \prod_{t=1}^{T} P(x_{t} \vert x_{t - (n+1)}, \dots, x_{t-1})
% \end{equation} 


% This gave raise to models based on $n$-gram statistics. The choice of $n$ yields different models; for example, the 
% \textit{unigram} language model ($n=1$) is defined as: 
% \begin{equation}
% P_{uni}(x_1, x_2, x_3, x_4) = P(x_1)P(x_2)P(x_3)P(x_4)
% \end{equation}

% where $P(x_i) = count(x_i)$ and $count$ is a function that counts tokens occurrence in a corpus.\\

% Similarly the \textit{bigram} language model ($n=2$) is defined as: 
% \begin{equation}
% P_{bi}(x_1,x_2,x_3,x_4) = P(x_1)P(x_2\vert x_1)P(x_3\vert x_2)P(x_4\vert x_3)
% \end{equation} 
% where
% \begin{equation}
% P(x_i\vert x_j) = \frac{count(x_i, x_j)}{count(x_j)}

\subsection{N-Gramas}
usar como exemplo n-gramas

\section{Redes Neurais Recorrentes}
\label{sec:RNN}

Na arquitetura de rumelhart e mccleland foi necessário acrescentar os indicadores de fronteira, e ainda, efetuar uma análise dos verbos em trigramas, ambos artifícios para tentar preservar a lógica sequencial.

Explicar redes neurais recorrentes melhor antes de introduzir a ideia de estados.

%a dificuldade desta estratégia motivou a implementação de um arquitetura melhor preparada para a realização da tarefa, as Redes Neurais Recorrentes (RNR). ?

A ideia central desse modelo consiste na retroalimentação dos elementos sequenciais, de modo que o \textit{input} de cada um deles serve, não somente para a previsão do próximo item da sequência, mas também para a formação de um componente intermediário, um \textit{estado}. Esses estados, representados na Figura \ref{fig:unfoldedrnn} como $\vect{h}$'s são matrizes que funcionam como uma espécie de memória condensada dos elementos precedentes e servem como input para os estados posteriores. Essa é uma maneira de retransmitir a cada momento os efeitos dos \textit{inputs} anteriores para o restante da sequência (\cite{Goodfellow-et-al-2016}). 

%\input{definitions/colors}
\input{definitions/styles}
\begin{figure}[ht!]
\centering
\scalebox{1.40}{
\begin{tikzpicture}[auto]

% RNN state cell =============================
\node[normal] (h) {$\vect{h}$};
\node[normal, below=30pt of h] (x) {$\vect{x}$};
\node[normal, above=30pt of h] (yhat) {$\hat{\vect{y}}$};



% edges
\path[tedge] (x) edge node[below right= -4pt] {}  (h) ;
\path[tedge] (h) edge [out=-400,in=-320,looseness=8, distance=125pt] node[above right] {} (h);
\path[tedge] (h) edge node[below right = -4pt] {} (yhat);


\end{tikzpicture}

} % scalebox
\caption{Grafo Computacional de uma Rede Neural Recorrente}
\label{fig:rnn-cell}
\end{figure}
\input{definitions/colors}
\input{definitions/styles}

% RNN STATE CELL ====================================

\newcommand{\rnnSimple}[4]{

% operations
\node[normal, minimum size=40pt,#4] (h#3) {$\vect{h}^{#1}$};
\node[normal, minimum size=40pt,below=30pt of h#3] (x#3) {$\vect{x}^{#1}$};
\node[normal, minimum size=40pt, above=30pt of h#3] (yhat#3) {$\hat{\vect{y}}^{#1}$};

% edges
\path[tedge] (x#3) edge node[below right= -4pt] {} (h#3);
\path[tedge] (h#3) edge node[below right = -4pt] {} (yhat#3);
}

\begin{figure}[ht!]
\centering
\hspace*{-1.0cm}
\scalebox{0.9}{
\begin{tikzpicture}[auto]

% timestep 1
\rnnSimple{(1)}{(0)}{t1}{}

% % timestep 0
\node[normal, minimum size=40pt,left=50pt of ht1] (ht0) {$\vect{h}^{(0)}$};

% % timestep 2
\rnnSimple{(2)}{(1)}{t2}{right=50pt of ht1};
\node[textonly, below= 80pt of ht1] (ontem) {Ontem};
\node[textonly, above= 80pt of ht1] (eu) {João};
\path [pil, bend left=40] (ontem) edge node {} (eu);

% % timestep 3
\rnnSimple{(3)}{(1)}{t3}{right=50pt of ht2};
\node[textonly, below= 80pt of ht2] (eu2) {João};
\path [pil, bend left=3] (eu) edge node {} (eu2);
\node[textonly, above= 80pt of ht2] (fui) {foi};
\path [pil, bend left=40] (eu2) edge node {} (fui);

% timestep future
\node[textonly, below= 80pt of ht3] (fui2) {foi};
\path [pil, bend left=3] (fui) edge node {} (fui2);
\node[textonly, above= 80pt of ht3] (na) {na};
\path [pil, bend left=40] (fui2) edge node {} (na);
\node[textonly, right= 15pt of ht3] (future) {...};

%  \node[left=of dummy] (g) {Ultimate lender}
%   edge[pil, bend right=45] (market.west)
%   edge[pil, bend right=45] (formidler.west)
%   edge[pil,<->, bend left=45] node[auto] {Direct (a)} (t);

% % state transfers
\path[tedge] (ht0) edge node[above right = 2pt] {} (ht1);
\path[tedge] (ht1) edge node[above right = 2pt] {} (ht2);
\path[tedge] (ht2) edge node[above right = 2pt] {} (ht3);

\end{tikzpicture}
}%\scalebox
\caption{Modelo de Linguagem com Rede Neural Recorrente}
\label{fig:unfoldedrnn}
\end{figure}




Os estados indicados na Fig. \ref{fig:unfoldedrnn} são calculados a partir da equação recorrente\footnote{O estado $\vect{h(0)}$ é normalmente inicializado de maneira aleatória.}:

\begin{equation}
\vect{h}^{(t)} = g(\vect{h}^{(t-1)}, \vect{x}^{(t)}; \vect{\theta})
\label{eq:rnn}
\end{equation}

Na prática, essa arquitetura simplificaria algumas das etapas do processo de predição de irregularidades verbais. A primeira delas seria o processo de codificação dos verbos, por exemplo, ao se utilizar uma rede recorrente, indicadores de fronteira e a ativação de unidades em trigramas não se fazem mais necessários. No treinamento desse tipo de rede, os inputs são gerados a partir de cada item da série, e os alvos, por sua vez, são os itens subsequentes respectivos. Isto significa, essencialmente, que a rede será treinada para prever item a item de uma sequência. O alvo mais simples para este tipo de processamento são fonemas, uma vez que traços fonológicos ou Wickelfeatures, por serem intrinsecamente mais complexos, exigiriam uma arquitetura de rede muito mais sofisticada. Por fim, o processo de decodificação dos verbos também torna-se desnecessário uma vez que o output é computado item após item. A Fig. \ref{fig:rnnpractice} ilustra o problema da predição de fonemas dada uma base de verbos flexionados. No exemplo, considera-se o treinamento do verbo \textit{"paru"} (já utilizando a notação escolhida). % trocar por u mas lembrar de explicar q na minha rede nao tem distinção entre esses u's

% Apesar da praticidade desse tipo de rede, ela infelizmente apresenta alguns problemas para tarefas que envolvem dependências de longa distância, o que normalmente é o caso quando se trata de tarefas linguísticas. O problema ocorre porque, a cada \textit{input}, as matrizes de pesos são atualizadas fazendo com que as últimas informações recebidas sejam mais relevantes do que as anteriores, impedindo o progresso do treinamento. Dessa forma, apresentam-se na literatura algumas soluções. Entre elas, uma arquitetura conhecida como \textit{Long Short-Term Memory - LSTM}.

\input{definitions/colors}
\input{definitions/styles}

% RNN STATE CELL ====================================

\begin{figure}[ht!]
\centering
\hspace*{-1.0cm}
\scalebox{0.9}{
\begin{tikzpicture}[auto]

% timestep 1
\rnnSimple{(1)}{(0)}{t1}{}

% % timestep 0
\node[normal, minimum size=40pt,left=50pt of ht1] (ht0) {$\vect{h}^{(0)}$};

% % timestep 2
\rnnSimple{(2)}{(1)}{t2}{right=50pt of ht1};
\node[textonly, below= 80pt of ht1] (p) {p};
\node[textonly, above= 80pt of ht1] (â) {a};

% % timestep 2
\rnnSimple{(3)}{(1)}{t3}{right=50pt of ht2};
\node[textonly, below= 80pt of ht2] (a) {a};
\node[textonly, above= 80pt of ht2] (r_o) {r};

% % state transfers
\path[tedge] (ht0) edge node[above right = 2pt] {} (ht1);
\path[tedge] (ht1) edge node[above right = 2pt] {} (ht2);
\path[tedge] (ht2) edge node[above right = 2pt] {} (ht3);

% timestep3
\node[textonly, below= 80pt of ht3] (r) {r};
\node[textonly, above= 80pt of ht3] (end) {\textupsilon};

\end{tikzpicture}
}%\scalebox
\caption{RNR - Modelo de Linguagem}
\label{fig:rnnpractice}
\end{figure}




tentar colocar setinhas de sentido ou explicar que começa no x1 e vai pro y1 etc

% \subsection{Long Short-Term Memory}
% \label{sec:LSTM}

% Uma LSTM continua sendo uma RNN, porém existem algumas especificidades em sua arquitetura que a tornam mais elaborada e mais adequada para atender aos treinamentos que envolvem longa dependência. A chave para entender como as LSTM's lidam com esse problema está no entendimento de um componente chamado de \textit{cell state}. Esse componente funciona como uma corrente transportadora de informação e é regulado por estruturas que funcionam como válvulas (os \textit{gates}). Os \textit{gates} são uma forma de filtragem de informação. Eles são basicamente responsáveis por selecionar informações antigas que podem ser esquecidas e novas que sejam relevantes. 

% %imagem de LSTM
% \input{tikz/lstm.tex}

% A Fig. \ref{fig:lstm} deve ser interpretada da seguinte forma: A cada novo input ($x(t)$), uma camada LSTM recebe o estado $h(t-1)$ gerado pelo input anterior assim como uma RNN comum receberia, mas recebe também o Cell State de (t-1). O Cell State traz consigo uma lista de todas as informações que ele considera que devem ser carregadas e mantidas durante todo o treinamento. Após a intervenção dos gates, o modelo pode decidir por agregar novas informações e esquecer outras nessa lista, gerando assim um novo Cell State (t). Além disso, a passagem desse input por essa camada também gera um novo estado ($h(t)$) que, além de servir como referência para o cálculo do output de (t), servirá também como referência para o próximo processamento ($x(t+1)$). (\cite{Goodfellow-et-al-2016})

% \section{Encoder-Decoder}
% \label{sec:enc-dec}
\chapter{Encoder-Decoder}
\label{ch:05}


\section{Introdução ao Modelo Enconder-Decoder}
\label{sec:intro-sec-sec}
Um modelo de mapeamento \textit{Encoder-Decoder} (\cite{enc-dec:2014}) é um sistema composto por duas Redes Neurais Recorrentes cuja principal função é mapear uma relação entre duas sequências distintas que possuem uma relação paradigmática. Modelos do tipo \textit{Encoder-Decoder} (também conhecidos como \textit{Seq2Seq} (\cite{seq2seq:2014}) têm sido bastante utilizados em tarefas linguísticas, especialmente no desenvolvimento de sistemas de diálogo e em tradução automática.

Em um contexto de tradução, por exemplo, o modelo recebe como \textit{input }uma sequência de uma língua de origem e produz como \textit{output} uma sequência em uma língua alvo. A sequência gerada precisa, além de preservar o conteúdo semântico da sequência de origem, apresentar uma sintaxe aceita pelos falantes da língua alvo. 

\begin{figure}[ht!]
\centering
\begin{tikzpicture}
\node[punkt] (seq1) {Sequência 1};
\node[punkt, right=40pt of seq1] (seq2) {Sequência 2};

\path[tedge] (seq1) -- (seq2);

\node[punkt, below=40pt of seq2] (seq21) {O menino pequeno gosta do seu cachorro.};

\node[punkt, left=40 pt of seq21] (seq12) {The small boy likes his dog.};

\path[tedge] (seq12) -- (seq21);

\node[text, below=10 pt of seq21] (nada) {};

\end{tikzpicture}
\caption{Objetivo do Modelo Encoder-Decoder: Mapeamento de uma Sequência à Outra} 
\label{fig:seq2seq_simple}
\end{figure}
Observa-se no caso da tradução, que não há uma correspondência exata entre os termos de cada uma das línguas. A sentença na língua inglesa possui um termo a menos, sendo que o pronome “his” corresponde aos termos “do seu” no português. Além disso, observa-se que cada uma das línguas possui uma sintaxe diferente. No primeiro caso, o adjetivo \textit{small} se posiciona antes do substantivo \textit{boy}. No segundo, essa ordem é invertida (\textit{menino pequeno}). 

As duas redes recorrentes funcionam da seguinte maneira (ver Fig. \ref{fig:seq2seq}): uma primeira rede, a denominada \textit{Encoder} é alimentada com uma sentença da língua de origem (em inglês, por exemplo). Os termos dessa sentença entram um após o outro na rede e alimentam os \textit{estados} ($\vect{h}^{(t)}$), porém nesse caso não é necessário relacionar cada termo a um correspondente ($y_t$) como no caso do modelo de linguagem. A rede \textit{Encoder}, portanto, não possui \textit{alvos}, ela serve apenas para acumular os dados da língua de origem. Para tanto, o resultado do último estado, gerado após a inserção de todos os termos da sequência de origem, serve como estado inicial ($\vect{h}^{(0)}$) para uma segunda rede recorrente, denominada de \textit{Decoder}. A rede \textit{Decoder}, por sua vez, recebe como seu primeiro \textit{input}, um \textit{token} que representa o início da sequência alvo (\textbf{<beg>}), inicializando o modelo de linguagem dessa língua, como no modelo de rede neural recorrente apresentado na Seção \ref{sec:RNN}. 

 \input{tikz/seq2seq.tex}
 
 Por fim, o aprendizado do modelo ocorre como já explanado no Capítulo \ref{ch:03}: os \textit{outputs} do modelo são comparados com os \textit{alvos}, o erro é propagado de volta através do algoritmo de \textit{backpropagation}, os pesos da rede são atualizados de modo a diminuir esse erro e o processo se repete até a conclusão de todas as \textit{épocas}. É importante ressaltar também que, embora a figura ilustre como exemplo duas sequências de três palavras, o modelo comporta sequências de tamanhos quaisquer e que não precisam coincidir entre si.
 
\section{A Questão do Aprendizado de Flexão dos Verbos}

Assim como no problema da tradução automática, pode-se retratar a questão do aprendizado de flexão dos verbos através de uma relação entre duas sequências que compartilham de um mesmo paradigma. No caso desta pesquisa, trataremos apenas da transformação partindo de uma forma base para a forma flexionada escolhida (Seç. \ref{sec:escopo}).

\begin{figure}[ht!]
\centering
\begin{tikzpicture}
\node[punkt] (seq1) {Verbo no Infinitivo};
\node[punkt, right=40pt of seq1] (seq2) {Verbo Flexionado};

\path[tedge] (seq1) -- (seq2);

\node[punkt, below=40pt of seq2] (seq21) {estudo};

\node[punkt, left=40 pt of seq21] (seq12) {estudar};

\path[tedge] (seq12) -- (seq21);

\node[text, below=10 pt of seq21] (nada) {};
\end{tikzpicture}
\caption{Relação entre um Verbo no Infinitivo e o Mesmo Flexionado} 
\label{fig:verbs}
\end{figure}

Diferentemente do caso de tradução, em que se busca o aprendizado das relações entre uma palavra e outra das línguas de origem e de alvo, no caso do aprendizado de flexão de verbos busca-se aprender as relações existentes entre uma forma verbal básica e uma flexionada a nível fonético. Desse modo, resta explicar como se deu a transformação dos verbos em vetores adequados para que o modelo fosse capaz de aprender tais relações.

\subsection{O modelo Encoder-Decoder Desenvolvido}

\subsubsection{Encoder}

Relembrando a Seç \ref{sec:inputs}, temos que cada fone é representado através de uma forma vetorial com 20 dimensões (correspondentes ao número de traços fonéticos do estudo). Desse modo, a camada de \textit{inputs} da rede \textit{Encoder} tem dimensão 20, e é responsável por receber os verbos na forma base (RAD + VT). Em seguida, o \textit{input} é direcionado para uma rede de tipo LSTM unidirecional. Essas duas camadas constituem o que é chamado de \textit{Encoder}.

\subsubsection{Decoder}

A rede \textit{Decoder}, por sua vez, funciona como um modelo de linguagem. O estado inicial desse modelo ($\vect{h}^{(0)}$) é o estado final gerado pela rede \textit{Encoder}. Além disso, também entram como \textit{input} na rede os respectivos verbos flexionados. A rede recorrente utilizada também é de tipo LSTM unidirecional.

A Figura \ref{fig:seq2seq} apresenta um esquema do modelo desenvolvido utilizando como exemplo o verbo irregular “\textit{ler}”.

\begin{figure}[ht!]
\centering

\scalebox{1.0}{
\begin{tikzpicture}[
  hid/.style 2 args={
    rectangle split,
    rectangle split horizontal,
    draw=#2,
    rectangle split parts=#1,
    fill=#2!20,
    outer sep=1mm}]
  % draw input nodes
  \foreach \i [count=\step from 1] in {l,e,r,{{$<$eos$>$}}}
    \node (i\step) at (2*\step, -2) {\i};
  % draw output nodes
  \foreach \t [count=\step from 4] in {l,e,i,{{o}}} {
    \node[align=center] (o\step) at (2*\step, +2.75) {\t};
  }
  % draw embedding and hidden layers for text input
  \foreach \step in {1,...,3} {
    \node[hid={3}{red}] (h\step) at (2*\step, 0) {};
    \node[hid={3}{red}] (e\step) at (2*\step, -1) {};    
    \draw[->] (i\step.north) -> (e\step.south);
    \draw[->] (e\step.north) -> (h\step.south);
  }
  % draw embedding and hidden layers for label input
  \foreach \step in {4,...,7} {
    \node[hid={3}{yellow}] (s\step) at (2*\step, 1.25) {};
    \node[hid={3}{blue}] (h\step) at (2*\step, 0) {};
    \node[hid={3}{blue}] (e\step) at (2*\step, -1) {};    
    \draw[->] (e\step.north) -> (h\step.south);
    \draw[->] (h\step.north) -> (s\step.south);
    \draw[->] (s\step.north) -> (o\step.south);
  }  
  % edge case: draw edge for special input token
  \draw[->] (i4.north) -> (e4.south);
  % draw recurrent links
  \foreach \step in {1,...,6} {
    \pgfmathtruncatemacro{\next}{add(\step,1)}
    \draw[->] (h\step.east) -> (h\next.west);
  }
  % draw predicted-labels-as-inputs links
  \foreach \step in {4,...,6} {
    \pgfmathtruncatemacro{\next}{add(\step,1)}
    \path (o\step.north) edge[->,out=45,in=225] (e\next.south);
  }
\end{tikzpicture}
}\caption{Esquema de Encoder-Decoder Utilizado} 
\label{fig:seq2seq}
\end{figure}

% \subsubsection{Metodologia Computacional}

% O modelo foi desenvolvido utilizando-se a API \textit{Keras} (\cite{chollet2015keras}). A arquitetura final pode ser verificada na Fig. \ref{fig:arq}. Na figura, notamos a adição de uma camada intermediária após a saída da rede. 





% \begin{figure}[!htb]
%         \center{\includegraphics[width=0.7\textwidth]
%         {img/encoder-decoder.png}}
%         \caption{\label{fig:arq} Arquitetura Utilizada}
%       \end{figure}

 \subsection{Pós-Processamento}

Assim como é necessária uma etapa de pré-processamento para transformar os verbos em vetores numéricos para a alimentação do modelo, também é necessário traduzir a saída do modelo (que neste momento corresponde a uma sequência de vetores numéricos) para uma sequência de caracteres e para a reconstrução do verbo flexionado predito.

Primeiramente, temos que a predição do verbo flexionado é o resultado da concatenação das predições do modelo de linguagem aprendido durante o treinamento. Dessa forma, para realizar uma predição, o algoritmo libera um vetor de cada vez. O algoritmo finaliza a predição de fones de um verbo assim que prevê um \textit{token} de final. Tais predições não constituem mais vetores binários, ao invés disso, o modelo solta vetores com valores decimais variando de 0 a 1 (conforme a função de ativação utilizada). Por essa razão, para o pós-processamento dos vetores de saída são feitas duas aproximações: (i) valores abaixo de 0.5 foram substituídos por 0 e valores acima são substituídos por 1; (ii) o resultado dessa operação pode não resultar em um fone válido, com por exemplo um vetor com presença de dois traços contraditórios (ex: fricativo e oclusivo), portanto é substituído (se necessário) pela representação vetorial cuja distância seja mínima quando comparada a todos os fones possíveis. 

Na álgebra linear existem várias maneiras para se medir a distância entre dois vetores. Nesta pesquisa, optou-se pela \textit{distância euclidiana} (ref). A distância euclidiana entre os pontos \textbf{p} e \textbf{q} é equivalente ao comprimento do segmento de reta que os conecta (${\displaystyle {\overline {\mathbf {p} \mathbf {q} }}}$). Utilizando coordenadas cartesianas, e sejam \textbf{q} = ($q_1, q_2, ..., q_n$) e \textbf{p} = ($p_1, p_2, ..., p_n$) dois pontos em um espaço n-dimensional, temos:

\begin{equation}
    d(\textbf{q}, \textbf{p)} = \sqrt{(q_1 - p_1)^2 + (q_2 - p_2)^2 + ... + (q_n - p_n)^2}\notag
\end{equation}
\begin{equation}
    = \sqrt{\sum_{i=1}^n (q_i-p_i)^2}.    
\end{equation}

Desse modo, o esquema apresentado na Fig. \ref{fig:pos-process} apresenta um resumo do esquema de pós-processamento utilizado neste trabalho.
\begin{figure}[H]
\centering
\begin{tikzpicture}
\node[punkt] (pred) {Predição\\$\hat{y}$};
\node[punkt, right=40pt of pred] (arr) {Aproximação\\$\hat{y}_{ap}$};
\node[punkt, right=40pt of arr] (dist) {Distância Euclidiana com Fones Candidatos};
\node[punkt, right=40pt of dist] (subs) {Substituição pelo Vetor Candidato mais Próximo};

\path[tedge] (pred) -- (arr);
\path[tedge] (arr) -- (dist);
\path[tedge] (dist) -- (subs);


\node[punkt, below=40 pt of pred] (pred1) {[0.8734, 0.235, 0.003, ...]};
\node[punkt, below=40pt of arr] (arr1) {[1.0, 0.0, 0.0, ...]};
\node[punkt, right=160pt of arr1] (subs1) {[1.0, 0.0, 0.0, ...]};


\path[tedge] (pred1) -- (arr1);
\path[tedge] (arr1) -- (subs1);

\node[text, below=10 pt of arr1] (nada) {};



\end{tikzpicture}
\caption{Decodificação dos Outputs do Modelo} 
\label{fig:outputs}
\end{figure}

 

% \subsection{Resumo}

% Em resumo, o projeto final apresentado foi composto por 8 etapas: 

% \begin{enumerate}
%     \item Composição do Corpus
%     \item Transcrição Fonética dos Verbos Coletados
%     \item Transformação dos Verbos Transcritos em Vetores Numéricos
%     \item Inserção dos Vetores Obtidos no modelo Encoder-Decoder
%     \item Treinamento do Modelo
%     \item Predições Vetoriais do Modelo Encoder-Decoder
%     \item Decodificação das Predições Vetoriais para Fones
%     \item Concatenação dos Fones para Formação de um Verbo
% \end{enumerate}




%\chapter{Corpus}
\label{ch:06}

Para a execução dos experimentos, o escopo dos experimentos foi restringido a um único tempo, modo, pessoa e número, já que o português apresenta formas irregulares em múltiplas combinações destes elementos e seria necessário construir uma rede diferente para cada uma destas combinações. Optou-se por estudar as irregularidades presentes nas flexões de primeira pessoa do singular no tempo presente, modo indicativo.  
O corpus utilizado para o treinamento dessa rede foi baseado nas informações contidas em \url{https://www.conjugacao.com.br/verbos-irregulares/} e \url{https://www.conjugacao.com.br/verbos-regulares/}.\\
Primeiramente, foi realizada uma etapa de extração dos verbos e suas respectivas conjugações para um arquivo \textit{.csv} via técnica de \textit{webscraping}, uma técnica que extrai informações contidas nas páginas da web [incluir ref]. Em seguida, os verbos irregulares foram selecionados manualmente para diferentes famílias de verbos, ou seja, grupos que continham o mesmo padrão de flexão. Alguns dos verbos irregulares listados na fonte de referência não eram irregulares no processo de flexão de interesse e portanto foram realocados para o grupo de verbos regulares. Como um exemplo disto, um dos motivos para o verbo \textit{acudir} ser considerado irregular é a flexão que ocorre na terceira pessoa do singular no tempo presente e modo indicativo (acode), porém como na primeira pessoa do singular o padrão regular se mantém, esse verbo foi realocado para o grupo dos verbos regulares. Em seguida, foi realizado um experimento na tentativa de utilizar o transcritor fonético automático dispobilizado por \cite{guide:2016}, porém o transcritor falhou na tentativa de transcrever verbos cuja escrita coincide com substantivos, como por exemplo "apoio", "peso", "toco", "posto", "jogo", entre outros. Dessa forma, como haviam problemas na utilização do algoritmo, optou-se por fazer uma transcrição manual. No total, foram obtidos 423 verbos, 83 verbos a menos que no experimento realizado por \cite{rumelhart:1986}.

Dos 423 verbos extraídos, 20 foram considerados verbos sem possível agrupamento e foram alocados para uma classe chamada \textit{None} (verbos como "ir", "trazer" e "saber"), totalizando uma base de 214 verbos regulares e 209 irregulares (50.6\% e 49.4\% respectivamente). A Tabela \ref{tab:classes} associa os nomes dados aos tipos de irregularidades encontradas e apresenta um exemplo de cada uma delas, além do número de exemplos de cada e a proporção desse número no total. 

\begin{table}[H]
\begin{center}
\begin{tabular}{|l|c|c|c|c|}
\toprule
& Classes & Exemplos & Contagem & Proporção\\
\midrule
1 & iar:eiu & ansiar, anseio & 9 & 2.13\%\\
2 & o\_ar:O\_u & tocar, toco & 30 & 7.09\%\\
3 & o\_ir:u\_o & cobrir, cubro & 7 & 1.65\%\\
4 & izer:igu & dizer, digo & 7 & 1.65\%\\
5 & fazer:fasu & fazer, faço & 15 & 3.55\%\\
6 & crer:eiu & crer, creio & 5 & 1.18\%\\
7 & entir:intu & sentir, sinto & 8 & 1.89\% \\
8 & edir:Esu & pedir, peço & 7 & 1.65\%\\
9 & or:oNu & pôr, ponho & 27 & 6.38\%\\
10 & e\_ir:i\_u & seguir, sigo & 27 & 6.38\%\\
11 & ter:teNu & ter, tenho & 10 & 2.36\%\\
12 & e\_ar:E\_u & testar, testo & 20 & 4.73\%\\
13 & ver:veju & ver, vejo & 6 & 1.42\%\\
14 & vir:veniu & vir, venho & 10 & 2.60\%\\
15 & None & saber, sei & 20 & 4.73\%\\
16 & Regulares & falar, falo & 214 & 50.59\%\\
\bottomrule
\end{tabular}
\end{center}
\captionof{table}{Organização do corpus}
\label{tab:classes}
\end{table}
\chapter{Experimentos e Resultados}
\label{ch:07}

Este capítulo será dedicado a explanações sobre o modelo construído e às predições resultantes. A Seção \ref{sec:treinamento} de treinamento do modelo trata especificamente dos hiperparâmetros\footnote{colocar alguma coisa aqui} utilizados e dos resultados das métricas utilizadas. Após o treinamento do modelo, segue a etapa de decodificação dos outputs (citada no Capítulo \ref{ch:05}). Somente nesta etapa pode-se analisar os verbos preditos pela rede e avaliar a acurácia obtida pelo modelo. A Seção \ref{sec:results} introduz a metodologia utilizada para a avaliação dos resultados das decodificações e apresenta uma visão geral do desempenho do modelo para cada classe de verbo. Em seguida, a Seção \ref{sec:interesting} analisa com mais detalhes resultados interessantes obtidos pelo modelo. 

\section{Treinamento}
\label{sec:treinamento}

Como foi introduzido no Capítulo \ref{ch:03}, o treinamento do modelo de redes neurais consiste sempre na introdução de um input em paralelo com seu alvo correspondente. Contudo ficou faltando uma discussão a respeito da frequência em que esses dados são apresentados ao modelo. Essa questão é importante pois a frequência é um dos fatores mais relevantes para o ajuste dos pesos das redes. Uma frequência muito baixa não é o suficiente para causar grandes modificações nos pesos (\textit{underfit}). Em contrapartida, uma frequência muito alta faz com que o modelo se torne especialista nos dados de treinamento, porém se torna incapaz de lidar com dados nunca antes vistos (\textit{overfit}). Para encontrar a frequência ideal, utiliza-se como auxílio um gráfico conhecido como curva de aprendizado, como o apresentado na Figura \ref{fig:training2000}. No ramo da modelagem em redes neurais, cada vez que o grupo de dados de treino é introduzido completamente no modelo, configura-se o que é chamado de uma \textit{época}. 

\subsubsection{Types x Tokens}
Outra questão importante a respeito da frequência dos dados concerne os conceitos \textit{word type} e \textit{word token} (\cite{Manning:1999}).   
Em Processamento de Linguagem Natural, o termo \textit{type} refere-se ao conjunto de palavras únicas presentes em um Corpus. A frase “Essa frase é uma frase de exemplo.”, portanto, possui 6 \textit{types}. O termo \textit{token}, em contrapartida, refere-se ao número total de termos presentes. Nesse caso, a mesma frase de exemplo possui 7 \textit{tokens}. Evidências na área de psicolinguística (\cite{Bybee:1995,janet:2018}) indicam que humanos aprendem a generalizar padrões fonológicos baseado na contagem de \textit{word types}, ignorando a frequência de uso das palavras. Com isso, nesta pesquisa cada verbo é visto como um \textit{word type}. Isso significa que a frequência de uso dos verbos não foi levada em consideração para a introdução dos mesmos no modelo. Ao invés disso, os verbos são tratados de forma igualitária e compartilham do mesmo número de \textit{épocas}.\\


Nos três primeiros gráficos da figura, exibem-se três diferentes métricas de avaliação em relação ao número de épocas realizadas. 
Como se pode observar, a partir de aproximadamente 300 épocas, a curva de aprendizado começa a se estabilizar em torno de um valor. Também é possível ver a diferença entre os scores de treino (curva azul) e teste (curva laranja). Essa diferença mostra que os dados de treino não estão sendo suficientes para aumentar o score nos dados de teste. Desse modo, concluiu-se que 300 épocas seria o momento de conclusão do treinamento do modelo. Os resultados nessa nova escala podem ser observados na Figura \ref{fig:training}.  

\begin{figure}[H]
  \centering
  \includegraphics[width=1.0\linewidth]{img/2000_precision.png}
  \caption{Curva de Aprendizado para 2000 épocas}
  \label{fig:training2000}
\end{figure}

\begin{figure}[H]
  \centering
  \includegraphics[width=1.0\linewidth]{img/300_fbeta.png}
  \caption{Curva de apredizado para 300 épocas}
  \label{fig:training}
\end{figure}

Para o entendimento das métricas utilizadas, faz-se necessária uma recapitulação a respeito do que se espera como output do modelo. Para cada verbo, o modelo prediz uma sequência de $n$ vetores de tamanho 21 (o número total de traços fonéticos somado aos símbolos de início e final), aonde $n$ é o número de fones do verbo alvo e cada vetor corresponde a um fone. Esses 21 valores são arredondados para 0 ou 1 dependendo de um determinado limiar, que neste caso é 0.5. As métricas são calculadas comparando-se esses $n$ vetores de output com seus respectivos alvos.

\subsubsection{Métricas}

A \textbf{Precisão} é uma métrica calculada de acordo com a Equação \ref{eq:precision}. Nesta métrica, avalia-se a proporção de verdadeiros positivos ($vp$) em relação ao número total de positivos que o modelo previu, ou seja, dentre os traços fonéticos preditos pelo modelo, quantos estavam de acordo com os presentes no alvo.


\begin{align}\label{eq:precision}
\text{Precisão = } \frac{vp}{vp + fp}
\end{align}

O \textbf{Recall}, também conhecido como revocação, foi calculado a partir da Equação \ref{eq:recall}. Nesta métrica busca-se entender a proporção de traços fonéticos que o modelo previu (verdadeiros positivos) em relação a todos que ele deveria ter previsto (verdadeiros positivos + falsos negativos). 

\begin{align}\label{eq:recall}
\text{Recall = } \frac{vp}{vp + fn}
\end{align}

Por fim, o F1 Score (Eq. \ref{eq:f1}) é a média harmônica das duas métricas já explanadas.

\begin{center}
\begin{align}\label{eq:f1}
  \text{F1 Score} = 2 \times \frac{precision \times recall}{precision + recall}
  \end{align}
\end{center}

Todas as métricas variam no intervalo [0,1] sendo que valores próximos a 1 indicam alto desempenho do modelo.
Os resultados exibidos na Figura \ref{fig:training} mostram que a métrica Recall atinge scores mais baixos que a precisão. Isso significa que o modelo tem uma certa tendência a gerar outputs com mais valores nulos do que deveria. A Precisão atingiu valores próximos a 0.8. Isso indica que o modelo prevê aproximadamente 20\% dos traços de forma equivocada.

O último gráfico da figura é o gráfico do Erro, já explanado no Capítulo \ref{ch:03}. Em uma situação ideal a curva de erro do teste deve tender a zero, porém vê-se que o erro apresenta um comportamento de queda até em torno de 100 épocas e depois começa a subir gradualmente. 



\section{Resultados da Decodificação}
\label{sec:results}
Como o tamanho do corpus disponível é limitado, não é interessante realizar um único experimento com uma única segmentação entre treino e teste, pois haveriam verbos em que o modelo nunca seria testado e outros que nunca seriam utilizados no treinamento. Dessa forma a acurácia do modelo poderia variar bastante dependendo dos verbos sorteados para cada um desses grupos. Assim, uma técnica de validação cruzada chamada \textit{K-Fold} (\cite{kfold:2018}) foi escolhida para o estudo dos resultados. A análise K-Fold permite que o modelo seja avaliado em todos os verbos do corpus. A técnica consiste, primeiramente, na formação de K subconjuntos de verbos mutuamente exclusivos de tamanhos próximos (idênticos caso a divisão por K seja exata). Em seguida, um desses subconjuntos é escolhido como o conjunto de teste enquanto que os K-1 restantes são utilizados como treino. O tamanho de K é definido a partir da escolha de proporção em que se deseja realizar a segmentação entre treino e teste, ou seja, para sempre manter a proporção de testes em 20\% de modo que estes verbos sejam sempre diferentes, o corpus precisa ser segmentado em 5 subconjuntos distintos. O desenho \ref{fig:kfold} ilustra o algoritmo. 

\input{definitions/colors}
\input{definitions/styles}
\begin{figure}[ht!]
\centering

\scalebox{1.0}{
\begin{tikzpicture}[H]

%first
\node[box3] (box1) {};
\node[box2, right=0pt of box1] (box2) {};
\node[box2, right=0pt of box2] (box3) {};
\node[box2, right=0pt of box3] (box4) {};
\node[box2, right=0pt of box4] (box5);

%second
\node[box2, below=15pt of box1] (box6) {};
\node[box3, right=0pt of box6] (box7) {};
\node[box2, right=0pt of box7] (box8) {};
\node[box2, right=0pt of box8] (box9) {};
\node[box2, right=0pt of box9] (box10);

%third
\node[box2, below=15pt of box6] (box11) {};
\node[box2, right=0pt of box11] (box12) {};
\node[box3, right=0pt of box12] (box13) {};
\node[box2, right=0pt of box13] (box14) {};
\node[box2, right=0pt of box14] (box15);

%fourth
\node[box2, below=15pt of box11] (box16) {};
\node[box2, right=0pt of box16] (box17) {};
\node[box2, right=0pt of box17] (box18) {};
\node[box3, right=0pt of box18] (box19) {};
\node[box2, right=0pt of box19] (box20);

%fifth
\node[box2, below=15pt of box16] (box21) {};
\node[box2, right=0pt of box21] (box22) {};
\node[box2, right=0pt of box22] (box23) {};
\node[box2, right=0pt of box23] (box24) {};
\node[box3, right=0pt of box24] (box25);

%legenda
\node[box2, right=35pt of box10] (box26) {};
\node[textonly, right=5pt of box26] (box27) {Treino};
\node[box3, below=10pt of box26] (box28) {};
\node[textonly, right=5pt of box28] (box29) {Teste};



\end{tikzpicture}
}\caption{Técnica de segmentação do Corpus via K-Fold} 
\label{fig:kfold}
\end{figure}

Além disso, há ainda a vantagem do algoritmo de estratificação, ou seja, a possibilidade de que, em cada um dos treinamentos, as proporções das classes de verbos se mantenham no treinamento. Isso significa que, por exemplo, a classe de irregularidades e $\rightarrow$ \textepsilon / \_C*]+ar (testar $\rightarrow$ testo) que possui 20 verbos, terá sempre 16 verbos verbos no treinamento e 4 no teste. Assim, após os 5 treinamentos diferentes, todos os verbos da classe foram testados e há a certeza de que haviam verbos dessa classe no treinamento. 

Como há um fator de aleatoriedade devido às inicializações dos pesos, e também como os verbos de treino e teste são sorteados pelo algoritmo no momento da segmentação do K-Fold, os resultados da rede podem variar. Para uma melhor compreensão de como ocorre essa variação para cada classe, foram realizadas 30 análises K-Fold. Isso significa que foram realizados 30 $\times$ 5 = 150 treinamentos no total.
A acurácia mais alta obtida entre esses treinamentos foi de 17\% (73 verbos). As análises a seguir foram feitas no intuito de se entender melhor as situações de maior dificuldade do modelo.

\subsection{Acurácia em cada Classe}
\label{sec:prop}

Um fator que pode influenciar no resultado é o tamanho do corpus e o desbalanceamento das classes. Portanto, cabe verificar se há alguma relação entre a proporção da classe no corpus e a acurácia. Na Fig. \ref{fig:kfoldprop} é possível observar que as acurácias de algumas das classes de verbos seguem uma certa tendência linear de acordo com as suas respectivas proporções no corpus, enquanto que há um outro grupo de classes que apresenta acurácia mais baixa independentemente da proporção. O coeficiente de correlação de Pearson encontrado entre essas variáveis foi de 0.42. O valor de correlação obtido indica que devem existir outros fatores influenciando nos resultados.

\begin{figure}[H]
  \centering
  \includegraphics[width=0.8\linewidth]{img/proporxacc.png}
  \caption{Acurácia Ordenada pela Proporção das Classes}
  \label{fig:kfoldprop}
\end{figure}

Para entender os resultados obtidos, o experimento com a acurácia mais alta (vide apêndice) será utilizado como referência para a análise de erros.

Ao analisar os erros ocorridos na classe do verbo “Pôr“, verifica-se que praticamente um terço (8/26) dos erros cometidos foram erros de \textit{regularização}, ou seja, o modelo se confundiu com a classe mais presente (a classe dos verbos regulares) e manteve o padrão de flexão regular. 
Em seguida, a próxima classe com acurácia baixa fora da tendência é a classe \textit{Single}, os verbos sem agrupamento. Esse resultado era totalmente esperado pois não havia outros verbos no corpus para que o modelo conseguisse detectar possíveis padrões para fazer predições corretas. 

A classe do verbo “testar” apresentou muitas predições incoerentes, porém também houve erros de regularização (4/19). Chama a atenção que, para o verbo “pegar“, o modelo acertou a flexão irregular transformando a vogal anterior meio-fechada em meio-aberta (e $\rightarrow$ \textepsilon), porém por algum motivo ficou faltando o traço de vozeamento para caracterizar o fone “g” ao invés de “k”, o que resultou na flexão equivocada de pegar $\rightarrow$ pEku.
Tal erro sugere uma revisão quanto a métrica acurácia para essa tarefa, uma vez que o erro do modelo não ocorreu no processo de flexão.

A classe do verbo “vir”, por sua vez, apresentou 5/11 de erros por regularização, os demais parecem erros completamente incompreensíveis.

A classe do verbo “ansiar” também parece estar fora da tendência, apresentou 7 erros incompreensíveis e por pouco acertou a flexão do verbo “passear“, porém substituiu a consoante “s” por “r“, ambas alveolares. 

A classe do verbo “dizer” apresentou apenas erros incompreensíveis e a classe do verbo “pedir” acertou a flexão irregular no verbo “pedir” porém não transformou o fone “d” em “s” (alveolar - surda - fricativa). Ao invés disso, previu “t” (alveolar - surda - oclusiva). Novamente, errou por causa de apenas de um traço fonético que não está em estudo.

Os erros observados motivaram o estudo de outras variáveis em relação a acurácia. Não obstante, a métrica de acurácia parece inadequada visto que em muitos casos o modelo foi capaz de identificar a flexão apropriada, porém se confundiu em um traço fonético que está fora de avaliação.

% \subsection{A relação entre acurácia e a perplexidade de cada classe}

% Outra variável que poderia influenciar nos resultados seria uma maior variabilidade, ou imprevisibilidade, de algumas classes em relação a outras. Suponha que exista uma classe na qual todos os verbos compartilhem do mesmo comprimento e variações mínimas, por exemplo:

% \begin{center}
%     akabar - abalar - akalar - asolar
% \end{center}

% Essa classe, por conter pouca variabilidade de fones, poderia ter vantagem em relação a uma outra em que há maior variabilidade, como por exemplo:

% \begin{center}
%     dezenkontrar - estourar - akal3ntar - sust3ntar - axar
% \end{center}

% Para quantificar as previsibilidades das classes, foi utilizada a medida de perplexidade. Em teoria da informação (ref), a perplexidade é uma medida do quão bem um modelo de probabilidade prevê uma amostra. Por esse motivo ela é bastante utilizada como referência para a comparação entre diferentes modelos de linguagem. Uma baixa perplexidade indica que o modelo proposto é bom em prever a amostra. O gráfico na Figura \ref{fig:kfoldperp} mostra a relação entre a acurácia do modelo e a perplexidade das classes, obtida a partir de um modelo de trigramas.  %A equação (tal) exibe a fórmula para o cálculo da métrica.


% \begin{figure}[H]
%   \centering
%   \includegraphics[width=0.8\linewidth]{img/perplexidade.png}
%   \caption{Acurácia por Perplexidade}
%   \label{fig:kfoldperp}
% \end{figure}

% Nota-se que a relação entre as duas variáveis é menos clara, com um coeficiente de correlação de Pearson = 0.38. 

\subsection{Acurácia e Comprimento Médio das Classes}

Outra questão que poderia influenciar na performance do modelo é o número médio de fones dos verbos na classe.  O verbo “3ntret3Nu” (entretenho), por exemplo, contém 9 fones para serem preditos. Como cada fone contém 21 traços, no total o modelo tem que conseguir prever 189 números. O verbo “falu” (falo), em contrapartida, possui 4 fones, ou seja, 84 números a serem preditos. Dessa forma, é natural pensar que quanto maior o número de predições necessárias, maior a possibilidade do modelo cometer algum erro. Apesar disso, a Figura \ref{fig:kfoldprop} não mostra uma relação nítida entre a acurácia e o comprimento médio com uma correlação de Pearson = 0.39. Apesar disso observa-se que as barras de confiança do gráfico de comprimento médio não exibem muita diferença, indicando homogeneidade entre as classes nesse aspecto.

\begin{figure}[H]
  \centering
  \includegraphics[width=0.8\linewidth]{img/comp_acc.png}
  \caption{Acurácia por Comprimento Médio}
  \label{fig:kfoldprop}
\end{figure}


% colocar no apendice os resultados
\section{Outros Erros Relevantes}
\label{sec:interesting}

Alguns erros interessantes como troca de classes e regularizações já foram apresentados na Seção \ref{sec:prop}, mas ainda há outros erros que merecem destaque. (Ver Apêndice para tabela completa com Classe, Input, Output e Alvo)

Na classe dos verbos \textit{Single}, um erro notável foi a flexão realizada para o verbo trazer (“trasu”), o que mostra que o modelo identificou o padrão de flexão da classe do verbo “fazer”. 

Na classe do verbo “pedir”, apresentou a flexão “espidu” para o verbo “espedir”, o que denota uma possível confusão com a classe do verbo “conseguir”.

Nos verbos regulares nota-se a presença de vários erros devido a troca de apenas um traço fonético. Como por exemplo para o verbo “convidar”, o output resultante foi “konviru”. Para o verbo “convencer”, “konfensu”.

Para os verbos da classe do verbo “seguir“, três verbos foram flexionados de acordo com a família do verbo “testar“. São eles: “ferir” (fEru), “vestir” (vEstu) e “repetir” (hepEtu).

A classe do verbo “ver” também apresentou erros interessantes: a confusão com a classe do verbo “vir” nos verbos “prever” (“preveNu”) e “entrever” (“entreveNu”). Também por pouco não acertou o alvo no verbo “rever” (hefexu), ficou faltando o traço de vozeamento nas duas últimas consoantes.

\section{Discussão}

Os gráficos apresentados na Fig. \ref{fig:training} e os resultados apresentados em seguida indicam que o modelo aparenta estar em \textit{overfit}. Uma justificativa para isso talvez seja a complexidade do modelo encoder-decoder para uma baixa quantidade de verbos adquiridos no Corpus. Um estudo semelhante realizado por \cite{kirov:2018} mostra um modelo do tipo encoder-decoder treinado em 4039 verbos da língua inglesa para a tarefa de aprendizado de verbos irregulares. Tal estudo resultou em uma acurácia de quase 100\%, porém com um dataset 10 vezes maior.

Para tentar balancear as classes de verbos irregulares, foi realizado um treinamento com uma técnica conhecida como \textit{oversampling} (ref) que consiste na replicação de exemplos das classes até que todas possuam o mesmo número para o treinamento. No entanto o experimento não resultou em um melhor desempenho do modelo. 




% \begin{itemize}
%     \item falar do modelo de atenção
%     \item grid search
% \end{itemize}



\chapter{Conclusão}
\label{ch:08}

Este trabalho teve como principal objetivo estudar o aprendizado de verbo irregulares do Português Brasileiro utlizando um modelo do tipo \textit{Encoder-Decoder} (\cite{enc-dec:2014}, \cite{seq2seq:2014}). 

O escopo da pesquisa foi restrito à 1$^{a}$ pessoa do singular, no tempo presente e modo indicativo. Em seguida, um corpus com 423 verbos foi criado e transcrito para uma representação fonética utilizando a metodologia desenvolvida no Capítulo \ref{ch:02}. O corpus criado foi primeiramente particionado segundo classes de irregularidade. No total, quinze classes irregulares foram formadas. A proporção de verbos regulares e irregulares no corpus montado foi de respectivamente 50.6\% e 49.4\%. Após a coleta e organização dos verbos, os mesmos foram pré-processados para serem introduzidos na camada de \textit{input} do modelo. 

O modelo \textit{Encoder-Decoder} foi configurado da seguinte maneira: As redes \textit{Encoder} e \textit{Decoder} são ambas redes recorrentes (do tipo LSTM) com uma camada escondida de 256 nós. Para o treinamento foram utilizadas 300 épocas com lotes de 128 verbos por vez. O modelo foi otimizado com o algoritmo \textit{Adam} disponível na API do \textit{Keras} (\cite{chollet2015keras}) com hiperparâmetros pré-definidos pela API e disponíveis na documentação. A função de custo utilizada foi a Entropia Cruzada Binária (\cite{francois2017deep}).

Para a avaliação do modelo construído, foi utilizada a técnica de validação cruzada chamada \textit{K-fold} (\cite{kfold:2018}), que permite que todos os verbos do corpus sejam testados pelo modelo. A métrica de avaliação escolhida foi a acurácia. Para considerar as variações possíveis durante os treinamentos, o algoritmo K-fold foi executado trinta vezes. Desse modo, foi possível estudar o comportamento médio do modelo nas diferentes classes do estudo.

A acurácia máxima atingida pelo modelo foi de 17\% considerando todos os verbos do corpus ($73/423$). Considerando apenas verbos regulares \textit{versus} verbos irregulares, o desempenho do modelo foi melhor no primeiro grupo, sendo as respectivas acurácias médias 17.88\% e 9.23\%. Considerando-se todas as classes do estudo, destaca-se a classe do verbo “botar” que, além de possuir alta proporção de exemplos no corpus, é também a classe com menor comprimento médio. Ainda, foram observados alguns erros interessantes de troca de famílias nos verbos irregulares, como por exemplo: repetir (hepeti $\rightarrow$ hepEtu), uma possível confusão com o grupo do verbo \textit{testar}. Também ocorreram alguns erros de super-regularização (32 erros no total). 

Durante a seção de discussão (Seç. \ref{sec:discuss}), observamos que o tamanho do corpus obtido mostrou-se incompatível com a arquitetura \textit{Encoder-Decoder}. Por outro lado, observa-se que os módulos desenvolvidos de pré e pós processamento dos verbos são independentes da arquitetura proposta. Nesse sentido, pode-se argumentar que o resultado obtido nesta pesquisa não é conclusivo quanto ao desempenho do \textit{Encoder-Decoder}. \\

Para pesquisas futuras, ficam algumas sugestões:

\begin{enumerate}

\item Obtenção de um corpus maior para a língua portuguesa
para que o algoritmo apresentado nesta pesquisa possa ser reavaliado;

\item Desenvolvimento de novos algoritmos de pré e pós processamento dos verbos. Outras representações vetoriais%\footnote{Uma opção é adicionar uma camada de \textit{embedding} (\cite{Jurafsky:2009:SLP:1214993}) antes da camada de \textit{input} do \textit{Encoder}}
 podem ser desenvolvidas; 

\item Construção de um modelo do tipo \textit{Transformer} (\cite{Vaswani2017AttentionIA}). A arquitetura \textit{Transformer} é considerada a nova arquitetura estado-da-arte para modelos de tradução automática. 

\end{enumerate}

Também fica como sugestão a realização de um teste psicolinguístico com verbos irregulares inventados para uma comparação entre as predições do modelo e as opiniões de falantes da língua, retomando assim o caráter cognitivo da questão. 

Para concluir, esperamos que a presente pesquisa tenha apresentado contribuições para o tema do aprendizado de verbos irregulares dentro do domínio dos modelos de Redes Neurais Artificiais. 
\clearpage
\appendix
\chapter{Tabela de Transcrição Fonética de Rumelhart e McClelland}
\chapter{Corpus}
\chapter{Resultados das Decodificações}
% \includepdf[pages=-]{results_analysis_by_family.pdf}





% cabeçalho para os apendices
\renewcommand{\chaptermark}[1]{\markboth{\MakeUppercase{\appendixname\ \thechapter}} {\MakeUppercase{#1}} }
\fancyhead[RE,LO]{}
\appendix

% \chapter{Apêndice de Resultados}
\label{ch09-appendice}

Tabelas com todos os resultados


COlocar o corpus
 pensar sobre isso aqui

% \include{ape-previous-work}

% ---------------------------------------------------------------------------- %
% Bibliografia
\setcitestyle{square}
\backmatter \singlespacing   % espaçamento simples
\bibliographystyle{marcos4} % citação bibliográfica textual
\bibliography{references}  % associado ao arquivo: 'bibliografia.bib'

% ---------------------------------------------------------------------------- %


% \printindex   % imprime o índice remissivo no documento 

\end{document}
