\chapter{Conclusão}
\label{ch:08}

Este trabalho revisitou a controversa questão do processo de aquisição de verbos irregulares sob a perspectiva do Português Brasileiro e o fez através de uma das técnicas de redes neurais mais recentes na literatura, o \textit{Encoder-Decoder} (ref). 

A abordagem consistiu na restrição do escopo da pesquisa para a 1$^{a}$ pessoa do singular, no tempo presente e modo indicativo. Em seguida, um corpus com 423 verbos foi criado e transcrito para uma representação fonética (utilizando a metodologia explicada no Capítulo \ref{ch:02}). Na sequência, foi realizado um levantamento bibliográfico a respeito dos modelos de redes neurais para a identificação das arquiteturas mais recentes e adequadas para a tarefa em questão. Quando do início deste projeto, a arquitetura \textit{estado-da-arte} em tarefas de mapeamento de sequências era o \textit{Encoder-Decoder} (ref). O corpus foi então tratado e pré-processado de modo a ser compatível com os parâmetros de entrada e saída do modelo. 

A acurácia máxima atingida pelo modelo foi de 17\% (73 acertos em 423 verbos). Apesar disso, discute-se no Capítulo de Resultados (\ref{ch:07}) a adequabilidade da métrica de acurácia para esta tarefa visto que em diversos momentos o modelo apresentou predições corretas quanto às transformações irregulares, mas errou na predição de traços fonéticos não relacionados com a tarefa em questão. Notou-se também que a classe irregular com melhor desempenho, além de ser aquela com mais exemplos no Corpus, é também a classe com menor comprimento médio (o $\rightarrow$ \textopeno/\_C*]ar). Ainda, foram observados alguns erros interessantes de troca de famílias nos verbos irregulares, como por exemplo: repetir (hepeti $\rightarrow$ hepEtu). Também ocorreram muitos erros por regularização (x) e alguns erros (x2) por conta de um traço fonético não identificado (ver Apêndice). 

Ficam também, como contribuição, algumas sugestões para pesquisas futuras: (i) Construção de um modelo do tipo \textit{Transformer} (ref) para melhorar o mapeamento das sequências; (ii) Realização de um teste psicolinguístico com verbos “irregulares” inventados para uma comparação entre o modelo e indivíduos; e por fim, (iii) Identificação de uma distribuição de frequência real com relação ao uso dos verbos do Português Brasileiro para alimentar o modelo de acordo. 

Para concluir, embora o objetivo principal do projeto não tenha sido o desenvolvimento ou defesa de uma teoria conexionista, os experimentos realizados resultaram em resultados interessantes do ponto de vista linguístico e podem contribuir para o fomento da discussão a respeito do tema.

% Sendo assim, concluímos que a presente pesquisa apresentou importantes contribuições para o tema do aprendizado de verbos irregulares dentro do domínio dos modelos de Redes Neurais,



%\begin{itemize}

%     \item melhorar a representacao fonética de modo que nao houvesse a possibilidade de ter combinações impossíveis de features
%     reducao de dimensoes 
%     \item falar de type e token
%     \item sugestões de pesquisas futuras
%      \item Falar sobre experimentos que poderiam ter sido feitos mas não foram, ex: treinar a rede de acordo com uma frequência mais realista. Experimentos psicolonguisticos, talvez experimentos com penalizações.
%      \item introduzir que a acurácia nao foi muito boa e que apesar disso tento entender os resultados dos modelos
%     \item 
% \end{itemize}




